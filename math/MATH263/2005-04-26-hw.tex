\documentclass[10pt]{article}

\usepackage{homework}
\usepackage{enumerate}
\usepackage{geometry}

\geometry{letterpaper}

\textwidth = 6.5 in
\textheight = 9 in
\oddsidemargin = 0.0 in
\evensidemargin = 0.0 in
\topmargin = 0.0 in
\headheight = 0.0 in
\headsep = 0.0 in
\parskip = 0.2in
\parindent = 0.0in

\title{MATH 263: Homework \#4}
\author{Jesse Farmer}
\date{26 April 2005}
\begin{document}
\maketitle

\begin{enumerate}
\item \emph{Find spaces whose fundamental groups are isomorphic to the following groups.}
\begin{enumerate}
\item \emph{$\Z/n\Z \times \Z/m\Z$}

The quotient space induced by the covering map $S^1 \rightarrow S^1$ given by $z \mapsto z^n$ has a fundamental group isomorphic to $\Z/n\Z$.  This can be seen because $S^1$ is path-connected and, taking $1$ as the base point, we have that there is a bijection between $\Z / p_\ast(\Z)$ and $p^{-1}(1)$.  Since each fiber has four elements and $p_\ast(\Z)$ is a subgroup of $\Z$ (since $p_\ast$ is easily seen to be a homomorphism), it follows that $\pi_1(p(S^1), 1) = Z/n\Z$.

Writing $z \mapsto z^n$ as $p_n$ then $p_n(S^1) \times p_m(S^1)$, i.e., the cartesian product of the quotient spaces of $S^1$ under $p_n$ and $p_m$ respectively, has a fundamental group isomorphic to the direct product of their respective fundamental groups, viz., $\Z/n\Z \times \Z/m\Z$.

\item \emph{$\Z/n_1\Z \times \Z/n_2\Z \times \cdots \times \Z/n_k\Z$}

By induction it is easy to see that $p_{n_1}(S^1) \times \cdots \times p_{n_k}(S^1)$ is such a space.
\item \emph{$\Z/n\Z \ast \Z/m\Z$}

Using the exact argumentation as above it follows that the wedge product of two circles under the quotient maps $p_n$ and $p_m$, respectively, have a free product of $\Z/n\Z \ast \Z/m\Z$ by van Kampen.

\item \emph{$\Z/n_1\Z \ast \Z/n_2\Z \ast \cdots \ast \Z/n_k\Z$}

Again, inductively, the space $\bigvee_{i=1}^k p_{n_k}(S^1)$ is such a space.
\end{enumerate}

\item \emph{Find a presentation for the fundamental group of $\P^2 \# T$.}

$\pi_1(\P^2)$ has a presentation of $\brac{a \mid a^2}$ and $\pi_1(T)$ has a presentation of $\brac{b,c \mid bcb^{-1}c^{-1}}$.  Since $\P^2 \# T$ can be viewed as the $6$-sided polgon with labeling $aabcb^{-1}c^{-1}$ it follows that $\brac{a,b,c \mid aabcb^{-1}c^{-1}}$ is a presentation for $\pi_1(\P^2 \# T)$.

\item \emph{Let $X$ be the space obtained from a seven-sided polygonal region by means of the labelling scheme $abaaab^{-1}a^{-1}$.  Show that the fundamental group of $X$ is the free product of two cyclic groups.}

The fundamental group of $X$ is isomorphic to $\Z \ast \Z/3\Z$.  This can be seen by demonstrating a presentation of $X$.  From 68.7, the fundamental group is isomorphic to the free product of $\Z/N_1 \ast \Z/N_2$ where $N_1$ and $N_2$ are two normal subgroups such that the normal subgroup $N$ generated by $abaaab^{-1}a^{-1}$ is the smallest containing both of them.  But in this case, we can write the labelling scheme as $aaab^{-1}a^{-1}ab$ so that $N_2$ is generated by $aaa$ and $N_2$ is generated by $b^{-1}a^{-1}ab = 1$.  Hence $N_1 = \{1\}$ and $N_2 = 3\Z = \brac{a \mid a^3}$.

\item \emph{Let $K$ be the Klein Bottle.}
\begin{enumerate}
\item \emph{Find a presentation for the fundamental group of $K$.}

The Klein Bottle is homeomorphic to a square with labeling $baba^{-1}$, and under this quotient map all vertices are identified, so it has a presentation of $\brac{a,b \mid baba^{-1}}$.

\item \emph{Find a double covering map $p:T \rightarrow K$ where $T$ is the torus.  Describe the induced homomorphism of fundamental groups.}
\end{enumerate}

\item
\begin{enumerate}
\item \emph{Show that the Klein Bottle $K$ is homeomorphic to $\P^2 \# \P^2$.}

Recall that the fundamental group of the Klein Bottle has a presentation of $\brac{a,b \mid baba^{-1}}$.  Cutting this along the diagonal with a line, labeled $c$, and gluing them together along the line labeled $b$ gives a quotient space with labeling $ccaa$.  Hence, under this quotient map, the fundamental group of the Klein Bottle has presentation $\brac{c,a \mid ccaa}$, which is also the presentation of $\P^2 \# \P^2$.

\item \emph{Show how to picture the $4$-fold projective plane as an immersed surface in $\R^3$.}

The $4$-fold projective plane can be immersed in $\R^3$ by taking two Klein bottles, cutting a small hole in each surface, and identifying the boundaries of the two holes.  In other words, the immersion is simply $K \# K$.

\end{enumerate}

\item \emph{Let $X$ be the quotient space of $S^2$ obtained by identifying the north and south poles to a single point.  Put a cell complex structure on $X$ and use this to compute $\pi_1(X)$.}

First, $X$ decomposes into an open-ended clinder and a distinct point with the boundaries of the circles at the end of the cylinder identified with the point.  The cylinder deformation retracts to $S^1$, so that $X$ is in fact a $CW$-complex consisting of one $0$-cell and one $2$-cell, i.e., the circle.  Hence $\pi_1(X) = \Z$.

This can also be seen by considering the space $X$ as the wedge of $S^2$ and $S^1$.  Since $S^2$ has trivial fundamental group, $\pi_1(X) \iso \set{0} \iso \Z = \Z$.

\item \emph{Compute the fundamental group of the space optained from two tori $S^1 \times S^1$ by identifying the circle $S^1 \times \set{x_0}$ in one torus with the corresponding circle $S^1 \times \set{x_0}$ in the other torus.}

Write the first torus as $\brac{a,b \mid aba^{-1}b^{-1}}$ and the second as $\brac{c,d \mid cdc^{-1}d^{-1}}$.  Identify the circle $b$ with the circle $d$.  Then this becomes a rectangle with labeling $(ac)b(ac)^{-1}d^{-1}$, but as $d = b$, this is really $(ac)b(ac)^{-1}b^{-1}$.  Hence the fundamental group is again that of a torus, viz., $\Z \times \Z$, which can be seen by writing the group presentation as $\brac{e,f \mid efe^{-1}f^{-1}}$ for $e = ac$ and $f = b = d$.

\item \emph{Consider the quotient space of a cube $I^3$ obtained by identifying each square face with the oppositce square face via the right-handed screw motion consisting of a translation by one unit in the direction perpendicular to the face combined with a one-quarter twist of the face about its center point.  Show this quotient space $X$ is a cell complex with two $0$-cells, four $1$-cells, three $2$-cells, and one $3$-cell.  Using this show that $\pi_1(X) \iso Q_8$, the quaterntion group of order eight.}

\item \emph{Let $X$ bt the subspace of $\R^2$ that is the union of the circles $C_n$ of radius $n$ and center $(n,0)$ for $n \in \N$.  Show that $\pi_1(X)$ is the free group $\coprod_{n \in \N} \pi_1(C_n)$, the same as for the infinite wedge sum $\vee_\infty S^1$.  Show that $X$ and $\vee_\infty S^1$ are in fact homotopy equivalent but not homeomorphic.}

\item
\begin{enumerate}
\item \emph{Show that $\Hom_G(H \bs G,K \bs G)$ can be identified with $K \bs \set{g \in G \mid gHg^{-1} \subset K}$.}

Every homomorphism of $G$-sets $\varphi \in \Hom_G(H \bs G,K \bs G)$ is completely determined by the coset to which it sends $He \in H \bs G$.  In otherwords, define $\varphi_a(xH) = xaK$.  Then for $h \in H$, 
\[
Ka = \varphi_a(He) = \varphi(Hh) = Kah
\]

so that $aha^{-1} \in K$, i.e., $aHa^{-1} \subset K$.  Each such homomorphism is unique and every homomorphism is of this form (since it must send $He$ to something).  In other words, the map $a \mapsto \varphi_a$ is a bijection between $K \bs \set{g \in G \mid gHg^{-1} \subset K}$ and $\Hom_G(H \bs G,K \bs G)$.

\item \emph{Show that $Hom_G(H \bs G,H \bs G) = N_G(H) / H$.}

This follows directly from the previous part since by definition $N_G(H) = \set{g \in G \mid gHg^{-1} \subset H}$.  Since $N_G(H)$ is the largest subgroup of $G$ in which $H$ is normal, $N_G(H) / H = H \bs N_G(H)$.

\end{enumerate}
\end{enumerate}
\end{document}
