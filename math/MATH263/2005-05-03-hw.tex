\documentclass[10pt]{article}

\usepackage{homework}
\usepackage{enumerate}
\usepackage{geometry}

\geometry{letterpaper}

\textwidth = 6.5 in
\textheight = 9 in
\oddsidemargin = 0.0 in
\evensidemargin = 0.0 in
\topmargin = 0.0 in
\headheight = 0.0 in
\headsep = 0.0 in
\parskip = 0.2in
\parindent = 0.0in

\title{MATH 263: Homework \#5}
\author{Jesse Farmer}
\date{03 May 2005}
\begin{document}
\maketitle

\begin{enumerate}
\item \emph{Show that $M$, the M\"{o}bius band, is homeomorphic to the punctured real projective plane.}

Recall that the Klein bottle $K$ is $\P^2 \# \P^2$.  Write $K$ as a square with labeling $aba^{-1}b$.  Trisect this square into three parts of equal height.  Then the middle section is a M\"{o}bius band, as are the upper and lower sections since the top and bottom edges are identified.  The two lines trisecting the square form the boundary of the M\"{o}bius bands, and, moreover, constitute a circle by the identification alone the left and right edges of $K$.  Hence $K$ is constructed by taking two M\"{o}bius bands and identifying their boundaries with a circle.  But recall that $K = \P^2 \# \P^2$, which is formed by removing two contractible neighborhoods from each $\P^2$ and identifying their boundaries, which are homotopic to $S^1$.  In terms of $CW$-complexes, it then follows that a punctured $\P^2$ is homeomorphic to $M$.

\item \emph{For $n > 1$ show that the fundamental group of the $n$-torus is not abelian.}

\begin{lemma}
\label{kernel_extension}
Let $f: G \rightarrow H$ be a homomorphism of groups.  If $I \subseteq \ker f$ is normal in $G$ then there exists a homomorphism $\tilde{f}: G/I \rightarrow H$ such that $\tilde{f} \circ p = f$, where $p: G \rightarrow G/I$ is the natural homomorphism.
\end{lemma}
\begin{proof}
Define $\tilde{f}(a + I) = f(a)$.  It suffices to show that $\tilde{f}$ is well-defined, since if it were then it is immediately a homomorphism by the fact that $f$ is a homomorphism.  Let $a + I = b + I$, so that $a-b \in I$.  Then since $I \subseteq \ker f$,
\[
0 = f(a-b) = f(a) - f(b) = \tilde{f}(a+I) - \tilde{f}(b+I)
\]

and therefore $\tilde{f}(a+I) = \tilde{f}(b+I)$.
\end{proof}

Let $G$ be the free group on $2n$ generators $\set{\alpha_1, \beta_1, \ldots, \alpha_n, \beta_n}$ and $H$ the free group on $2$ generators, $\set{\gamma, \delta}$.  Define a homomorphism $\varphi: G \rightarrow H$ by sending $\alpha_1, \beta_1$ to $\gamma$ and all other generators of $G$ to $\delta$.  The fundamental group of the $n$-fold torus is isomorphic to $G$ modulo the least normal subgroup $N$ generated by $[\alpha_1, \beta_1] \cdots [\alpha_n, \beta_n]$ where $[\alpha_i, \beta_i] = \alpha_i\beta_i\alpha_i^{-1}\beta_i^{-1}$.  However, the image of $[\alpha_i, \beta_i]$ under $\varphi$ is the empty word since for fixed $i$, $\alpha_i, \beta_i$ map to the same element in $F$.  Hence $N \subseteq \ker \varphi$ and by \eqref{kernel_extension} there exists a well-defined homomorphism from $G/N$ to $H$.

But $G/N \iso \pi_1(T_n)$, where $T_n$ denotes the $n$-fold torus.  Clearly $\varphi$ (and therefore $\tilde{\varphi}$) is surjective.  If $\pi_1(T_n)$ were abelian then $\pi_1(T_n) / \ker \tilde{\varphi}$ would also be abelian and isomorphic to $H$, which is impossible since $Z(H) = 0$.  Therefore $\pi_1(T_n)$ cannot be abelian.

\item \emph{For $m > 1$ show that the fundamental group of the $m$-fold projective plane is not abelian.}

Let $\P_m$ denote the $m$-fold torus.  Let $G$ be the free group on $m$ generators, $\set{\alpha_1, \ldots, \alpha_n}$, and let $H = \Z/2\Z \ast \Z/2\Z$, generated by $\set{\gamma, \delta}$.  Define a homomorphism $\varphi: G \rightarrow H$ by sending $\alpha_1$ to $\gamma$ and $\alpha_i$ to $\delta$ for $i > 1$.  Then $\pi_1(\P_m) \iso G/N$ where $N$ is the least normal subgroup generated by $\alpha_1^2 \cdots \alpha_m^2$.  But clearly $\varphi$ sends this element to the empty word, as $x^2$ is trivial for all $x \in \Z/2\Z$.  Again by \eqref{kernel_extension}, there exists a surjective homomorphism $\tilde{\varphi}: \pi_1(\P_m) \rightarrow H$.  As above this is impossible, since the first isomorphism theorem would them imply that $H$ is abelian.  Therefore $\pi_1(\P_m)$ is not abelian.

\item \emph{Calculate $H_1(\P^2 \# T)$.  Assuming that the list of compact surfaces given in Theorem 75.5 is a complete list, to which of these surfaces if $\P^2 \# T$ homeomorphic?}

$H_1(\P^2 \# T)$ is isomorphic to the free abelian group $F$ with three generators, call them $a,b,c$, generated by the element $a^2bcb^{-1}c^{-1}$.  Since $F$ is abelian, the element $bcb^{-1}c^{-1}$ is the identity element, and $p(a^2) = p(a)^2$.  Hence $H_1(\P^2 \# T) \iso \frac{\Z \times \Z \times \Z}{2 \Z} \iso \Z \times \Z \times \Z/2\Z$.  Assuming the list is complete, this would imply that $\P^2 \# T$ is homeomorphic to $\P_3$, the $3$-fold projective plane.

\item \emph{Let $K$ be the Klein Bottle.  Calculate $H_1(K)$ directly.}

Recall that $K = \brac{a,b \mid aba^{-1}b}$.  Hence $H_1(K)$ is isomorphic to the free abelian group on two generators generated by the element $aba^{-1}b$.  Since the group is abelian, $aba^{-1}b$ becomes $a^2$, and it follows that $H_1(K) \iso \Z \times \Z/2\Z$.

\item \emph{Let $X$ be the quotient space obtained from an $8$-sided polygonal region $P$ by pasting its edges together according to the labeling scheme $acadbcb^{-1}d$.}

\begin{enumerate}
\item \emph{Check that all vertices of $P$ are mapped to the same point of the quotient space $X$ by the pasting map.}

Since I'm not drawing this, you get to see it written out.  In all instances ``tip'' and ``end'' mean relative to the direction in the labeling scheme.  The tip of $d$ is connected to the end of $a$, which is also connected to the tip of $c$ and the end of $b$.  But the tip of $b$ is connected to both the tip and end of $c$, and this suffices since the other vertices belong to edges already considered.


\item \emph{Calculate $H_1(X)$.}

Under the projection map from a free group on $4$ generators to its abelianization, $acadbcb^{-1}d$ becomes $a^2c^2d^2$.  Hence $H_1(X)$ is isomorphic to the abelian group on $4$ generators modulo the subgroup generated by $a^2c^2d^2$.  Rewriting the basis as $a,b,c,(a+c+d)$ shows that this is, in fact, the free abelian group on $4$ generators modulo the subgroup generated by $2(a+c+d)$.  In other words, $H_1(X) \iso \Z \times \Z \times \Z \times \Z/2\Z$.
\item \emph{Assume $X$ is homeomorphic to one of the surfaces given in Theorem 75.5, which surface is it?}

$X$ would seem to be homeomorphic to $\P_3$, the $3$-fold projective plane, since it has a torsion subgroup of rank $2$ and the quotient with this subgroup of rank $3$.

\end{enumerate}

\end{enumerate}
\end{document}
