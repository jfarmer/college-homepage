\documentclass[11pt]{article}
\textwidth = 6.5 in
\textheight = 9 in
\oddsidemargin = 0.0 in
\evensidemargin = 0.0 in
\topmargin = 0.0 in
\headheight = 0.0 in
\headsep = 0.0 in
\parskip = 0.2in
\parindent = 0.0in
\usepackage{amsfonts}
\usepackage{amsmath}
\usepackage{amssymb}

\title{MATH 208: Homework \#4}
\author{Jesse Farmer}
\date{02 February 2004}
\begin{document}
\maketitle
\begin{enumerate}

\item \emph{Prove that $\mathbb{Q}[x]$ is dense in $C[0,1]$.}

\[
p(x) = a_nx^n + a_{n-1}x^{n-1} + \cdots + a_1x + a_0
\]
with $a_n,\ldots,a_0 \in \mathbb{R}$.  And

\[
q(x) = b_nx^n + b_{n-1}x^{n-1} + \cdots + b_1x + b_0
\]

with $b_n,\ldots,b_0 \in \mathbb{Q}$.

By the density of $\mathbb{Q}$ in $\mathbb{R}$, choose each $b_i$ such that $|a_i - b_i| < \frac{\epsilon}{n}$.  Since $x \in [0,1]$ we have
\begin{eqnarray*}
|p(x) - q(x)| &=& |a_nx^n + a_{n-1}x^{n-1} + \cdots + a_1x + a_0 - (b_nx^n + b_{n-1}x^{n-1} + \cdots + b_1x + b_0)| \\
&=& |(a_n-b_n)x^n + (a_{n-1}-b_{n-1})x^{n-1} + \cdots + (a_1-b_1)x + a_0-b_0| \\
&\leq& |a_n-b_n|x^n + |a_{n-1}-b_{n-1}|x^{n-1} + \cdots + |a_1-b_1|x + |a_0-b_0| \\
&\leq& |a_n-b_n| + |a_{n-1}-b_{n-1}| + \cdots + |a_1-b_1| + |a_0-b_0| \\
&<& \epsilon
\end{eqnarray*}

By Stone-Weierstrass we know that $\mathbb{R}[x]$ is dense in $C[0,1]$, so it follows that $\mathbb{Q}[x]$ is also dense in $C[0,1]$.

\item \emph{Let $\{T_j: V \rightarrow W\}$ be a sequence of bounded linear maps such that $T_j \rightarrow T$ pointwise.  Show that $T$ is linear.}

We know that $T(v_1+v_2) = lim_{n \rightarrow \infty}T_n(v_1+v_2)$, so all that remains to be shown is that $lim_{n \rightarrow \infty}T_n(v_1+v_2) = T(v_1) + T(v_2)$.

Pick $N_1,N_2$ such that for $n,m \geq N$
\begin{eqnarray*}
|T_n(v_1) - T(v_1)| &<& \frac{\epsilon}{2} \\
|T_m(v_1) - T(v_1)| &<& \frac{\epsilon}{2}
\end{eqnarray*}

If we pick $N = \max\{N_1,N_2\}$ it follows that for all $n \geq N$
\begin{eqnarray*}
|T_n(v_1 + v_2) - (T(v_1) + T(v_2))| &=& |T_n(v_1) + T(v_2) - (T(v_1) + T(v_2))| \\
&\leq& |T_n(v_1) - T(v_1)| + |T_n(v_2) - T(v_1)| \\
&<& \epsilon
\end{eqnarray*}

We can simply pick $N$ such that $|T_n(\alpha v) - \alpha T(v)| = |\alpha T_n(v) - \alpha T(v)| < \epsilon$.  Therefore $T$ is linear.

\item \emph{Let $A$ be a $n \times n$ matrix and $T: \mathbb{R}^n \rightarrow \mathbb{R}^n$ its corresponding linear map.  Compute $\|T\|$ if $\mathbb{R}^n$ is endowed with the $p-norm$.}

\item \emph{Find a topological basis for $l^\infty(F)$ under the $l^\infty$ norm.}

Let $V = \left\{(a_n) \mid a_i \in \{0,1\}\right\}$.  This should work, as we can approximate values arbitrarily far out in the sequence with just one basis element, but I'm too tired to concentrate.

\item \emph{Find a topological basis for $C([0,1],F)$ under the $sup$ norm.}

Since we're taking $F = \mathbb{R}, \mathbb{C}$, and we have proved the first already, take $F = \mathbb{C}$.

Every $f \in C([0,1],\mathbb{C})$ can be written as $f = u + iv$ for $u,v \in C([0,1],\mathbb{R})$.  We can pick $p,q \in \mathbb{R}[x]$ such that $|p - u| < \frac{\epsilon}{2}$ and $|q - v| < \frac{\epsilon}{2}$ by the Stone-Weierstrass  Then $|(u-p) - i(v-q)| \leq |u-p| + |v-q| < \epsilon$.  Additionally, by the first problem, we can use $\mathbb{Q}[x]$ instead of $\mathbb{R}[x]$.

\item \emph{Does Stone-Weierstrass provide a topological basis for some well-known normed linear spaces?}

\item \emph{Show that $C([0,1],F)$ is not complete under the $\|\cdot\|_1$ norm, where $\|f\|_1 := \int_0^1|f(x)|dx$.}

\item \emph{Find the dual spaces of the following vector spaces.}
\begin{enumerate}
\item \emph{$l^p(F)$}
\item \emph{$l^\infty(F)$}
\item \emph{$C_0(F)$}
\item \emph{$C(F)$}
\end{enumerate}

\item \emph{Let $X$ be a complete, separable metric space with $\varnothing \neq A \subseteq X$.  Show that if $A$ is perfect then $A$ is uncountable.}

$A$ cannot be finite as every neighborhood around each point contains an infinite number of points.  Assume that $A$ is infinite and countable, and let $A = \{a_1,a_2,a_3,\ldots\}$.  We will show that no sequence can cover all of $A.$, and that since every point in $A$ can be described by a Cauchy sequence $A$ must be uncountable.  As on Monday we will denote $S_r(x) := \overline{B_r(x)}$.

First, it is obvious that any perfect set $P$ in a complete metric space $X$ is complete since every Cauchy sequence converges and no limit point could be in $X \setminus P$ by the definition of a perfect set.

For any sequence $\{x_1, x_2, \ldots\}$ we do the following.  Take $x_1 \in A$ and for $r_1 > 0$ consider $B_{r_1}(x_1)$.  We can pick $x_2 \in  B_{r_1}(x_1)$ and $r_2 > 0$ such that $S_{r_2}(x_2) \subsetneq S_{r_1}(x_1)$ with $x_1 \notin S_{r_2}(x_2)$. Continue inductively, choosing $r_{k+1} > 0$ such that $x_{k+1} \in S_{r_{k+1}}(x_{k+1}) \subsetneq S_{r_k}(x_k)$ and $x_k \notin S_{r_{k+1}}(x_{k+1})$.

We then have a sequence of $S_{r_1}(x_1) \supsetneq S_{r_2}(x_2) \supsetneq S_{r_3}(x_3) \supsetneq \cdots$.  By the Lemma from Monday's problem session, $\bigcap_{k=1}^\infty S_{r_k}{x_k} \neq \varnothing$.  But whatever point is in this set could not be one of the points of the sequence, since $x_k \notin S_{r_{k+1}}(x_{k+1})$.  Therefore $A$ is uncountable.

\item \emph{Prove the following statements about the Cantor set, denoted $C$.}
\begin{enumerate}
\item \emph{$x \in C$ if and only if the ternary expansion of $x$ contains only $0$ and $2$.}

Define the following
\begin{eqnarray}
A_0 &=& [0,1] \\
A_n &=& A_{n-1} \setminus \bigcup_{k=0}^\infty \left( \frac{1+3k}{3^n},\frac{2+3k}{3^n} \right)
\end{eqnarray}

Then $C = \bigcap_{n=0}^\infty A_n$.
We see that at the $n^{th}$ step the left-most interval removed is always $\left(\frac{1}{3^n},\frac{2}{3^n}\right)$, and that the ternary expansion for this is $\left(0.00 \ldots 01, 0.00 \ldots 02\right)$ where the digit is in the $n^{th}$ place.  Note also that, for example, $0.1$ can be rewritten as $0.0\overline{2}$, so number of this form can be said to ``not contain any ones.''  Since every other interval is an integral multiple of the first, we get that at the $i^{th}$ step all digits whose ternary expansion contains a $1$ in the $i^{th}$ place is removed.  Therefore this construction of the Cantor set is equivalent to removing all points from $[0,1]$ whose ternary expansion contains a $1$.

\item \emph{$C$ is uncountable.}

Since $C$ is closed and perfect (and $\mathbb{R}$ is a complete, separable metric space), $C$ is uncountable.

\item \emph{$C$ is closed.}

Since each $A_n$ is closed, the Cantor set, which is the arbitrary intersection of the $A_n$, is also closed.

\item \emph{$C$ is perfect.}

Let $x \in C$ and look at its ternary expansion $0.a_1a_2a_3 \ldots$.  If $x$ has an infinite number of $2$s then we can easily find a sequence which approximates $x = 0.a_1a_2a_3 \ldots$ by defining $x_n$ as follows

\begin{equation*}
x_n = 0.b_1b_2b_3b_4 \ldots = 
\begin{cases}
b_i = a_i & i \leq n \\
b_i = 0 & i > n
\end{cases}
\end{equation*}

Then this is clearly a ternary approximation for $x$.

If $x$ has a finite ternary expansion then let $k$ denote the place of the final $2$ in that expansion and define $x_n$ as follows

\begin{equation*}
x_n = 0.b_1b_2b_3b_4 \ldots = 
\begin{cases}
b_i = a_i & i \neq k+n \\
b_i = 2 & i = k+n 
\end{cases}
\end{equation*}

Then $|x - x_n| \leq \frac{2}{3^{k+n}}$.

Therefore $C$ is a perfect set.
\item \emph{$C$ is nowhere dense.}

We will show that no open set is contained in $C$, immediately implying that $C$ is nowhere dense.

Assume there is an open set in $C$, then since $C \subset \mathbb{R}$ there is some open interval $\left(a,b\right)$ in $C$.  But this means that $\left(a,b\right)$ must be contained in all of the closed intervals which comprise each $A_i$.  Pick $k \in \mathbb{N}$ such that $\frac{1}{3^k} < b - a$.  Then $\left(a,b\right)$ is not contained in $A_k$ since each interval has a length of $\frac{1}{3^k}$.  Therefore no open set is contained in $C$.

\item \emph{Define $D(C) = \{x-y \mid x,y \in C\}$.  Show that $D(C) = [-1,1]$.}
\end{enumerate}
\end{enumerate}
\end{document}
