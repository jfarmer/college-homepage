\documentclass[10pt]{article}

\usepackage{homework}
\usepackage{enumerate}
\usepackage{geometry}

\geometry{letterpaper}

\textwidth = 6.5 in
\textheight = 9 in
\oddsidemargin = 0.0 in
\evensidemargin = 0.0 in
\topmargin = 0.0 in
\headheight = 0.0 in
\headsep = 0.0 in
\parskip = 0.2in
\parindent = 0.0in

\title{CMSC 277: Homework \#1}
\author{Jesse Farmer}
\date{6 Obctober 2005}
\begin{document}
\maketitle

\begin{enumerate}
\item \emph{Using the Order Form of Recursion on $\N$, define $X$ and $g : X^\ast \to X$ so that the corresponding
$f$ is the Fibonacci sequence.}

Let $X = \N$ and $\sigma \in X^\ast$.  Define $g: X^\ast \to X$ as follows:
\[
g(\sigma) = \begin{cases} \sigma(|\sigma|) + \sigma(|\sigma|-1) & |\sigma| \geq 2 \\ 1 & |\sigma| < 2 \end{cases}
\]

Thus $f(0) = g(\lambda) = 1$ and $f(1) = g(f[0]) = 1$, but $f(n) = g(f[0] \ast \cdots \ast f[n-1]) = f(n-1) + f(n-2)$ for all $n \geq 2$.

\item \emph{Let $A = \set{1,2,3,4,5,6,7}$, $B = \set{5}$, and $\cal{H} = \set{h_1,h_2}$ where $h_1$ and $h_2$ are given as in the homework sheet.}
\begin{enumerate}
\item \emph{Calculuate $V_3$ and $W_3$.}

Let $V_0 = B = \set{5}$.  Then we have the following
\begin{align*}
V_0 &= \set{5} \\
V_1 &= V_0 \cup \set{h_1(5)} \cup \set{h_2(5,5)} = \set{5,7} \\
V_2 &= V_1 \cup \set{h_1(7)} \cup \set{h_2(7,7), h_2(5,7), h_2(7,5)} = \set{4,5,7} \\
V_3 &= V_2 \cup \set{h_1(4)} \cup \set{h_2(4,4), h_2(4,7), h_2(7,4), h_2(4,5), h_2(5,4)} = \set{4,5,7}
\end{align*}

Since $B = \set{5}$, the only witnessing sequence of length $1$ is $5$.  As $h_1(5) = 5$ and $h_2(5,5) = 7$ we have that the only witnessing sequences of length two are $55$ and $57$.  Hence the only witnessing sequences of length three are $555$, $577$, $575$, and $574$ since $h_1(7)=4$ and all other combinations of $h_k$ and elements with witnessing sequences of length two, viz., $5$ and $7$, are already one of $\set{4,5,7}$.  Hence $W_3 = \set{4,5,7}$.
\item \emph{Calculuate $G(A,B,\cal{H})$.}

In general if $V_n = V_{n+1}$ for some $n \in \N$ then it is easy to see that $G(A,B,\cal{H}) = V_n$ since then
\begin{align*}
V_{n+2} &= V_{n+1} \cup \set{h(a_1, \ldots, a_k) \mid h \in \cal{H}_k, a_i \in V_{n+1}} \\
&= V_{n} \cup \set{h(a_1, \ldots, a_k) \mid h \in \cal{H}_k, a_i \in V_{n}} \\
&= V_{n+1}
\end{align*}

and hence $V_k = V_n$ for all $k \geq n$ by induction.  Since $V_2 = V_3$ it follows that $G(A,B,\cal{H}) = \set{4,5,7}$.
\end{enumerate}

\item \emph{Let $A = \R$, $B = \set{\frac{1}{2}, 3}$, and $\cal{H} = \set{h_2, h_3}$ given by $$h_2(x,y) = \frac{x+y}{2}$$ and $$h_3(x,y,z) = \begin{cases} |x-y| & x \neq y \\ \sqrt{z} & x=y \end{cases}$$  Show that $G(A,B,\cal{H}) \subseteq [0,3]$.}

Let $C \subset [0,3]$.  We will show that $C \cup h_2(C^2) \cup h_3 (C^3) \subset [0,3]$ so that, by induction, the proposition follows immediately.  Let $x,y \in C$, then
\[
0 \leq \frac{x+y}{2} \leq \frac{3+3}{2} = 3
\]

so that $h_2(x,y) \in [0,3]$ for all $x,y \in C$.  Now let $a_1,a_2,a_3 \in C$.  By the definition of $[0,3]$ it follows that $|a_i - a_j| \in [0,3]$ for all $i,j \in \set{1,2,3}$ so it suffices to show that $h(x) = \sqrt{x} \in [0,3]$ for all $x \in C$.  But this follows immediately, too, from the monotonicity of $h$, i.e., $0 \leq h(x) \leq 3$ for all $x \in C \subseteq [0,3]$.

Since $B \subset [0,3]$ we have $V_1 \subset [0,3]$, and that if $V_n \subseteq [0,3]$ then $V_{n+1} \subseteq [0,3]$.  Hence $V_n \subseteq [0,3]$ for all $n \in \N$ and therefore the proposition follows by the definition of $G(A,B,\cal{H})$.

\item \emph{Let $A = \R^+$, $B_1=\set{\sqrt{2}}$, $B_2 = \set{\sqrt{2}, 16}$, and $\cal{H} = \set{h}$ where $h: A \to A$ is defined by $x \mapsto x^2$.}
\begin{enumerate}
\item \emph{Describe $G(A, B_1, \cal{H})$ and $G(A, B_2, \cal{H})$ explicitly.}

Let $G_1 = G(A, B_1, \cal{H})$ and $G_2 = G(A, B_2, \cal{H})$.  Define $X = \set{2^{2^{k-1}} \mid k \in \N}$.  We claim that $X = G_1 = G_2$.  First, we show that $X$ is inductive with respect to both $B_1$ and $B_2$.  $\sqrt{2} = 2^{2^{-1}}$ and $16 = 2^{2^2}$, so $B_1, B_2 \subset X$.  Let $x \in X$, then $x^2 = \left(2^{2^{k-1}}\right)^2 = 2^{2^k} \in X$.  Hence $G_1, G_2 \subseteq X$.

Let $x = 2^{2^{k-1}} \in X$ for some $k \in \N$.  Then
\[
\sqrt{2} \ast 2 \ast 2^2 \ast \cdots \ast 2^{2^{k-2}} \ast 2^{2^{k-1}}
\]

is a witnessing sequence for $x$ in both $G_1$ and $G_2$.  Hence $X = G_1 = G_2$.

\item \emph{Show that $G(A, B_1, \cal{H})$ is free but $G(A, B_2, \cal{H})$ is not free.}

$G_1$ satisfies condition three vacuously since $\abs{\cal{H}} = 1$, and condition two trivially since $x \mapsto x^2$ is injective on $\R^+$.  Since $h(G_1) = \set{2^{2^k} \mid k \in \N} = G_1 \setminus \set{\sqrt{2}}$, the first condition is also satisfied.

$G_2$ is not free since $16 \in h(G_2)$, so it does not satisfy the first condition.

\item \emph{Define $\iota: B_2 \to \R$ by $\iota(\sqrt{2}) = 0$ and $\iota(16) = \frac{7}{2}$.  Define $g: \R^+ \times \R^+ \to \R^+$ by $(a,x) \mapsto \log_2 a + x$.  Show that there exists a function $f: G(A, B_2, \cal{H}) \to \N$ such that $f(b) = \iota(b)$ for all $b \in B_2$ and $f(h(a)) = g(a,f(a))$ for all $a \in G_2$.}

I am going to assume we are not simply taking $f: G \to \N$, since $\frac{7}{2}$ is not in $\N$, making the problem statement trivially false.  Instead, we'll take it to be $\Q$.

If $f(16) = g(4, f(4)) = \iota(16)$ then $f$ is well-defined since $16$ is generated by $h$ and $B_1$, so that every subsequent element of $G$ is accounted for.  But this is precisely what happens:
\begin{align*}
f(\sqrt{2}) &= 0 \\
f(2) &= \log_2 \sqrt{2} + 0 = \frac{1}{2} \\
f(4) &= \log_2 2 + \frac{1}{2} = \frac{3}{2} \\
f(16) &= \log_2 4 + \frac{3}{2} = \frac{7}{2} = \iota(16)
\end{align*} 

Every subsequence value of $f$ is therefore determined uniquely by $g$.

\end{enumerate}
\item \emph{Let $A = \N^+$, $B = \set{3,7}$, and $\cal{H} = \set{h_1, h_2}$ where $h_1(n) = 20n + 1$ and $h_2(n,m) = 2^n(2m+1)$.  Show that $G(A,B, \cal{H})$ is free.}

Let $G = G(A,B, \cal{H})$.  The first condition is easily satisfied since since $h_2(n,m)$ is even for all $n,m \in A$ and $7 < 13 < h_1(n)$ for all $n \in A$.  $h_1$ is injective over all $\R$ (in fact, it is bijective), so it is certainly injectiver over $G$.

To see that $h_2$ is injective let $(n_1, m_1), (n_1, m_1) \in A^2$ and assume $2^{n_1}(2m_1+1) = 2^{n_2}(2m_2+1)$.  If $n_1 < n_2$ then it must be that $2^{n_2-n_1} \mid 2m_1+1$, which is absurd as $2m_1 + 1$ is always odd.  The case for $n_2 < n_1$ follows \emph{mutatis mutandis}, and hence $n_1 = n_2$.  Therefore $2m_1+1 = 2m_2+1$, and hence $m_1=m_2$.  Therefore $h_2$ is injective over $A^2$ (and, in particular, $G^2$).

The third condition is satisfied since $h_1$ is always odd and $h_2$ is always even, and therefore $G$ is free.

\item \emph{Let $(A, B, \cal{H})$ be a generating system that is not free.  Show that there exists a set $X$ and functions $\iota: B \to X$ and $g_h: (A \times X)^k \to X$ such that there is no function $f: G \to X$ satisfying $f \mid_B = \iota$ and $f(h(a_1, \ldots, a_k)) = g_h(a_1, f(a_1), \ldots, a_k, f(a_k))$.}

We will proceed case-by-case, assume that $f$ is a function satisfying both the above properties.

\begin{enumerate}
\item Assume there exists $h \in \cal{H}^k$ and $(a_1, \ldots, a_k) \in G^k$ such that $h(a_1, \ldots, a_k) = b_0 \in B$.

Let $X = \N$ and define $\iota(b) = 1$ for all $b \in B$ and $g_h(a'_1, f(a'_1), \ldots, a'_j, f(a'_j)) = 0$ for all $(a'_1, \ldots, a'_j) \in G^k$ and $h \in H^j$, for all $j$..  Then 
\[
1 = \iota(b_0) = f(b_0) = f(h(a_1, \ldots, a_k)) = g_h(a_1, f(a_1), \ldots, a_k, f(a_k)) = 0
\]

which is absurd.

\item Assume there exists some $h \in \cal{H}_k$ and distinct $\vec{a} \in G^k, \vec{b} \in G^k$ such that $h(\vec{a}) = h\vec{b})$.

Let $X = A^\ast$.  Define $\iota(b) = b$ and $g_h(a_1, f(a_1), \ldots, a_k, f(a_k)) = a_1 \ast \cdots \ast a_k$.  Then
\begin{align*}
b_1 \ast \cdots b_k &= g_h(b_1, f(b_1), \ldots, b_k, f(b_k)) \\
&= f(h(\vec{b}))  \\
&= f(h(\vec{a})) \\
&= g_h(a_1, f(a_1), \ldots, a_k, f(a_k)) \\
&= a_1 \ast \cdots \ast a_k
\end{align*}

which implies $\vec{a} = \vec{b}$, a contradiction.

\item Case 3
\end{enumerate}

\end{enumerate}
\end{document}
