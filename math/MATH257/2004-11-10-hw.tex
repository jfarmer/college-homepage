\documentclass[letterpaper, 11pt]{article}
\textwidth = 6.5 in
\textheight = 9 in
\oddsidemargin = 0.0 in
\evensidemargin = 0.0 in
\topmargin = 0.0 in
\headheight = 0.0 in
\headsep = 0.0 in
\parskip = 0.2in
\parindent = 0.0in
\usepackage{amsfonts}
\usepackage{amsmath}
\usepackage{amssymb}

\newcommand{\brac}[1]{
\left\langle #1 \right\rangle
}

\newcommand{\Aut}{\text{Aut}}

\newcommand{\Z}{\mathbb{Z}}
\newcommand{\R}{\mathbb{R}}
\newcommand{\N}{\mathbb{N}}

\title{MATH 257: Homework \#6}
\author{Jesse Farmer}
\date{10 November 2004}
\begin{document}
\maketitle
\begin{enumerate}
\item \emph{Let $G$ be a finite abelian group with order $n=mk$ where $(m,k)=1$.  Define $G(r) = \{g \in G \mid g^r = 1\}$.  Prove that $G=G(m)G(k)$.}

Since $(m,k) = 1$ there exist integers $a,b$ such that $am + bk = 1$.  Then
\[
g = g^{am+bk} = g^{am}g^{bk}
\]

but $\left(g^{am}\right)^k = g^{amk} = g^{an} = \left(g^n\right)^k = 1$, so that $g^{am} \in G(k)$.  Similarly $g^{bk} \in G(m)$, and therefore $G \subseteq G(m)G(k)$.  The opposite inclusion is obvious and our statement is proven.

\item \emph{Let $H$ and $K$ be groups, $\varphi: K \rightarrow \Aut(H)$ be a group homomorphism.  Prove that $C_{\tilde{K}})(H) = \ker \varphi$, where $H$ and $K$ are isomorphic copies of $H$ and $K$ in $H \rtimes_{\varphi} K$.}

From theorem 10 part 5, $khk^{-1} = \varphi(k)(h)$.  Denote the identity map by $I$, then
\begin{eqnarray*}
\ker \varphi &=& \{k \in K \mid \varphi(k) = I\} \\ 
&=& \{k \in K \mid \varphi(k)(h) = h,\, \forall h \in H\} \\
&=& \{k \in K \mid khk^{-1} = h,\, \forall h \in H \} \\
&=& C_K(H)
\end{eqnarray*}

Note that in truth $\ker \varphi$ contains elements of $K$ while $C_K(H)$ contains elements of the subgroup isomorphic to $K$, so that in actuality $\ker \varphi \cong C_K(H)$.

\item \emph{Let $H = (\Z_n,+)$ and $A = (\Z_n^{\times},\cdot)$.  Define $\tau(a,b): \Z_n \rightarrow Z_n$ by $\overline{x} \mapsto \overline{ax+b}$ with $(a,n) = 1$ and $b \in \Z$ and $G = \{\tau(a,b) \mid a,b \in \Z,\,(a,n)=1\}$.}
\begin{enumerate}
\item \emph{Define $\phi_{\overline{a}}: H \rightarrow H$ by $\overline{h} \mapsto \overline{ha}$.  Show that $\phi_{\overline{a}} \in \Aut(H)$.}

Denote $\phi_{\overline{a}}$ by $\varphi$ for brevity's sake.  $\varphi$ is surjective since there exists an $\overline{h'} = \overline{ha^{-1}}$ such that $\varphi\left(\overline{h'}\right) = \overline{h}$ for all $\overline{h} \in H$.  It is trivially injective since every $a$ has an inverse by construction.

That it is a homomorphism is also equally obvious:
\[
\varphi\left(\overline{h_1} + \overline{h_2}\right) = \varphi\left(\overline{h_1 + h_2}\right) = \overline{(h_1+h_2)a} = \overline{h_1a} + \overline{h_2a} = \varphi\left(\overline{h_1}\right) + \varphi\left(\overline{h_2}\right)
\]
\item \emph{Show that $\phi: A \rightarrow \Aut(H)$ is an injective group homomorphism.}

$\phi$ is a homomorphism since
\[
\phi\left(\overline{a_1a_2}\right)\left(\overline{h}\right) = \overline{ha_1a_2} = \phi\left(\overline{a_2}\right)\left(\overline{ha_1}\right) = \left(\phi\left(\overline{a_1}\right) \circ \phi\left(\overline{a_2}\right)\right)\left(\overline{h}\right)
\]

If $\phi\left(\overline{a_1}\right) = \phi\left(\overline{a_2}\right)$ then $\overline{ha_1} = \overline{ha_2}$ for every $h \in H$.  In particular this is true for $h = 1$, so $\overline{a_1} = \overline{a_2}$.  Therefore $\phi$ is an injective group homomorphism.

\item \emph{Show that for $\overline{a} \in A$ and $\overline{b} \in H$, $\tau(1,b)^{\tau(a,0)} = \tau(1,ab)$.}

We will identify $a$ with $\overline{a}$ and $b$ with $\overline{b}$.  Let $x \in \Z_n$ then
\begin{eqnarray*}
\left(\tau(a,0)\tau(1,b)\tau(a,0)^{-1}\right)(x) &=& \left(\tau(a,0)\tau(1,b)\right)\left(a^{-1}x\right) \\
&=&  \left(\tau(a,0)\right)\left(a^{-1}x+b\right) \\
&=& x+ab \\
&=& \tau(1,ab)(x)
\end{eqnarray*}
\item \emph{Show that $G \cong H \rtimes_\phi A$.}

Once again we identify $a$ with $\overline{a} \in A$ and $b$ with $\overline{b} \in H$.  Define the map $\psi: H \rtimes_\phi A \rightarrow G$ by $(a,b) \mapsto \tau(a,b)$.  Note that in general $\tau(a_1, b_1) \circ \tau(a_2,b_2) = \tau(a_1a_2, b_1a_2 + b_2)$.  Recall that the operation on $H$ is addition and the operation on $A$ is multiplication modulo $n$.  Then
\begin{eqnarray*}
\psi(a_1,b_1)(a_2,b_2) &=& \psi(a_1a_2, \phi(a_2)(b_1) + b_2) \\
&=& \tau(a_1a_2, \phi(a_2)(b_1) + b_2) \\
&=& \tau(a_1a_2, a_2b_1 + b_2) \\
&=& \tau(a_1, b_1) \circ \tau(a_2,b_2)
\end{eqnarray*} 

and therefore $\psi$ is a homomorphism.  $\psi$ is trivially surjective from our construction.  It is injective since if $\tau(a_1,b_2)(x) = \tau(a_2,b_2)(x)$ for all $x \in \Z_n$ then $a_1x+b_1 = a_2x+b_2$ for all $x \in \Z_n$, including $x=0$ and $x=1$.  Hence $a_1 = a_2$ and $b_1 = b_2$ in $\Z_n$.  Therefore $G \cong H \rtimes_\phi A$.
\end{enumerate}
\end{enumerate}
\end{document}
