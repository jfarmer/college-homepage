\documentclass[letterpaper, 11pt]{article}
\textwidth = 6.5 in
\textheight = 9 in
\oddsidemargin = 0.0 in
\evensidemargin = 0.0 in
\topmargin = 0.0 in
\headheight = 0.0 in
\headsep = 0.0 in
\parskip = 0.2in
\parindent = 0.0in
\usepackage{amsfonts}
\usepackage{amsmath}
\usepackage{amssymb}

\newcommand{\brac}[1]{
\left\langle #1 \right\rangle
}

\newcommand{\Z}{\mathbb{Z}}
\newcommand{\R}{\mathbb{R}}
\newcommand{\N}{\mathbb{N}}

\title{MATH 257: Homework \#4}
\author{Jesse Farmer}
\date{27 October 2004}
\begin{document}
\maketitle
\begin{enumerate}

\item \emph{Let $\varphi: G \rightarrow H$ be a homomorphism and $E \leq H$.  Prove that $\varphi^{-1}(E) \leq G$.  If $E \unlhd H$ show that $\varphi^{-1}(E) \unlhd G$.  Deduce that $\ker \varphi \unlhd G$.}

A subset of $H$ of a group $G$ is a subgroup if and only if for all $x,y \in G$, $xy^{-1} \in G$.  Let $x,y \in \varphi^{-1}(E)$, then $\varphi(x), \varphi(y) \in E$.  Because $E \leq H$, $\varphi(y^{-1}) = \varphi(y)^{-1} \in E$.  Hence $\varphi(x)\varphi(y^{-1}) = \varphi(xy^{-1}) \in E$ and therefore $xy^{-1} \in \varphi^{-1}(E)$.

If $E \unlhd H$ and $x \in \varphi^{-1}(E)$ is arbitrary, then $\varphi(gxg^{-1}) = \varphi(g)\varphi(x)\varphi(g)^{-1} = h\varphi(x)h^{-1}$.  This is in $E$ since $\varphi(x) \in E$ and $E \unlhd H$, hence $gxg^{-1} \in \varphi^{-1}(E)$ and $\varphi^{-1}(E) \unlhd G$.

$\ker \varphi = \varphi^{-1}(e)$, i.e., the kernel is the fiber of the identity element.  Since $\{e\} \unlhd H$, it follows that $\ker \varphi = \varphi^{-1}(\{e\}) \unlhd G$.

\item \emph{Let $\varphi: G \rightarrow H$ be a group homomorphism with kernel $K$ and let $a,b \in \varphi(G)$.  Let $X \in G / K$ be the fiber above $a$ and $Y$ the fiber above $b$.  Fix an element $u$ of $X$ so $\varphi(u) = a$.  Prove that if $XY = Z$ in the quotient group $G/K$ and $w \in Z$ then there exists $v \in Y$ such that $uv = w$.}

If $w \in XY$ then $\varphi(w) = ab$ since there exist $x,y$ such that $w=xy$ where $\varphi(x)=a$ and $\varphi(y)=b$.  If $u \in X$ is fixed any $w \in Z$ is arbitrary, then $\varphi(u^{-1}w) = \varphi(u)^{-1}\varphi(w) = a^{-1}ab = b$.  Hence $u^{-1}w \in Y$, and this is precisely the $v$ for which we are looking.

\item \emph{Prove that if $N \unlhd G$ and $H \leq G$ then $N \cap H \unlhd H$.}

Let $x \in N \cap H$.  Then $x \in N$ and $x \in H$.  For any $h \in H \leq G$, $hxh^{-1} \in N$ since $N \unlhd G$ and $h \in G$, and $hxh^{-1} \in H$ since $H$ is a subgroup of $G$.  Therefore $hgh^{-1} \in N \cap H$, i.e., $N \cap H \unlhd H$.

\item \emph{Prove that if $G/Z(G)$ is cyclic then $G$ is abelian.}

In general, if $N \unlhd G$ then $(gN)^n = g^nN$.  This is easy to see through induction.  Since the group operation on the quotient group is well-defined, i.e., $gHgH = HggH = g^2H$, if it is true for $n \in \mathbb{N}$ then $(gN)^{n+1} = (gN)^ngN = g^nNgN = Ng^ngN = g^{n+1}N$.  That is is true for $n \in \mathbb{Z}$ is proven identically by considering $g^{-1}N$.

Since $Z(G)$ is certainly normal -- it is an abelian subgroup -- this holds here.  Assume $G/Z(G) = <xZ(G)>$ for some $x \in G$, then $G/Z(G) = \{(xZ(G))^n \mid n \in \mathbb{Z}\} = \{x^nZ(G) \mid n \in \mathbb{Z}\}$.  If $g \in G$ then $g \in gH = x^kZ(G)$ for some $k \in \mathbb{Z}$, and hence $g = x^kz$ for some $k \in \mathbb{Z}$ and some $z \in Z(G)$.  Let $g_1, g_2 \in G$ then $g_1 = x^jz$ and $g_2 = x^kz'$ for some $k,j \in \mathbb{Z}$.  Since $z$ and $z'$ commute with every element of $G$,
\[
g_1g_2 = (x^jz)(x^kz') = z(x^jx^k)z' = zx^{j+k}z' = z'x^{k+j}z = z'(x^kx^j)z = (x^kz')(x^jz) = g_2g_1
\]

Therefore $G$ is abelian


\item \emph{Let $H \leq G$ and let $g \in G$.  Show that if the right coset $Hg$ equals some left coset of $H$ in $G$ then it equals $gH$ and hence $g \in N_G(H)$.}

Let $x \in G$ be such that $gH = Hx$, then certainly $g \in gH = Hx$.  However, $g \in Hg$.  Since cosets partition $G$, and $Hg \cap Hx \neq \emptyset$, $gH = Hx = Hg$.

\item \emph{Prove that there are the same number of left cosets as right cosets.}

Consider the map $\varphi(gH) = Hg^{-1}$ from the set of left cosets to the set of right cosets.  The map $\varphi^{-1}(Hg) = g^{-1}H$ is both a left and right inverse since
\[
\left(\varphi \circ \varphi^{-1}\right)(Hg) =  \varphi(g^{-1}H) = Hg
\]
and
\[
\left(\varphi^{-1} \circ \varphi\right)(gH) = \varphi^{-1}(Hg^{-1}) = gH
\]

Hence $\varphi$ is a bijection, and so the number of left and right cosets must be equal.

\item \emph{Let $G$ be a finite group and $H \leq G$, $N \unlhd G$.  Prove that if $(|H|, [G: N]) = 1$ then $H \leq N$.}

Let $\varphi: G \rightarrow G/N$ be the natural group homomorphism.  Then $\varphi \mid_{H}:H \rightarrow G/N$ is still a homomorphism.  This means $\ker \varphi \mid_{H} \unlhd H$ and so $|\varphi(H)| \mid |H|$.  Since $H \leq G$, $\varphi(H) \leq G/N$ and therefore $|\varphi(H)| \mid [G:N]$ by Lagrange's theorem.  However, $(|H|, [G:N]) = 1$ so $|\varphi(H)| = 1$, i.e., $\varphi(H) = \{e\}$.  This implies $hN = N$ for all $h \in H$ and therefore $H \leq N$.

\item \emph{Determine the last two digits of $3^{3^{100}}$.}

In general, the order of any element of a group divides the order of the group, and hence if $x \in |G|$ then $x^{|G|} = e$.  Since an element $x$ of the integers modulo $n$ is a unit if and only if $(x,n)=1$, $\left|(\mathbb{Z}/n\mathbb{Z})^{\times}\right| = \varphi(n)$ and therefore $x^{\varphi(n)} \equiv 1 \mod n$.

Note that $\varphi(100) = 40$ and $(3,100) = 1$, hence $3^{40} \equiv 1 \mbox{ (mod $100$)}$.  Also, $$3^{100} \equiv (3^4)^{10} \equiv 81^{10} \equiv 1 \mbox{ (mod $40$)}$$

Therefore there exists a $k$ such that $$3^{3^{100}} = 3^{40k + 1} \equiv 3^{40}3 \equiv 3 \mbox{ (mod $100$)}$$

The last two digits are $0$ and $3$.
\item \emph{Let $\sigma = (1\,2\,3\,4\,5)$ in $S_5$.  Find $\tau$ such that the following are satisfied:}
\begin{enumerate}
\item \emph{$\tau\sigma\tau^{-1} = \sigma^2$}

$\sigma^2 = (1\,3\,5\,2\,4)$, so we must find a $\tau$ such that $tau(1)=3$, $\tau(2)=5$, $\tau(3)=2$, $\tau(4)=4$, and $\tau(5)=1)$.  Computing the cycles for this gives $\tau = (1\,3\,2\,5)$.

\item \emph{$\tau\sigma\tau^{-1} = \sigma^{-1}$}

$\sigma^{-1} = (1\,5\,4\,3\,2)$, so $\tau$ must be such that $\tau(1)=1$, $\tau(2) = 5$, $\tau(3)=4$, $\tau(4) = 3$, $\tau(5)=2$.  This gives $\tau = (2\,5\,)(3\,4)$.

\item \emph{$\tau\sigma\tau^{-1} = \sigma^{-2}$}

$\sigma^{-2} = (31425)$, and using the exact same procedure as the previous two yields $\tau=(1\,3\,4\,2)$.
\end{enumerate}

\item \emph{For each of the following determine if $\sigma_1$ and $\sigma_2$ are conjugate, and if so a permutation $\tau$ such that $\tau\sigma_1\tau^{-1} = \sigma_2$.}
\begin{enumerate}
\item \emph{$\sigma_1 = (1\,2)(3\,4\,5)$ and $\sigma_2=(1\,2\,3)(4\,5)$.}

Yes, these two permutations are conjugate and $\tau = (1\,3\,5\,2\,4)$.  The procedure used is the same as that used in the previous problems.

\item \emph{$\sigma_1 = (1\,5)(3\,7\,2)(10\,6\,8\,11)$ and $\sigma_2 = (3\,7\,5\, 10)(4\, 9)(13\, 11\, 2)$}

Yes.  $\tau = (1\,4)(3\,10\,11\,7\,6\,9\,5\,8\,12\,13)$.

\item \emph{$\sigma_1 = (1\,5)(3\,7\,2)(10\,6\,8\,11)$ and $\sigma_2 = \sigma_1^3$}

These cannot be conjugate because they have different orders, $12$ and $4$ respectively.

\item \emph{$\sigma_1 = (1\,3)(2\,4\,6)$ and $(3\,5)(2\,4)(5\,6)$}

These cannot be conjugate either because they have different orders, $6$ and $2$, respectively.

\end{enumerate}
\end{enumerate}
\end{document}
