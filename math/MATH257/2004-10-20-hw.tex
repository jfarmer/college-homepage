\documentclass[letterpaper, 11pt]{article}
\textwidth = 6.5 in
\textheight = 9 in
\oddsidemargin = 0.0 in
\evensidemargin = 0.0 in
\topmargin = 0.0 in
\headheight = 0.0 in
\headsep = 0.0 in
\parskip = 0.2in
\parindent = 0.0in
\usepackage{amsfonts}
\usepackage{amsmath}
\usepackage{amssymb}

\newcommand{\brac}[1]{
\left\langle #1 \right\rangle
}

\newcommand{\Z}{\mathbb{Z}}
\title{MATH 257: Homework \#3}
\author{Jesse Farmer}
\date{20 October 2004}
\begin{document}
\maketitle
\begin{enumerate}

\item \emph{Prove that if $A \subset B$ then $\brac{A} \leq \brac{B}$.  Give an example where $A \subset B$ and $A \neq B$, but $\brac{A} = \brac{B}$.}

Both $\brac{A}$ and $\brac{B}$ are groups under the same operation, so it suffices to show that $\brac{A} \subset \brac{B}$.  Every element of $\brac{A}$ can be written as $a_1^{\epsilon_1} \cdots a_k^{\epsilon_k}$.  By hypothesis, however, $a_i \in B$, and hence $a_1^{\epsilon_1} \cdots a_k^{\epsilon_k} \in \brac{B}$.

If there exists an element in $b \in B \setminus A$ such that $b = a_1a_2$ for some $a_1, a_2 \in A$, then, even though $A \neq B$, $\brac{A} = \brac{B}$.

\item \emph{A group $H$ is called \emph{finitely generated} if there is a finite set $A$ such that $H = \brac{A}$.}
\begin{enumerate}
\item \emph{Prove that every finite group is finitely generated.}

If $H$ is finite then certainly $H = \brac{H}$, and is therefore finitely generated.

\item \emph{Prove that $\mathbb{Z}$ is finitely generated.}

Clearly $(\mathbb{Z},+) = \brac{1} = \{n \cdot 1 \mid n \in \mathbb{Z}\}$.

\item \emph{Prove that every finitely generated subgroup of the additive group of $\mathbb{Q}$ is cyclic.}

Consider $H = \brac{\frac{p_1}{q_1}, \ldots, \frac{p_n}{q_n}}$.  For any $a \in H$ there exist $m_1, \ldots, m_n$ such that 
\[
a = \sum_{i=1}^n \frac{m_ip_i}{q_i} = \frac{\sum_{i=1}^n \left( m_ip_i \prod_{k \neq i} q_k \right)}{\prod_{k=1}^n q_k} = n \frac{1}{k}
\]

where $n$ is the numerator of the above expression and $k$ the denominator.  Hence $H \leq \brac{\frac{1}{k}}$ (it is a group, and certainly a subset of the latter), and since the subgroup of a cyclic group is itself cyclic, $H$ must be cyclic.

\item \emph{Prove that $\mathbb{Q}$ is not finitely generated.}

From the previous part we see it is sufficient to show that $\mathbb{Q}$ is not cyclic.  This is clear since, if it were the case, every element could be written as $\frac{np}{q}$ for some $p,q,n \in \mathbb{Z}$.  All that is necessary to construct an element not of this form is to pick two primes $p_1, p_2$ not equal to $q$ or $p$.  There exists no $n$ such that $\frac{p_1}{p_2} = n\frac{p}{q}$.

\end{enumerate}

\item \emph{Let $A = \{ \pi \in S_n \mid |\pi| = 2\}$.  Show that $S_n = \brac{A}$.}

Obviously $\brac{A} \subset S_n$, so it is sufficient to show that $S_n \subset \brac{A}$, i.e., every element of $S_n$ can be written as the composition of $2$-cycles.  Since any permutation can be decomposed into disjoint cycles, it is also sufficient to show that a single cycle can be written as the product of $2$-cycles.  Consider the cycle $(c_1c_2 \cdots c_n)$.  We claim
\[
(c_1c_2 \cdots c_n) = (c_1c_2)(c_1c_3)\cdots(c_1c_i)(c_1c_{i+1})\cdots(c_1c_n)
\]

This $2$-cycle representation is correct since $c_i \mapsto c_{i+1}$, $c_1 \mapsto c_2$, and $c_n \mapsto c_1$.

\item \emph{Let $G$ be a group and $a_1, \ldots, a_n \in G$ with $a_ia_j = a_ja_i$ for all $i,j$.  Prove that $H = \brac{a_1, \ldots, a_n} = \{a_1^{m_1} \cdots a_n^{m_n} \mid m_i \in \mathbb{Z}\}$ and $H$ is Abelian.}

Because of commutability we can group element of the same base together, and add their exponents.  Induction on $n$ would perhaps be more rigorous.

\item \emph{Let $G$ be a group and $a,b \in G$ with $|a| = |b| = |ab| = 2$.  Prove that $\brac{a,b} = \{1,a,b,ab\}$ is of order $4$.}

Every element $g$ of $\brac{a,b}$ satisfies the property $g^2 =1$, and so by the previous homework $\brac{a,b}$ is Abelian.  Hence $g = a^nb^m$ for some $m,n \in \mathbb{N}$.  However, since every power of $a$, $b$, or $ab$ is either itself or the identity (because they are all of order $2$), it follows that $a^nb^m$ is one of $a,b, ab$.  Note that $a,b,e \neq ab$ since, if $ab = a$ or $ab = b$ then $b$ or $a$ would be of order $1$, respectively, and if $ab = e$ then $a=b$.

\item \emph{Prove that up to isomorphism there are exactly two groups of order $4$.}

Any group of order $4$ can be written as $\{e,a,b,c\}$.  The order of any element must divide the order of the group, hence the orders of $a,b,c$ must be some combination of $2$ and $4$, respectively (since only the identity has order $1$).  If two distinct non-identity elements $a,b$ have order $2$ then $a \neq b^{-1}$ since this implies $a=b$.  The contrapositive of this shows that if $a = b^{-1}$ then $a,b$ do not have order $2$, i.e., they must have order $4$.  Hence there must be an even number of elements of order $4$, which means there are either none or there are two.  This completely determines the structure of the group since the identity has order $1$, and therefore there are at most two distinct groups of order $4$, up to isomorphism.  It is easy to check that two such groups exist and that they are $\mathbb{Z}_4$ and $\mathbb{Z}_2 \times \mathbb{Z}_2$.

\item \emph{Find the number of subgroups of order $n$ of $S_n$ when $n$ is prime and when $n=4$.}

Up to isomorphism the answer is $1$ and $2$, respectively.  However, not counting isomorphic groups, there are $(n-2)!$ subgroups of order $n$ when $n$ is prime.  This is because we know each non-identity element of the subgroup must have order $n$ and hence be composed of commuting $n$-cycles, and hence has one cycle of length $n$.  Fixing the first element of that cycle and permuting the others determines all the other members of the group, but for each of these $(n-1)!$ subgroups, there are $n-1$ permutations that result in the same group (e.g., $\{e, (123), (132)\}$ is the same group, not simply isomorphic to, $\{e, (132), (123)\}$), so there are $(n-2)!$ actual subgroups.

When $n=4$ there are two possibilities for the orders of the elements, but from previous problems we know that the elements must have orders of the form $1,2,2,2$ or $1,2,4,4$.  There are three of the former and one of the later, making for $4$ in total.

\item \emph{Let $H \leq G$ and $g \in G$.  Prove that if the right coset equals some left coset of $H$ in $G$ then it equals the left coset $gH$ and $g \in N_G(H)$.}

Clearly if $gH = Hg$ then $g \in N_G(H)$, since every element of $h$ can be written as $gh'g^{-1}$ for some $h' \in H$.  It suffices to prove that if $gH = Hx$ then $gH = Hg$.

$g \in gH$ since $e \in H$, and therefore $g \in Hx$.  $Hx$ and $Hg$ are either disjoint or coincide, and since they share at least one element, namely $g$, they must coincide.  Therefore $gH = Hx = Hg$.

\item \emph{Prove that if $H$ and $K$ are finite subgroups of $G$ whose orders are relatively prime then $H \cap K = \{e\}$.}

Since $H \cap K \leq K$ and $H \cap K \leq H$, the order of $H \cap K$ must divide both $|H|$ and $|K|$, but since these two integers are relatively prime the largest integer to do so is $1$ and $|H \cap K|$ must be $1$.

\item \emph{Let $G$ be a group and $H_1, H_2$ be subgroups of $G$ of finite index.  Prove that $H_1 \cap H_2$ is a subgroup of $H_1$ of finite index and $[H_1 : H_1 \cap H_2] \leq [G: H_2]$.  Deduce that $[G: H_1 \cap H_2]$ is finite.}

Let $h_1, h_2 \in H_1 \cap H_2$.  Then $h_1^{-1}h_2 \in H_1$ and $h_1^{-1}h_2 \in H_2$ since $H_1$ and $H_2$ are groups, which implies $h_1^{-1}h_2 \in H_1 \cap H_2$.  That is, $H_1 \cap H_2 \leq H_1$.  Now consider $H_1 / H_1 \cap H_2$.  The smallest $H_1 \cap H_2$ could be is if it were $\{e\}$, in which case $H_1 / H_1 \cap H_2 \cong H_1$.  Similarly, the largest it could be is if $H_1 = H_2$, in which case $H_1 / H_1 \cap H_2 \cong \{e\}$.  Therefore $[H_1 : H_1 \cap H_2] \leq |H_1| < \infty$.  And this is not true, since $|H_1|$ is not necessarily finite, but I misread the problem and it's late now.

By Proposition $13$ in Chapter $3$,
\[
[H_1 : H_1 \cap H_2] = \frac{|H_1|}{|H_1 \cap H_2|} = \frac{|H_1H_2|}{|H_2|} \leq \frac{|G|}{|H_2|} = [G: H_2]
\]

Since $H_1 \cap H_2 \leq H_1$, it must be that $|H_1 \cap H_2| \leq |H_1|$ and hence
\[
[G : H_1 \cap H_2] = \frac{|G|}{|H_1 \cap H_2|} \leq 
\]
\end{enumerate}
\end{document}
