\documentclass[letterpaper, 11pt]{article}
\textwidth = 6.5 in
\textheight = 9 in
\oddsidemargin = 0.0 in
\evensidemargin = 0.0 in
\topmargin = 0.0 in
\headheight = 0.0 in
\headsep = 0.0 in
\parskip = 0.2in
\parindent = 0.0in
\usepackage{amsfonts}
\usepackage{amsmath}
\usepackage{amssymb}

\title{MATH 257: Homework \#1}
\author{Jesse Farmer}
\date{06 October 2004}
\begin{document}
\maketitle
\begin{enumerate}

\item \emph{Prove that $a_n10^n + a_{n-1}10^{n-1} + \cdots + a_0 \equiv a_n + a_{n-1} + \cdots + a_0 \mbox{ (mod 9)}$.}

We will show that, in general, if $x \equiv y \mbox{ (mod n)}$ then 
\[
	a_mx^m + a_{m-1}x^{m-1} + \cdots + a_0 \equiv a_my^m + a_{m-1}y^{m-1} + \cdots + a_0 \mbox{ (mod n)}
\]

This is equivalent to showing that $n \mid a_m(x^m - y^m) + a_{m-1}(x^{m-1} - y^{m-1}) + \cdots + a_1(x-y)$.  It is therefore sufficient to show that $n | (x^r - y^r)$ for every $r \in \mathbb{N}$.  This is easy to see since, by hypoethesis, we have $n | (x-y)$ and
\[
n \mid (x-y)\sum_{k=1}^{r} x^{k-1}y^{r-k} = x^r - y^r
\]

Therefore $n \mid a_r(x^r - y^r)$ for every $r \in \mathbb{N}$, and the congruence is proven.  The problem is a special case where $x=10$, $y=1$, and $n=9$.

\item \emph{Find the remainder of $37^{100}$ when divided by $29$.}

The answer is $23$.  It is easier to calculate if we use Fermat's Little Theorem since 29 is prime, so
\[
37^{100} \equiv (8^{16})(8^{28})^3 \equiv 8^{16} \equiv  (8^2)^8 \equiv 64^{8} \equiv 6^8 \equiv (6^2)^4 \equiv 7^4 \equiv 400 \equiv 23 \mbox { (mod 29)}
\]

\item \emph{Define $\tau_x(a,b): \mathbb{Z}_n \rightarrow \mathbb{Z}_n$ as $\overline{x} \mapsto \overline{ax+b}$ and $G = \{\tau_x(a,b) \mid a,b \in \mathbb{Z}, (a,n)=1\}$.}
\begin{enumerate} 
\item \emph{Show that each element of $G$ is a well-defined permutation on $\mathbb{Z}_n$.}

Let $x_1, x_2 \in \overline{x}$ so that $x_1 \equiv x_2 \mbox{ (mod n)}$.  By the fact that addition and multiplication are well-defined on $\mathbb{Z}_n$, $ax_1+b \equiv ax_2+b \mbox{ (mod n)}$, i.e., $\overline{ax_1+b} = \overline{ax_2+b}$.  The inverse of an arbitrary $\tau_x(a,b)$ is constructed explicitly below, and hence each $\tau_x(a,b) \in G$ is a bijection from $\mathbb{Z}_n$ to $\mathbb{Z}_n$, i.e., a permutation of $\mathbb{Z}_n$.
 
\item \emph{Show that if $\alpha, \beta \in G$ then $\alpha\beta, \alpha^{-1} \in G$.}

Let $\alpha, \beta \in G$ and define $\alpha := \tau_x(a,b)$ and $\beta := \tau_x(c,d)$.  Since $(a,n) = 1$, $a^{-1}$ exists.  We claim $\alpha^{-1} = \gamma := \tau_x(a^{-1}, -a^{-1}b) \in G$.
\[
x(\gamma\alpha) \equiv a(a^{-1}x - a^{-1}b) + b) \equiv aa^{-1}x - aa^{-1}b + b \equiv x - b + b \equiv x \mbox{ (mod n)}
\]
and
\[
x(\alpha\gamma) \equiv a^{-1}(ax + b) - b) \equiv a^{-1}ax + a^{-1}ab - b \equiv x + b - b \equiv x \mbox{ (mod n)}
\]

Moreover,
\[
\alpha\beta = \overline{c(ax + b) + d} = \overline{cax + cb + d} = \tau_x(ca, cb+d) \in G
\]
\item \emph{Find $|G|$ if $n$ is prime.}

Let $\tau_x(a,b) = \tau_x(a',b')$ so that $ax+b \equiv a'x+b' \mbox{ (mod n)}$.  This implies $x(a-a') + (b-b') \equiv 0 \mbox{ (mod n)}$, i.e., $a \equiv a'$ and $b \equiv b' \mbox{ (mod n)}$.  In general this means there are $\varphi(n)$ ways to choose $a$, where $\varphi$ is Euler's totient function, and $n$ ways to choose $b$, and hence $|G| = n\varphi(n)$.  For $n$ prime $\varphi(n) = n-1$ (since all elements of $\mathbb{Z}_n \setminus \{0\}$ are units), so in this case $|G| = n(n-1)$.
\end{enumerate}

\item \emph{Let $G = \{x \in \mathbb{R} \mid x \in [0,1)\}$ and for all $x,y \in G$ define $x \star y = x+y - [x+y]$.  Show that $(G, \star)$ is an Abelian group.}

Let $x,y \in G$ be arbitrary.  If $0 \leq x+y < 1$ then $[x+y] = 0$, so $0 \leq x+y - [x+y] < 1$.  Otherwise, if $1 \leq x+y < 2$ then $[x+y] = 1$, so $0 \leq x+y - [x+y] < 1$.  Therefore $x \star y \in G$.

The identity is clearly $0$ since for $x \in G$, $[x] = 0$.  $x^{-1} = 1-x$ since 
\[
x \star (1-x) = x + (1 - x) - [x + (1 - x)] = 1 - [1] = 0
\]

Commutativity is inherited from $\mathbb{R}$.  Let $x,y,z \in G$, then
\begin{eqnarray*}
(x \star y) \star z &=& (x+y - [x+y]) \star z = x+y+z - [x+y] - \big[x+y+z - [x+y]\big] \\
x \star (y \star z) &=& x \star (y+x - [y+z]) = x+y+z - [y+z] - \big[x+y+z - [y+z]\big]
\end{eqnarray*}

So it is sufficient to show
\begin{equation}
\label{star_group}
\big[x+y+z - [x+y]\big] - [y+z] = \big[x+y+z - [y+z]\big] - [x+y]
\end{equation}

From the definition of $G$ it is clear that $[x+y], [y+z] \in \{0,1\}$.  If both are $0$ or both are $1$ then (\ref{star_group}) is obvious, so assume without loss of generality that $[x+y] = 0$ and $[y+z] = 1$.  Then, letting $a = x+y+z$, 
\[
\big[x+y+z - [x+y]\big] - \big[x+y+z - [y+z]\big] = [a] - [a-1]= 1
\]

but
\[
[y+z] - [x+y] = 1
\]

Combining the above two yields (\ref{star_group}), and hence $\star$ is associative.  Therefore $(G, \star)$ is an Abelian group.

\item \emph{Let $\pi \in S_n$ and define $\pi^i$ recursively by $\pi^i = \pi^{i-1}\pi$.  The order of $\pi$ is $$|\pi| = \min\{i \in \mathbb{N} \mid \pi^i = I\}$$}
\begin{enumerate}
\item \emph{Show that $|\pi|$ is the least common multiple of the lengths of the cycles of $\pi$.}

Consider $\pi$ as the product of disjoint cycles $c_1, c_2, \ldots, c_k$, and let the length of the cycle $c_i$ be $l_i$.  If $c_i^n = I$, the identity, then $n \mid l_i$ since, if some element is permuted by $c_i$ it must be permuted some multiple of $l_i$ times for it to return to its original position because of the injective nature of disjoint cycles.  Since composition of disjoint cycles is commutative,
\[
\pi^n = (c_1c_2 \cdots c_k)^n = c_1^nc_2^n \cdots c_k^n
\]

If $\pi^n = I$ then $c_i^n = I$ and hence $n \mid l_i$ for $i=1,2,\ldots,k$.  The smallest such $n$ to do this is by definition the least common multiple of the $l_i$, i.e., the least common multiple of the lengths of the cycles of $\pi$.

\item \emph{Let $N(n,m)$ be the number of permutations in $S_n$ of order $m$.  Determine $N(n,m)$ for $n \leq 5$ and for all $m$.}

Since $\left|S_n\right| = n!$, this provides a way of checking whether the calculated values are correct.  Also, in general, there are $\binom{n}{m}(m-1)!$ cycles of length $m$.  We can see this by choosing a subset of size $m$, calling the first element $m_1$, and permuting the other $m-1$ elements.

\begin{enumerate}
\item[\underline{$n=1$}:]  Since all permutations are the identity, $N(1,1) = 1$.
\item[\underline{$n=2$}:]  $N(2,1) = 1$, and $N(2,2) = 1$.
\item[\underline{$n=3$}:]  $N(3,1) = 1$, $N(3,2) = \binom{3}{2}(2-1)! = 3$, $N(3,3) = \binom{3}{3}(3-1)! = 2$
\item[\underline{$n=4$}:]  $N(4,1) = 1$, $N(4,2) = \binom{4}{2}(2-1)! + \frac{\binom{4}{2}}{2} = 9$, $N(4,3) = \binom{4}{3}(3-1)! = 8$, $N(4,4) = \binom{4}{4}(4-1)! = 6$
\item[\underline{$n=5$}:] $N(5,1) = 1$, $N(5,2) = \binom{5}{2} + \binom{5}{2}\binom{3}{2}\binom{1}{1} = 25$, $N(5,3) = \binom{5}{3}(3-1)! = 20$, $N(5,4) = \binom{5}{4}(4-1)! = 30$, $N(5,5) = 4! = 24$, $N(5,6) = \binom{5}{3}(3-1)! = 20$.
\end{enumerate}

\end{enumerate}

\item \emph{For what for $n,m \in \mathbb{Z}$ can the map $f: \mathbb{Z} \rightarrow \mathbb{Z}$ defined as $f(x) = x^2$ be considered as a map from $\mathbb{Z}_n$ to $\mathbb{Z}_m$?}

The map must be well-defined, so for any two $x,y \in \overline{x}$, $f(x) \equiv f(y) \mbox{ (mod m)}$.  In particular, let $x \in \overline{x}$ be arbitrary and let $y = x+n$.

Simply expanding the required congruence, $x^2 \equiv (x+n)^2 \mbox{ (mod m)}$ shows that $f$ is well-defined if and only if the following is true:
\begin{equation}
\label{square_function}
2nx + n^2 \equiv 0 \mbox{ (mod m), } \forall x \in \mathbb{Z}
\end{equation}

We claim that (\ref{square_function}) is true if and only if $m \mid 2n$ and $m \mid n^2$.  That this condition is sufficient is obvious, so assume (\ref{square_function}) is valid for all $x \in \mathbb{Z}$.  In particular this means (\ref{square_function}) must be valid for $x=0$, and hence $m \mid n^2$.  Similarly, it must be valid for $x=1$, and hence (since $m \mid n^2$) $m \mid 2n$, which proves our claim.  Note that this works even when considering the trivial group $\{0\}$, since $1 \equiv 0 \mbox{ (mod 1)}$ and certainly $m$ will always divide $0$.
\end{enumerate} 
\end{document}
