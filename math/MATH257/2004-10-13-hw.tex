\documentclass[letterpaper, 11pt]{article}
\textwidth = 6.5 in
\textheight = 9 in
\oddsidemargin = 0.0 in
\evensidemargin = 0.0 in
\topmargin = 0.0 in
\headheight = 0.0 in
\headsep = 0.0 in
\parskip = 0.2in
\parindent = 0.0in
\usepackage{amsfonts}
\usepackage{amsmath}
\usepackage{amssymb}

\title{MATH 257: Homework \#2}
\author{Jesse Farmer}
\date{13 October 2004}
\begin{document}
\maketitle
\begin{enumerate}
\item \emph{Prove that for all $n > 1$ that $\mathbb{Z}/n\mathbb{Z}$ is not a group under multiplication.}

$(\mathbb{Z}/n\mathbb{Z},+,\cdot)$ is a ring with $1 \neq 0$ if $n > 1$, so for all $a \in \mathbb{Z}/n\mathbb{Z}$,
\[
0\cdot a = (0+0)\cdot a = 0\cdot a + 0\cdot a \Rightarrow 0\cdot a = 0 
\]

Hence there cannot exist a multiplicative inverse for $0$, and therefore $(\mathbb{Z}/n\mathbb{Z},\cdot)$ is never a group if $n > 1$.
\item \emph{If $a,b$ are commuting elements of a group $G$, prove that $(ab)^n = a^nb^n$ for all $n \in \mathbb{Z}$.}

This is obviously true for $n=1$, so assume it is true for $n = k \in \mathbb{N}$, then
\[
(ab)^{k+1} = (ab)^k(ab) = a^kb^kab = a^kab^kb = a^{k+1}b^{k+1}
\]

To show this for all $k \in \mathbb{Z}$, consider $(ab)^{-n}$.  If $a$ and $b$ are commuting elements then $a^{-1}b^{-1} = (ba)^{-1} = (ab)^{-1} = b^{-1}a^{-1}$, so $a^{-1}, b^{-1}$ are also commuting elements.  Therefore $(ab)^{-1} = a^{-1}b^{-1}$.  Assume the statement is true for $n = k$, then
\[
(ab)^{-(k+1)} = (ab)^{-k}(ab)^{-1} = a^{-k}b^{-k}a^{-1}b^{-1} = a^{-k}a^{-1}b^{-k}b^{-1} = a^{-(k+1)}b^{-(k+1)}
\]

Therefore if $a,b$ are commuting elements then $(ab)^n = a^nb^n$ for all $n \in \mathbb{Z}$.

\item \emph{Prove that if $x^2 = 1$ for all $x \in G$ then $G$ is an Abelian group.}

If $x^2 = 1$ for all $x \in G$ then $x = x^{-1}$ for all $x \in G$.  Let $x,y \in G$ be arbitrary, then
\[
xy = (xy)^{-1} = y^{-1}x^{-1} = yx
\]
\item \emph{Show that an element has order $2$ in $S_n$ if and only if its cycle decomposition is a product of commuting $2$-cycles.}

This problem is a special case of the next problem where $p=2$.

\item \emph{Let $p$ be a prime.  Show that an element has order $p$ in $S_n$ if and only if its cycle decomposition is a product of commuting $p$-cycles.  Show that this need not be the case if $p$ is not prime.}

It is important in this problem to note that if a cycle has length one it is omitted from the ``cycle'' decomposition.  Therefore there is never a case where the length of a ``cycle'' is $1$.

If $\pi \in S_n$ is the product of commuting $p$-cycles then obviously the order of $p$, the least common multiple of the lengths of the cycles, is $p$.  Let $c_1, c_2, \ldots, c_k$ be the cycles into which $\pi$ is decomposed, $l_1, l_2, \ldots, l_k$ their respective lengths, and assume $\pi$ has an order of $p$.  Then
\[
p = \mbox{lcm}(l_1, \ldots, l_k) = \min\{d : l_i \mid d, i=1,2,\ldots,k\}
\]

Since each $l_i \mid p$, $l_i = 1$ or $l_i = p$. As was stated before the case where $l_i = 1$ is ruled out by our notation, and therefore $l_i = p$.

To show that $p$ must be prime, take $p=6$.  Then there are permutations of order $6$ in $S_5$ (e.g., $(123)(45)$) even though it is impossible to create a single cycle of order $6$.

\item \emph{If $\varphi: G \rightarrow H$ is a group isomorphism show that $|\varphi(x)|=|x|$ for all $x \in G$.  Deduce that any two isomorphic groups have the same number of elements of order $n$ for each $n \in \mathbb{N}$.  Is this true if $\varphi$ is only assumed to be a homomorphism?}

Inductively it is clear that $\varphi(x^n) = \varphi(x)^n$ for $n \in \mathbb{Z}$.  Let $|x|=n$ and $|\varphi(x)|=j$, $n,j > 0$.  Then
\[
\varphi(x^j) = \varphi(x)^j = 1 = \varphi(1) = \varphi(x^n)
\]

By the injectivity of $\varphi$, $x^j = x^n$, which implies $x^{n-j} = 1$.  Since neither $n$ nor $j$ is $0$, and by hypothesis $n$ is the smallest non-zero number such that $x^n = 1$, it must be the case that $n-j=0$, i.e., $n=j$.

Define $G_n = \{x \in G \mid |x| = n\}$ and $H_n$ similarly.  $\varphi(G_n) = H_n$ since, for every element $h \in H_n$ there exists an element $g \in G$ such that $\varphi(g) = h$.  However, from above, $|g| = |h| = n$, and therefore $g \in G_n$.  For the same reason, every element of $G_n$ is sent to an element of $H_n$.  Since $\varphi$ is also injective it follows that $\varphi\mid_{G_n}: G_n \rightarrow H_n$ is a bijection, and hence $G_n$ and $H_n$ have the same cardinality.

This need not be the case if $\varphi$ is only assumed to be a homomorphism.  The map $\varphi(x) = e_H$ is a non-bijective homomorphism, and $|\varphi(x)| \neq |x|$ for all $x \in G$ unless $G$ is the trivial group.

\item \emph{Prove that $(\mathbb{R}^{\times}, \cdot) \ncong (\mathbb{C}^{\times}, \cdot)$.}

Assume for contradiction that $\varphi: (\mathbb{C}^{\times}, \cdot) \rightarrow (\mathbb{R}^{\times}, \cdot)$ is a group isomorphism.  There exists a $k \in \mathbb{R}^{\times}$ such that $\varphi(i) = k$.  However,
\[
1 = \varphi(1) = \varphi(i^4) = k^4
\]

and therefore $\varphi(i) = -1$ since $\varphi$ is a bijection and $\varphi(1) = 1$.  Then
\[
-1 = \varphi(-1) = \varphi(i^2) = (-1)^2 = 1
\]

This is a contradiction, and therefore no such $\varphi$ can exist.

\item \emph{Prove that $(\mathbb{Z}, +) \ncong (\mathbb{Q}, +)$.}

Let $\varphi: \mathbb{Z} \rightarrow \mathbb{Q}$ be a group isomorphism.  Then, since every element of $\mathbb{Z}$ is generated by $1$, we can write $n = \epsilon_1 + \epsilon_2 + \cdots + \epsilon_k$ where $\epsilon_i = \pm  1$.  Then $\varphi(n) = \sum_{i=1}^k \varphi(\epsilon_i) = \sum_{i=1}^k \pm \varphi(1)$, i.e., every element of the image set must be generated by $\varphi(1)$.  This is clearly not the case ($(\mathbb{Q}, +)$ is not cyclic -- indeed, it is not even finitely generated), and so no such isomorphism can exist.

To show that that $(\mathbb{Q}, +)$ is not cyclic, assume it is generated by some element $\frac{p}{q} \in \mathbb{Q}$.  Then $\left< \frac{p}{q} \right> = \{ k \frac{p}{q} \mid k \in \mathbb{Z} \}$.  Clearly $\frac{p'}{q} \in \mathbb{Q}$ is not in this set, where $p'$ is an integer which is not divided by $p$.

\newpage
\item \emph{For the following show that the specified subset is not a subgroup of the given group:}
\begin{enumerate}
\item \emph{The set of $2$-cycles of $S_n$ for $n \geq 3$.}

The identity is a $1$-cycle and therefore not contained in this set.

\item \emph{The set of reflections in $D_{2n}$ for $n \geq 3$.}

The identity is not a reflection.

\item \emph{For $n > 1$ a composite integer and $G$ a group with an element of order $n$, the set $H = \{x \in G \mid |x| = n\} \cup \{1\}$.}

Write $n=ab$ for $a,b \neq 1$.  Assume $H$ is a subgroup of $G$, then since $x \in H$ by hypothesis, $x^a \in H$.  However, $(x^a)^b = x^{ab} = 1$, and $b < n$, so the order of $x^a$ is less than $n$, a contradiction.  Therefore $H$ cannot be a subgroup of $G$.

\item \emph{The set of odd integers and $0$ in $\mathbb{Z}$.}

The set is not closed under addition since the sum of two odd integers is never odd.

\item \emph{The set of real numbers whose square is a rational number (under addition).}

Let $p,q$ be primes with $p \neq q$.  Assume for contradiction that $\sqrt{pq} \in \mathbb{Q}$, i.e., there exist $m,n$ with $(m,n) = 1$ such that $\sqrt{pq} = \frac{m}{n}$.  Then $n^2pq = m^2$ and $p \mid m$, so write $m = pk$ for some $k \in \mathbb{Z}$.  This yields $n^2pq = p^2k^2 \Rightarrow n^2q = k^2p$.  Since $p \nmid q$, this also implies $p \mid n$, i.e., $(m,n) \geq p$, a contradiction.  Therefore $\sqrt{pq} \notin \mathbb{Q}$ for $p,q$ prime and $p \neq q$.

Let $p,q$ be primes as above.  Then certainly $\left(\sqrt{p}\right)^2$ and $\left(\sqrt{q}\right)^2$ are rational.  However
\[
(\sqrt{p}+\sqrt{q})^2 = p + 2\sqrt{pq} + q
\]

This is rational only if $\sqrt{pq} \in \mathbb{Q}$, which as was shown above is never the case.  Hence this set is not closed under addition.

\end{enumerate}

\item \emph{Prove that $G$ cannot have a subgroup $H$ with $|H| = |G|-1$, and $|G| > 2$.}

If $|H| = |G| - 1$ then there must be one and only one element contained in $G$ that is not contained in $H$.  Call this element $g$.  Take $x,y \in H$, $x \neq 1$, and write $y = x^{-1}g \neq g$.  Then $xy = g \notin H$, so $H$ is not closed under the group operation and therefore cannot be a subgroup of $G$.

\item \emph{Let $H_1 \leq H_2 \leq \cdots$ be an ascending chain of subgroups of $G$.  Prove that $\bigcup_{i=1}^\infty H_i$ is a subgroup of $G$.}

Let $h \in \bigcup_{i=1}^\infty H_i$. Then there exists a $k \in \mathbb{N}$ such that $h \in H_k$.  Since $H_k$ is by hypothesis a group, $k^{-1} \in H_k \subset \bigcup_{i=1}^\infty H_i$.

Similarly, let $h_1, h_2 \in \bigcup_{i=1}^\infty H_i$.  Then there exist $i,j$ such that $H_i \subset H_j$ and $h_1 \in H_i$, $h_2 \in H_j$.  Since this implies $h_2 \in H_j$ then, since $H_j$ is by hypothesis a group, $h_1h_2 \in H_j \subset \bigcup_{i=1}^\infty H_i$.  Therefore $\bigcup_{i=1}^\infty H_i$ is a group. 

\item \emph{Let $G$ be a group and for fixed $g \in G$ define a map from $G$ to $G$ $$\varphi_g(h) = ghg^{-1}$$}
\begin{enumerate}
\item \emph{Prove that $\varphi_g$ is an isomorphism of $G$.}

Let $\varphi_g^{-1}(h) = g^{-1}hg$, then
\[
\varphi_g(\varphi_g^{-1}(h)) = g(g^{-1}hg)g^{-1} = (gg^{-1})h(gg^{-1}) = h
\]
and
\[
\varphi_g^{-1}(\varphi_g(h)) = g^{-1}(ghg^{-1})g = (g^{-1}g)h(g^{-1}g) = h
\]

Therefore $\varphi_g$ is a bijection.  Moreover, let $h_1, h_2 \in G$,
\[
\varphi_g(h_1h_2) = g(h_1h_2)g^{-1} = (gh_1)g^{-1}g(h_2g^{-1}) = (gh_1g^{-1})(gh_2g^{-1}) = \varphi_g(h_1)\varphi_g(h_2)
\]

so $\varphi_g$ is a homomorphism.
\item \emph{Prove that $\psi: G \rightarrow \text{Aut}(G)$ defined by $\psi(g) = \varphi_g$ is a homomorphism.}

Let $h,g_1, g_2 \in G$, then
\[
\varphi_{g_1g_2}(h) = (g_1g_2)h(g_1g_2)^{-1} = g_1(g_2hg_2^{-1})g_1^{-1} = g_1\varphi_{g_2}(h)g_1^{-1} = (\varphi_{g_1} \circ \varphi_{g_2})(h)
\]

Therefore $\psi(g_1g_2) = \psi(g_1) \circ \psi(g_2)$, i.e., $\psi: (G, \cdot) \rightarrow (\text{Aut}(G), \circ)$ is a group homomorphism.

\end{enumerate}

\item \emph{Let $G$ be a group and $H$ be a subgroup of $G$.  Define $$ X = \{gHg^{-1} \mid g \in G\}$$}
\begin{enumerate}
\item \emph{Prove that $N_G(H) = \{g \in G \mid gHg^{-1} \subset H\}$ is a subgroup of $G$.}

First, $N_G(H) \neq \emptyset$ since $1 \in N_G(H)$.  Let $g \in N_G(H)$.  For every $h \in H$ there exists an $h' \in H$ such that $ghg^{-1} = h'$.  Hence, $g^{-1}h'g = h \in H$, i.e., $g^{-1} \in N_G(H)$.  Similarly, let $g_1, g_2 \in N_G(H)$. Then for every $h \in H$ there exist $h', h''$ such that $g_1hg_1^{-1} = h'$ and $g_2gg_2^{-1} = h''$, hence
\[
g_1g_2hg_2^{-1}g_1^{-1} = g_1h'g_1^{-1} = h'' \in H
\]

Therefore $N_G(H)$ is a subgroup of $G$.
\item \emph{Prove that the map $\pi: G \rightarrow \text{Sym}(X)$ defined by $\pi(g) = \phi_g$, where $\phi_g(H') = gH'g^{-1}$ for $H' \in X$, is a homomorphism.}

Let $h, g_1, g_2 \in G$ be arbitrary, then
\[
\phi_{g_1g_2}(H') = g_1g_2H'g_2^{-1}g_1^{-1} = g_1(g_2H'g_2^{-1})g_1^{-1} = g_1\phi_{g_2}(H')g_1^{-1} = (\phi_{g_1} \circ \phi_{g_2})(H')
\]

and therefore $\pi(g_1g_2) = \pi(g_1) \circ \pi(g_2)$, i.e., $\pi: (G, \cdot) \rightarrow (\text{Sym(X)}, \circ)$ is a group homomorphism.
\end{enumerate}

\item \emph{Let $G$ be a group and $g \in G$.  Prove that if $|g| < \infty$ then $g^i = g^j$ if and only if $i \equiv j \mbox{ (mod $|g|$)}$, and that if $|g| = \infty$ then $g^i = g^j$ if and only if $i=j$.}

In general, since $|g|$ is the smallest integer $n$ such that $g^n = e$ it follows that if $g^k = 1$ then $|g| \mid k$, i.e., $k$ must be some multiple of the order of $g$.  Assume $g$ is of finite order, then $g^i = g^j$ if and only if $g^{i-j} = e$, which is true if and only if $|g| \mid (i-j)$, i.e., $i \equiv j \mbox{ (mod $|g|$)}$.

If $g$ is of infinite order then there exists no non-zero integer $n$ such that $g^n = e$, so $g^{i-j} = e$ if and only if $i-j = 0$, i.e., $i = j$.

\end{enumerate}
\end{document}
