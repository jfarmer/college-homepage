\documentclass[11pt]{article}
\textwidth = 6.5 in
\textheight = 9 in
\oddsidemargin = 0.0 in
\evensidemargin = 0.0 in
\topmargin = 0.0 in
\headheight = 0.0 in
\headsep = 0.0 in
\parskip = 0.2in
\parindent = 0.0in
\usepackage{amsfonts}
\usepackage{amsmath}
\usepackage{amssymb}
\usepackage{pslatex}
\usepackage{enumerate}
\title{MATH 207: Homework \#7}
\author{Jesse Farmer}
\date{17 November 2003}
\begin{document}
\maketitle
\begin{enumerate}
\item \emph{Let $\mathfrak{I}$ be the ideal of all Cauchy sequences which converge to zero. If $(a_n) \in \mathfrak{I}$ then there exists an $N$ such that all $a_k$ are the same sign when $k \geq N$.}

If there were an infinite number of $a_k$ of opposite signs then, since the sequence is Cauchy, we can make them arbitrarily close.  But this would mean that each term of the sequence could be made arbitrarily close to zero, i.e., the sequence converges to zero -- a contradiction.  Since there are then only a fininte number of terms of opposite sign, simply pick a $N$ greater than the last term with an opposite sign and all the terms after that must then be the same sign.

\item \emph{The ideal generated by $\mathfrak{I}$ and any $(b_n) \notin \mathfrak{I}$ contains the identity.}

Let $(b_n) \in \mathfrak{I}_b$, where $\mathfrak{I}_b$ is the ideal generated by $\mathfrak{I}$ and $(b_n)$.  Define
 \[ c_n = \left\{
              \begin{array}{ll}
                   0 & b_n \neq 0\\
                   1 & b_n = 0
              \end{array}
       \right.
\]

Since $(b_n)$ is Cauchy and does not converge to zero it cannot contain an infinite number of zeroes (otherwise it would converge to zero since there would be a $b_k = 0$ for $k$ greater than any natural number).  Clearly, then, $(c_n)$ is Cauchy and converges to zero by the fact that there are only a finite number of zeroes in $(b_n)$, and so $(c_n) \in \mathfrak{I}_b$.  But this also means $(b_n + c_n) \in \mathfrak{I_b}$, which is invertible as it contains no zeroes.  Therefore $1_\mathfrak{C} = (b_n + c_n)^{-1}(b_n + c_n) \in \mathfrak{I}_b \Rightarrow \mathfrak{I}_b = \mathfrak{C}$.

\item \emph{Show that the order on $\mathbb{R}$ is well-defined.}

\item \emph{Define the notion of a bounded sequence in $\mathbb{R}$.}

A sequence $(a_k)$ in $\mathbb{R}$ is bounded if the set $\{a_k\}_{k \in \mathbb{N}}$ is bounded.

\item \emph{Prove that every bounded sequence in $\mathbb{R}$ has a convergent subsequence.}

Every sequence in $\mathbb{R}$ has a monotonic subsequence by Lemma 3.48 and every bounded, monotonic sequence (a subsequence is certainly a sequence) converges by Lemma 3.49.  Therefore, every sequence in $\mathbb{R}$ has a convergent subsequence.

\item \emph{Prove that every Cauchy sequence in $\mathbb{R}$ is bounded.}

Let $(a_k)$ be a Cauchy sequence in $\mathbb{R}$.  Pick $N \in \mathbb{N}$ such that $|a_n - a_m| < 1$ for $m,n \geq N$.  Certainly $|a_n - a_N| < 1$ for $n \geq N$, and by the triangle inequality $|a_n| < 1 + |a_N|$.  Let $M = \max\{|a_1|,|a_2|,\ldots,1+|a_N|\}$.  Then $(|a_k|)$ is bounded by $M$.

\item \emph{Find the accumulation points of the following sets:}
\begin{enumerate}
\item \emph{$S=(0,1)$}

$[0,1]$
\item \emph{$S = \{(-1)^n + \frac{1}{n}\mid n \in \mathbb{N}\}$}

$\{-1,1\}$

\item \emph{$S = \mathbb{Q}$}

$\mathbb{R}$
\item \emph{$S = \mathbb{R}$}

$\mathbb{R}$
\item \emph{$S = \mathbb{Z}$}

$\emptyset$
\item \emph{$S$ is the set of rational numbers whose denominator is prime}

$\{0\}$
\end{enumerate}

\item
\begin{enumerate}
\item \emph{Find an infinite subset of $\mathbb{R}$ which does not have an accumulation point.}

Choose $\mathbb{Z}$ as the infinite subset.  Each point is isolated, so every neighborhood around a point in the set does not contain another point in the set.  That is, no point is an accumulation point (though the set is closed).

\item \emph{Find a bounded subset of $\mathbb{R}$ which does not have an accumulation point.}

Choose $\{0\}$ as the subset.  Since this is obviously bounded and is a subset of $\mathbb{Z}$, it does not contain an accumulation point by above.
\end{enumerate}

\item
\begin{enumerate}
\item \emph{Show that $\emptyset$ and $\mathbb{R}$ are the only subsets of $\mathbb{R}$ which are both open and closed in $\mathbb{R}$.}

\underline{Lemma}: \emph{Let $\mathbb{X}$ be some set where the notion of an open set is adequately defined.  $\mathbb{X}$ has a non-trivial open and closed subset if and only if $\mathbb{X}$ can be written as the union of two disjoint open subsets.}

\underline{Proof}: Let $\emptyset \neq A \subset \mathbb{X}$ be both open and closed.  Then $A^c$ is also open and $A \cup A^c = \mathbb{X}$.  Likewise, let $A,B \subset \mathbb{X}$ be open.  If $A \cap B = \emptyset$ and $A \cup B = \mathbb{X}$ then $A^c = B$ and $B^c = A$, i.e., they are both also closed and $A,B \subset \mathbb{X}$. 

We now show that $\mathbb{R}$ cannot be written as the union of two disjoint open intervals and thus that $\emptyset$ and $\mathbb{R}$ are the only sets both open and closed in $\mathbb{R}$.  In general, any interval in $\mathbb{R}$ cannot be written as the union of two disjoint open sets.

Assume for contradiction that $I \subseteq \mathbb{R}$ is an interval that can be written as such.

Pick $x \in A \cap I$ and $y \in B \cap I$, and assume without loss of generality that $x \leq y$.  Define

\[ S = \{n \mid [x,n] \subseteq A\}\]

$S$ is non-empty since it certainly contains $x$, and $y$ is an upper bound (otherwise it would be in $A$).  Hence, $S$ has a supremum, call it $\alpha$.  Since $x \leq \alpha$ and $\alpha \leq y$ we have $\alpha \in I$ as $I$ is an interval.  $\alpha$ must then be in one of $A$ or $B$.

Assume $\alpha \in A$.  Then there is an $\epsilon > 0$ such that $(\alpha - \epsilon, \alpha + \epsilon) \subset A$ by the openness of $A$.  $\alpha - \epsilon < \alpha$, so there exists a $z \in S$ such that $\alpha - \epsilon \leq z < \alpha$, which means $[x,z] \subseteq A$.  Since $\alpha - \epsilon \leq z$, $[z,\alpha + \frac{\epsilon}{2}] \subseteq A \Rightarrow [x,\alpha+\frac{\epsilon}{2}] \subseteq A$.  But then $\alpha+\frac{\epsilon}{2} \in S$, contradicting the assumption that $\alpha$ is the least upper bound.

Assume $\alpha \in B$. Then there is an $\epsilon > 0$ such that $(\alpha - \epsilon, \alpha + \epsilon) \subset B$ by the openness of $B$.  $\alpha - \epsilon < \alpha$, so there exists a $z \in S$ such that $\alpha - \epsilon < z < \alpha$, which means $[x,z] \subseteq A$.  But then $z \in A$ and $z \in (\alpha - \epsilon, \alpha + \epsilon) \subset B$, a contradiction.

Our assumption that $I$ could be written as a union of two disjoint open sets was therefore wrong.  By the lemma, $\mathbb{R}$ contains no non-trivial sets which are both open and closed.  $\mathbb{R}$ and $\emptyset$ are both trivially open and closed in $\mathbb{R}$ and are therefore the only two subsets of $\mathbb{R}$ which are both open and closed.

\item \emph{Show that every nonempty open subset in $\mathbb{R}$ can be written as a countable union of pairwise disjoint open intervals.}

Let $S$ be a nonempty open subset of $\mathbb{R}$ and $x$ be an arbitrary point in $S$.  By definition $x$ there is at least one open interval contained in $S$ which contains $x$.  Let $C_x$ be the union of all these open intervals and $\alpha = \sup C_x$ and $\beta = \inf C_x$.  Clearly $C_x \subseteq (\alpha,\beta)$.  If $x \neq y  \in (\alpha,\beta)$ then $y$ is in some open interval contained in $S$ that also contains $x$, i.e., $y \in C_x$.  Thus, $C_x = (\alpha,\beta)$ and is not a subset contained in a larger open interval in $S$.  Each $C_x$ forms an equivalence relation on $S$ as $C_x = C_y$ or $C_x \cap C_y = \emptyset$ (otherwise $C_x$ and $C_y$ would be contained in a larger interval).  Each $C_x$ contains at least one rational number, so we see that $\bigcup_{k \in \mathbb{Q} \cap S}C_k = S$.  Since $\mathbb{Q}$ is countable, $S$ is the countable union of pairwise disjoint open intervals.

\item \emph{Show that the arbitrary union of open sets is open.}

Let $S = \bigcup_{k}S_k$ where each $S_k$ is an open set and $k$ is any parameter.  If $x_0 \in S$ then it is contained in at least one of the $S_k$, but there exists an $\epsilon > 0$ such that $(x_0 - \epsilon, x_0 + \epsilon) \subset S_k$ because $S_k$ is open.  Since $x \in S_k \Rightarrow x \in S$, this interval must also be contained in $S$. Therefore, given some $x_0 \in S$ there exists an $\epsilon > 0$ such that $(x_0 - \epsilon, x_0 + \epsilon) \subset S$, i.e., any arbitrary union of open sets is also open.

\item \emph{Show that the finite intersection of open sets is open.}

Let $S = \bigcap_{k=1}^nS_k$ where each $S_k$ is an open set and $k \in \mathbb{N}$.  If $x_0 \in S$ then $x_0$ is in each of $S_1,S_2,\ldots,S_n$, but all of these are open.  That is, for each $S_k$ there exists an $\epsilon_k > 0$ such that $(x_0 - \epsilon_k, x_0 + \epsilon_k) \subset S_k$.  If we choose $\epsilon = \min\{\epsilon_1,\epsilon_2,\ldots,\epsilon_n\}$ then for any $x_0 \in S$ we have $(x_0 - \epsilon, x_0 + \epsilon) \subset S$.  Therefore any finite intersection of open sets is open.
\end{enumerate}

\item 
\begin{enumerate}
\item \emph{Show that a subset of $\mathbb{R}$ is closed if and only if it contains all its accumulation points.}

Let $S \subseteq \mathbb{R}$ be a set. If $S$ is closed then $S^c$ is open.  Assume for contradiction that $x_0 \in S^c$ is an accumulation point for $S$.  Every neighborhood around $x_0$ has a point in $S$, but because $S^c$ is open some of these neighborhoods will also be contained in $S^c$, a contradiction.  Conversely, if $S$ is not closed then $S^c$ is not open.  That is, there exists some $x_0 \in S^c$ such that for any $\epsilon > 0(x_0 - \epsilon , x_0 + \epsilon) \nsubseteq S^c$.  So there is some $x \in S$ such that $x \in (x_0 - \epsilon , x_0 + \epsilon)$ for all $\epsilon > 0$ and $x_0$ is an accumulation point of $S$ not in $S$.

Therefore a set $S \subset \mathbb{R}$ is closed if and only if it contains all its accumulation points.

\underline{Lemma}: If $\{S_\alpha\}$ is a collection of sets $S_\alpha$ then
\begin{equation}
\label{OpenClosedUnionIntersect}
\left(\bigcup_\alpha S_\alpha \right)^c = \bigcap_\alpha \left(S_\alpha^c\right) \mbox{ and }
\left(\bigcap_\alpha S_\alpha \right)^c = \bigcup_\alpha \left(S_\alpha^c\right)
\end{equation}

\begin{tabular}{lll}
\underline{Proof}: $x \in \left(\bigcup_\alpha S_\alpha \right)^c$ & $\Leftrightarrow$ & $x \notin \bigcup_\alpha S_\alpha$\\
						& $\Leftrightarrow$ & $x \notin S_\alpha$ for all $\alpha$\\
						& $\Leftrightarrow$ & $x \in S_\alpha^c$ for all $\alpha$\\ 
						& $\Leftrightarrow$ & $x \in \bigcap_\alpha \left(S_\alpha^c\right)$
\end{tabular}

Likewise we have the complement of the intersection is the union of the complements.

\item \emph{Show that the intersection of arbitrarily many closed sets is closed.}

Let $\{S_\alpha\}$ be a collection of closed sets.  By (\ref{OpenClosedUnionIntersect}), $\left(\bigcap_\alpha S_\alpha \right)^c = \bigcup_\alpha \left(S_\alpha^c\right)$.  $S_\alpha^c$ is open since $S_\alpha$ is closed, so  $\bigcup_\alpha \left(S_\alpha^c\right)$ is open.  Hence $\left(\bigcap_\alpha S_\alpha \right)^c$ is open and therefore $\bigcap_\alpha S_\alpha$ is closed.

\item \emph{Show that the finite union of closed sets is closed.}

Let $\{S_n\}$ be a finite collection of closed sets.  By (\ref{OpenClosedUnionIntersect}), $\left(\bigcup_{i=1}^n S_i \right)^c = \bigcap_{i=1}^n \left(S_i^c\right)$.  $S_i^c$ is open since $S_i$ is closed, so $\bigcap_{i=1}^n \left(S_i^c\right)$ is also open.  Hence $\left(\bigcup_{i=1}^n S_i \right)^c$ is open and therefore $\bigcup_{i=1}^n S_i$ is closed.
\end{enumerate}

\item \emph{Show that $\mathbb{R} \times \mathbb{R}$ with addition and multiplication defined \ldots blah blah}

\item \emph{Show that if we identify $z = a + bi$ with the point $(a,b) \in \mathbb{R}^2$ then $|z|$ gives the length of the vector $(a,b)$.}

The length of the vector $(a,b)$ is found trivially using the Pythagorean Theorem as $|(a,b)| = \sqrt{a^2 + b^2} = |a + bi|$.

\item \emph{Show that the absolute value on $\mathbb{C}$ satisfies all the properties of the absolute value defined on $\mathbb{R}$}

Let $z,w \in \mathbb{C}$ be arbitrary and $|z| = \sqrt{z\bar{z}} = \sqrt{\Re(z)^2 + \Im(z)^2}$

That $|z| \geq 0$ is obvious and clearly $|z| = 0 \Leftrightarrow z \bar{z} = 0 \Leftrightarrow z = 0$

Let $z=a+bi$ and $w=c+di$.  Then $|zw|^2 = (ac-bd)^2 + (ad+bc)^2 = (a^2+b^2)(c^2+d^2) = |z|^2|w|^2 = (|z||w|)^2 \Rightarrow |zw| = |z||w|$

Next, $|\Re(z)| \leq |z|$ since $|a| = \sqrt{a^2} \leq \sqrt{a^2 + b^2}$.

\begin{tabular}{lll}
$|z+w|^2$ 	&$=$&		$(z+w)(\bar{z}+\bar{w})$\\
	 	&$=$& 		$z\bar{z}+z\bar{w}+\bar{z}w+w\bar{w}$\\
		&$=$&		$|z|^2+2\Re(z\bar{w})+|w|^2$\\
		&$\leq$&	$|z|^2+2|z\bar{w}|+|w|^2$\\
		&$=$&		$|z|^2+2|z||w|+|w|^2$\\
		&$=$&		$(|z| + |w|)^2$
\end{tabular}

And therefore $|z+w| \leq |z| + |w|$

\item
\begin{enumerate}
\item \emph{Show that $\emptyset$ and $\mathbb{C}$ are the only subsets of $\mathbb{C}$ which are both open and closed.}
\item \emph{Show that every open set in $\mathbb{C}$ can be written as a countable union of open balls.}


\item \emph{Give an example of an open set in $\mathbb{C}$ that cannot be written as the countable union of pairwise disjoint open balls}

Assume for contradiction that $\mathbb{C}$ can be written as a countable union of pairwise disjoint open balls.  Let $\mathbb{C}_0 = \mathbb{C}-B_r$ where $\emptyset \neq B_r \subset \mathbb{C}$ is one of the pairwise disjoint open balls.  $B_r$ is open, but so is $\mathbb{C}_0$, so $\mathbb{C}_0^c = B_r$ must also be closed.  By above, $B_r$ must be $\mathbb{C}$ or the empty set, a contradiction.

\item \emph{Show that an arbitrary union of open sets in $\mathbb{C}$ is an open set in $\mathbb{C}$.}

Let $S = \bigcup_{k}S_k$ where each $S_k$ is an open set and $k$ is any parameter.  If $z_0 \in S$ then it is contained in at least one of the $S_k$, but there exists an $r > 0$ such that $B_r(z_0) \subset S_k$ because $S_k$ is open.  Since $z \in S_k \Rightarrow z \in S$, this interval must also be contained in $S$. Therefore, given some $z_0 \in S$ there exists a $B_r(z_0) \subset S$, i.e., any arbitrary union of open sets is also open when in $\mathbb{C}$.

\item \emph{Show that a finite intersection of open sets in $\mathbb{C}$ is an open set in $\mathbb{C}$.}

Let $S = \bigcap_{k=1}^nS_k$ where each $S_k$ is an open set and $k \in \mathbb{N}$.  If $z_0 \in S$ then $z_0$ is in each of $S_1,S_2,\ldots,S_n$, but all of these are open.  That is, for each $S_k$ there exists an $r_k > 0$ such that $B_{r_k}(z_0) \subset S_k$.  If we choose $r = \min\{r_1,r_2,\ldots,r_n\}$ then for any $z_0 \in S$ we have $B_r(z_0) \subset S$.  Therefore any finite intersection of open sets is open.

\item \emph{Show that an arbitrary intersection of closed sets in $\mathbb{C}$ is a closed set in $\mathbb{C}$.}

Let $\{S_\alpha\}$ be a collection of closed sets.  By (\ref{OpenClosedUnionIntersect}), $\left(\bigcap_\alpha S_\alpha \right)^c = \bigcup_\alpha \left(S_\alpha^c\right)$.  $S_\alpha^c$ is open since $S_\alpha$ is closed, so  $\bigcup_\alpha \left(S_\alpha^c\right)$ is open.  Hence $\left(\bigcap_\alpha S_\alpha \right)^c$ is open and therefore $\bigcap_\alpha S_\alpha$ is closed.

\item \emph{Show that a finite union of closed sets in $\mathbb{C}$ is a closed set in $\mathbb{C}$.}

Let $\{S_n\}$ be a finite collection of closed sets.  By (\ref{OpenClosedUnionIntersect}), $\left(\bigcup_{i=1}^n S_i \right)^c = \bigcap_{i=1}^n \left(S_i^c\right)$.  $S_i^c$ is open since $S_i$ is closed, so $\bigcap_{i=1}^n \left(S_i^c\right)$ is also open.  Hence $\left(\bigcup_{i=1}^n S_i \right)^c$ is open and therefore $\bigcup_{i=1}^n S_i$ is closed.

\end{enumerate}

\item \emph{Define the notion of a bounded sequence in $\mathbb{C}$.}

A sequence $(z_n)$ in $\mathbb{C}$ is bounded if there exists some $M \in \mathbb{R}$ such that $|z_k| \leq M$ for all $k \in \mathbb{N}$.

\item \emph{Prove that every Cauchy sequence in $\mathbb{C}$ is bounded.}

Let $(z_n)$ be a Cauchy sequence in $\mathbb{C}$, then there exists an $N \in \mathbb{N}$ such that $|z_n - z_N| < 1$ for all $n \geq N$.  Since the triangle inequality holds in $\mathbb{C}$, we have $|z_n| < 1 + |z_N|$ for all $n \geq N$.  Choose $M = \max\{|z_1|,|z_2|,|z_3|,\ldots,|a_{N-1}|, 1+|a_N|\}$.  Thus $|z_k| \leq M$ for all $k \in \mathbb{N}$.

\item \emph{Show that every bounded sequence in $\mathbb{C}$ has a convergent subsequence.}

\item
\begin{enumerate}
\item \emph{$S = \{z \in \mathbb{C} \mid |z|=1 \}$}

$S$

\item \emph{$S = \{z \in \mathbb{C} \mid \Re(z) > \Im(z)\}$}

$\{z \in \mathbb{C} \mid \Re(z) \geq \Im(z)\}$

\item \emph{$S = \{a+bi \mid a,b \in \mathbb{Q} \}$}

$\mathbb{C}$

\item \emph{$S = \{a+bi \mid a,b \in \mathbb{Z} \}$}

$\emptyset$

\end{enumerate}

\item
\begin{enumerate}
\item \emph{Show that if $S$ is a subset of $\mathbb{C}$ then every neighborhood of an accumulation point of $S$ contains infinitely many points of $S$.}

Assume for contradiction that there is some open ball $B$ around $z$ which contains only a finite number of points in $\mathbb{C}$.  Let $r_1,r_2,\ldots,r_n$ be those points of $B \cap \mathbb{C}$ which are distinct from $z$ and define 

 \[ r = \min_{1 \leq i \leq n}\{|z - r_i|\}\]

The minimum of a finite positive set of elements is positive, so $r > 0$ and $B_r(z)$ contains no point of $\mathbb{C}$ distinct from $z$.  Therefore $z$ is not an accumulation point, a contradiction.

\item \emph{Prove that every bounded infinite set in $\mathbb{C}$ has an accumulation point in $\mathbb{C}$.}
\end{enumerate}

\item \emph{Heine-Borel for $\mathbb{C}$}

\item 
\begin{enumerate}
\item \emph{Show that a subset of $\mathbb{C}$ is closed if and only if it contains all its accumulation points.}
\item \emph{Show that the intersection of arbitrarily many closed sets is closed.}

Let $\{S_\alpha\}$ be a collection of closed sets.  By (\ref{OpenClosedUnionIntersect}), $\left(\bigcap_\alpha S_\alpha \right)^c = \bigcup_\alpha \left(S_\alpha^c\right)$.  $S_\alpha^c$ is open since $S_\alpha$ is closed, so  $\bigcup_\alpha \left(S_\alpha^c\right)$ is open.  Hence $\left(\bigcap_\alpha S_\alpha \right)^c$ is open and therefore $\bigcap_\alpha S_\alpha$ is closed.

\item \emph{Show that the finite union of closed sets is closed.}

Let $\{S_n\}$ be a finite collection of closed sets.  By (\ref{OpenClosedUnionIntersect}), $\left(\bigcup_{i=1}^n S_i \right)^c = \bigcap_{i=1}^n \left(S_i^c\right)$.  $S_i^c$ is open since $S_i$ is closed, so $\bigcap_{i=1}^n \left(S_i^c\right)$ is also open.  Hence $\left(\bigcup_{i=1}^n S_i \right)^c$ is open and therefore $\bigcup_{i=1}^n S_i$ is closed.

\end{enumerate}

\item \emph{Prove that if $z = e^\frac{2k \pi i}{n}$ for $k \in \mathbb{Z}$ and $0 \leq k \leq n$ then $z^n = 1$.}

We have $e^{\theta i} = \cos(\theta) + i\sin(\theta)$, so $z^n = (e^\frac{2k \pi i}{n})^n = e^{2k \pi i} = \cos(2k \pi) + i\sin(2k \pi) = 1 + i0 = 1$.
\item \emph{Construct $\mathbb{Q}(x)$ from $\mathbb{Q}[x]$.}

\item \emph{Apply the above construction to the ring $\mathbb{Z}[x]$ and show that you get something familiar.}

\item \emph{Show that the order given below for $\mathbb{Q}(x)$ is well-defined and non-Archimedian.}

Let $a(x)=1$ and $b(x) = 2x$.  Both of these are positive, so we need to find a $n \in \mathbb{N}$ such that $na(x) > b(x)$.  If such a number existed then $na(x) - b(x) = -2x + n$ would have a positive leading coefficient, which is clearly impossible.  Therefore this order defined on $\mathbb{Q}(x)$ is non-Archimedian.

\item \emph{Let $R$ be an ordered integral domain and let $K$ be the field of fractions of $R$.  Show that there is a unique order on $K$ that preserves the order on $R$.}
\end{enumerate}
\end{document}
