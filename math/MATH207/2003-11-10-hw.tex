\documentclass[11pt]{article}
\textwidth = 6.5 in
\textheight = 9 in
\oddsidemargin = 0.0 in
\evensidemargin = 0.0 in
\topmargin = 0.0 in
\headheight = 0.0 in
\headsep = 0.0 in
\parskip = 0.2in
\parindent = 0.0in
\usepackage{amsfonts}
\usepackage{amsmath}
\usepackage{amssymb}
\usepackage{pslatex}
\title{MATH 207: Problem Set 6}
\author{Jesse Farmer}
\date{10 November 2003}
\begin{document}
\maketitle
\begin{enumerate}
\item \emph{Let $F$ be a field and $A \in M_2(F)$.  Show that $A$ has a inverse if and only if $det(A) \neq 0$}

By question 2, $A$ is invertible if and only if $det(A)$ is invertible in $F$.  But since $F$ is a field the only element that is not invertible is $0$.  Therefore $A$ is invertible if and only if $det(A) \neq 0$;

\item \emph{Let $R$ be a commutative ring with one and $A \in M_2(R)$.  When does $A$ have an inverse?}

\underline{Claim}: $A$ is invertible if and only if $det(A)$ is invertible in $R$.

\underline{Lemma}: For any matrices $A,B \in M_2(R)$, $det(AB)=det(A)det(B)$

Let $A=\left( \begin{array}{cc} a_{11} & a_{12} \\ a_{21} & a_{22} \\ \end{array} \right)$ and $B=\left( \begin{array}{cc} b_{11} & b_{12} \\ b_{21} & b_{22} \\ \end{array} \right)$

\begin{tabular}{lll}
$det(AB)$ 	&=& $det(\left( \begin{array}{cc} a_{11}b_{11} + a_{12}b_{21} & a_{11}b_{12} + a_{12}b_{22} \\ a_{21}b_{11} + a_{22}b_{21} & a_{21}b_{12} + a_{22}b_{22} \\ \end{array} \right))$\\ 
		&=& $a_{12} a_{21} b_{12} b_{21} - a_{11} a_{22} b_{12} b_{21} - a_{12} a_{21} b_{11} b_{22} + a_{11}a_{22}b_{11}b_{22}$\\ 
		&=& $(a_{11}a_{22} - a_{12}a_{21})(b_{11}b_{22} - b_{12}b_{21})$ \\ 
		&=& $det(A)det(B)$
\end{tabular}

Assume $A$ is invertible, then there exists a $B \in M_2(R)$ such that $AB=I_2$.  But this means $det(AB)=det(I_2)=1$.
Since the determinent is distributive, $1 = det(AB) = det(A)det(B) \Rightarrow det(B)=det(A)^{-1}$.  That is, $det(A)$ is invertible.

Assume $det(A)$ is invertible and let 
$A=\left( \begin{array}{cc} a & b \\ c & d \\ \end{array} \right)$.  

We see that $det(A)=ad-bc$, and $A\cdot\frac{1}{ad-bc}
\left( \begin{array}{rr} d & -b \\ -c & a \\  \end{array} \right) = 
\frac{1}{ad-bc}\left( \begin{array}{cc} ad-bc & 0 \\ 0 & ad-bc \\  \end{array} \right) =
\left( \begin{array}{cc} 1 & 0 \\ 0 & 1 \\  \end{array} \right) = I_2$.  That is, $A$ is invertible.

Therefore $A$ is invertible if and only if $det(A)$ is invertible in $R$.

\item \emph{What are the zero divisors of $M_2(R)$?}

\underline{Claim}: $A$ is a zero divisor if and only if $A$ is not invertible.

Assume $A$ is is both invertible and a zero divisor.  Then there exists a $0 \neq B \in M_2(R)$ such that $AB=0$.
$0 = A^{-1}0 = A^{-1}(AB) = (A^{-1}A)B = I_2B = B$, but by hypothesis $B \neq 0$. 

The proof of the converse escapes me.

\item \emph{Let $(R,+,\cdot,<)$ be an ordered integral domain.  Show that $R$ has a subring which is order isomorphic to $\mathbb{Z}$.}

Let $R \supset S^+=\{1, 1+1, 1+1+1,\ldots\}$.  This set must be infinite, since, if it were not, it would not be ordered (as e but every subset of an ordered set is necessarily ordered.  Assume for contradiction that $S^+$ is not well-ordered; that is, there is some non-empty subset which does not have a leave element.  Every subset is bounded below by $1$, so $1$ is not in this subset.  Likewise, if $n$ is not in this subset then $n+1$ cannot be in this subset since that would then be the least element.  This subset must be empty, a contradiction of our assumption that this was a non-empty subset.  Therefore $S^+$ is well ordered.

If we take $-1 \cdot S^+ = S^-$, then this is the set of additive inverses of the elements in $S^+$.  Letting $S= S^+ \bigcup \{0\} \bigcup S^-$, we see that we now have additive inverses and  Therefore, by Problem 13 on Homework 3, this is order isomorphic to $\mathbb{Z}$.

\item \emph{Show that $\mathbb{Q}$ does not satisfy the least upper bound property.}

Let $S = \{p \in \mathbb{Q} | p^2 < 2\}$.  We know that $p^2 = 2 \Rightarrow p \notin \mathbb{Q}$.  So we want to show that this number is the upper bound.  Let $\alpha = \sup{S} \in \mathbb{Q}$.

\begin{itemize}
\item Assume $\alpha^2 > 2$

$\alpha^2 > 2 \Leftrightarrow \alpha^2 + 2\alpha > 2 + 2\alpha \Leftrightarrow \alpha > \frac{2\alpha + 2}{\alpha+2}$

\begin{tabular}{lll}
$(\frac{2\alpha + 2}{\alpha+2})^2 > 2$ 	& $\Leftrightarrow$ 	& $4\alpha^2 + 8\alpha + 4 > 2\alpha^2 + 8\alpha + 8$ \\
					& $\Leftrightarrow$	& $4\alpha^2 > 2\alpha^2 + 4$ \\
					& $\Leftrightarrow$	& $2\alpha^2 > 4$ \\
					& $\Leftrightarrow$	& $\alpha^2 > 2$
\end{tabular}

So $\frac{2\alpha + 2}{\alpha+2}$ is an upper bound but less than $\alpha$, contradicting the assumption that $\alpha = \sup{S}$.

\item Assume $\alpha^2 < 2$

$\alpha^2 < 2 \Leftrightarrow \alpha^2 + 2\alpha < 2 + 2\alpha \Leftrightarrow \alpha < \frac{2\alpha + 2}{\alpha+2}$

\begin{tabular}{lll}
$(\frac{2\alpha + 2}{\alpha+2})^2 < 2$ 	& $\Leftrightarrow$ 	& $4\alpha^2 + 8\alpha + 4 < 2\alpha^2 + 8\alpha + 8$ \\
					& $\Leftrightarrow$	& $4\alpha^2 < 2\alpha^2 + 4$ \\
					& $\Leftrightarrow$	& $2\alpha^2 < 4$ \\
					& $\Leftrightarrow$	& $\alpha^2 < 2$
\end{tabular}

So $\frac{2\alpha + 2}{\alpha+2} \in S$ but greater than $\alpha$, contradicting the assumption that $\alpha = \sup{S}$.
\end{itemize}
Hence the supremum must satisfy $\alpha^2 = 2$, but no rational number does this.  Therefore we have a bounded set of rationals, $S$, whose supremum is not in $\mathbb{Q}$.  That is, $\mathbb{Q}$ does not satisfy the least upper bound property.

\item \emph{Show that there is a bijection from the normal subgroups of $\frac{G}{N}$ and the normal subgroups of $G$ containing $N$ if $N \unlhd G$.}

We know that the function defined by $N \mapsto \varphi^{-1}(N)$ is a bijection between subgroups of $\frac{G}{N}$ and subgroups of $G$ containing $N$, so all that is left to show is that, given some $H \leq G$ and $\bar{H}=\frac{N}{N} \leq \frac{G}{N}$, this map preserves normality. 

Define $f: \frac{G}{N} \rightarrow \frac{G}{H}$ by $f(xN)=xH$.  Since the subgroups are normal, we see that this is a well-defined homomorphism whose kernel is $\frac{H}{N}$ and whose image is $\frac{G}{H}$.  By the first isomorphism theorem we see that $\frac{G/N}{G/N}$ is isomorphic to $\frac{G}{H}$.

I believe this implies our statement about normality, but I'm not sure how.

\item \emph{Let $G$ be a finite group and $H$ be a subgroup of $G$ with index $k$.  Show that there exists a set of elements $x_1,x_2,\ldots,x_k$ in $G$ which can serve as complete coset representatives for both left and right cosets of $H$.}

\item \emph{Find all possible areas of lattice squares in $\mathbb{R}^2$.}
Every lattice square in $\mathbb{R}^2$ is generated by constructing a line between the origin and an arbitrary point $(a,b)$ where $a,b \in \mathbb{Z}$.  The area of this square is $a^2 + b^2$, so an integer is the area of a lattice square if and only if it is the sum of two squares.  Thus, we must discover which integers are the sum of two squares.

\underline{Claim}: A positive integer n can be represented as the sum of two squares if and only if its prime factorization contains no odd powers of primes congruent to 3 modulo 4.

\item \emph{Find all positive integers which can be the length of the hypotenuse of a right triangle with legs of integer length.}
 
\item \emph{Define a polyhedron in $\mathbb{R}^n$.}

It is a union of s-simplices for with $s \leq r$, that is closed under intersection, and such that the only time that one of simplices is contained in another is as a face.  An n-simplex is the convex hull of (n+1) points in some Euclidian space.

\item \emph{Find a Cauchy sequence in $\mathbb{Q}$ which does not converge in $\mathbb{Q}$.}

Define 
\[ a_n = \left\{
              \begin{array}{ll}
                   1 & n = 1\\
                   \frac{2a_{n-1} + 2}{a_{n-1}+2} & n > 1
              \end{array}
       \right.
\]

By the previous problem using this sequence, we see that $1>a_2>a_3>\ldots>a_n>\ldots>\sqrt{2}$, so this set is clearly bounded in $\mathbb{Q}$.  Moreover, this sequence is clearly Cauchy since it is strictly decreasing and bounded.  Assume it converges in $\mathbb{Q}$, then $\lim_{n\to\infty}a_n=L$ and $\lim_{n\to\infty}a_{n+1}=L$, but $a_{n+1} = \frac{2a_n+ 2}{a_n+2}$.  Thus $\frac{2L+2}{L+2} = L \Rightarrow L^2+2L=2L+2 \Rightarrow L=\sqrt{2}$.  But then $L \notin \mathbb{Q}$.  So $(a_n)$ is Cauchy but does not converge in $\mathbb{Q}$.

\item \emph{Show that if $(a_n),(b_n)$ are Cauchy sequences then $(a_n + b_n)$ is a Cauchy sequence.}

We have 
$\forall r>0 \exists N_1 \in \mathbb{N} \ni m,n>N_1 \Rightarrow |a_n - a_m|<\frac{r}{2}$ and 
$\forall r>0 \exists N_2 \in \mathbb{N} \ni m,n>N_2 \Rightarrow |b_n - b_m|<\frac{r}{2}$

Let $N=max(N_1,N_2)$, then  $\forall r>0,|(a_n + b_n) - (a_m + b_m)| \leq |a_n - a_m| + |b_n - b_m| < \frac{r}{2} + \frac{r}{2} = r$

Therefore if $(a_n)$ and $(b_n)$ are Cauchy sequences then so is $(a_n + b_n)$.

\item \emph{Let $(R,+,\cdot)$ be a ring with one. Show that $(R^x,\cdot)$ is a group.}
\begin{itemize}
\item Associativity is inhereted.
\item Each element has an inverse by definition of $(R^x,\cdot)$.
\item $1\cdot1 = 1$, so there is an identity in $R^x$
\item Let $a,b \in R^x$, then $a^{-1},b^{-1}$ exist.  $(b^{-1}a^{-1})(ab) = b^{-1}(a^{-1}a)b = b^{-1}b = 1$.  So the operation is an internal law of composition.
\end{itemize}

\item \emph{Describe $(\mathbb{Z}^x_n,\cdot)$ for $2 \leq n \leq 16$.}

\begin{tabular}{r|l|l}
n	&	elements			& Isomorphic group \\
\hline
2	&	1				& $\{e\}$ \\ 
3	&	1,2				& $(\mathbb{Z}_2,+)$ \\ 
4	&	1,3				& $(\mathbb{Z}_2,+)$ \\ 
5	&	1,2,3,4				& $(\mathbb{Z}_4,+)$ \\ 
6	&	1,5				& $(\mathbb{Z}_2,+)$ \\ 
7	&	1,2,3,4,5,6			& $(\mathbb{Z}_6,+)$ \\ 
8	&	1,3,5,7				& $(\mathbb{Z}_2 \times \mathbb{Z}_2,+)$ \\ 
9	&	1,2,4,5,7,8			& $(\mathbb{Z}_3 \times \mathbb{Z}_2,+)$ \\ 
10	&	1,3,7,9				& $(\mathbb{Z}_4,+)$ \\ 
11	&	1,2,3,4,5,6,7,8,9,10		& $(\mathbb{Z}_{10},+)$ \\ 
12	&	1,5,7,11			& $(\mathbb{Z}_2 \times \mathbb{Z}_2,+)$ \\ 
13	&	1,2,3,4,5,6,7,8,9,10,11,12	& $(\mathbb{Z}_{12},+)$ \\ 
14	&	1,3,5,9,11,13			& $(\mathbb{Z}_3 \times \mathbb{Z}_2, +)$ \\ 
15	&	1,2,4,7,8,11,13,14		& $(\mathbb{Z}_5 \times \mathbb{Z}_3,+)$ \\
16	&	1,3,5,7,9,11,13,15		& $(\mathbb{Z}_4 \times \mathbb{Z}_4,+)$
\end{tabular}

\item \emph{Let $C$ be the set of all Cauchy sequences in $\mathbb{Q}$. Show that $I=\{(a_n) \in C | a_n \rightarrow 0\}$ is a maximal ideal.}

First we show that  $I$ is an ideal.  Let $(a_n),(b_n) \in I$, then we have $\forall r>0 \exists N_1 \in \mathbb{N} \ni n>N \Rightarrow |a_n|<\frac{r}{2}$ and $\forall r>0 \exists N_2 \in \mathbb{N} \ni n>N_2 \Rightarrow |b_n|<\frac{r}{2}$.  Let $N=max(N_1,N_2)$ then $\forall r>0$ we have $|a_n + b_n| \leq |a_n| + |b_n| < \frac{r}{2} + \frac{r}{2} = r$.  Therefore $(a_n),(b_n) \in I \Rightarrow (a_n + b_n) \in I$.  Now, let $(b_n) \in C$.  Because $(b_n)$ is Cauchy it is eventually bounded from above by some constant, call it $A$.  Choose $N$ so that $|a_n| < \frac{r}{A}$.  Then $|a_nb_n|<\frac{r}{A}A = r$ for $n \geq N$.  $I$ is therefore an ideal of $C$.

To prove that $I$ is maximal it suffices to show that for any $(a_k) \in C$ the ideal generated by $I$ and $(a_k)$ is equal to $C$.  Because $(a_k) \notin I$ there is an $M$ such that $a_k$ is always nonzero for $k \geq M$.  Define $r_k$ as follows ($b_k$ is any element of $C$):
\[ r_k = \left\{
              \begin{array}{ll}
                   \frac{b_k}{a_k} & k \geq M\\
                   1 & k < M
              \end{array}
       \right.
\]

$r_k$ is Cauchy since, 
\[|\frac{a_n}{b_n} - \frac{a_m}{b_m}| = \frac{1}{|a_na_m|}|b_na_m - a_nb_m| \]
If we choose $P$ such that $|a_k| \geq P$ for sufficiently large $k$ and choose $K$ such that $|b_k| \leq K$ for sufficiently large $k$ we have
\[|\frac{a_n}{b_n} - \frac{a_m}{b_m}| \leq \frac{K}{P^2}|a_n - a_m| \]
for sufficiently large $m,n$.  That this is Cauchy follows immediately from our assumption that $(a_k)$ is Cauchy.  Let $i_k = b_k - r_ka_k$ and note that this is eventually zero, i.e., $(i_k) \in I$.  Now $(b_k) = (r_k)(a_k) + (i_k)$, but the terms on the right-hand side belong to the ideal generated by $I$ and $(a_k)$.  Therefore $(b_k)$ is in this ideal, i.e., this ideal is all of $C$.  Therefore, $I$ is a maximal ideal.
\end{enumerate}
\end{document}
