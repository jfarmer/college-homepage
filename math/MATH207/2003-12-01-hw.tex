\documentclass[12pt]{article}
\textwidth = 6.5 in
\textheight = 9 in
\oddsidemargin = 0.0 in
\evensidemargin = 0.0 in
\topmargin = 0.0 in
\headheight = 0.0 in
\headsep = 0.0 in
\parskip = 0.2in
\parindent = 0.0in
\usepackage{amsfonts}
\usepackage{amsmath}
\usepackage{amssymb}
\title{MATH 207: Homework \#9}
\author{Jesse Farmer}
\date{01 December 2003}
\begin{document}
\maketitle
\begin{enumerate}
\item \emph{Find $Aut D_{2n}$ and $Aut S_n$.}

Let $\varphi: D_{2n} \rightarrow D_{2n}$ be an automorphism.  We automatically have, then, $\varphi(I)=I$ and for any $x \in D_{2n},\varphi(x^{-1})=\varphi(x)^{-1}$.  Moreover, we see that if the order of an element $x$ is $n$ then the order of $\varphi(x)$ is also $n$.

First we consider only the subgroup of rotations, specifically $\varphi(r)$.  An element is a generator of this subgroup if and only if it has order $n$, but an automorphism preserves order.  Therefore an automorphism must send $r$ to another generator.  Trivially, if an element $r^k$ is a generator then the properties of an automorphism are preserved.  Therefore any automorphism of $D_{2n}$ sends $r$ to some generator of the rotational subgroup.

We now consider $f$, the flip element.  As each flip is of order two, sending $f$ to any of $f,\ldots,fr^{n-1}$.  As it is impossible to have more automorphisms here, these are all the automorphisms.

As for the symmetric group, I haven't the slightest.

\item \emph{Find $Aut{\mathbb{Q}(\sqrt{2})}$.}

Let $\varphi: \mathbb{Q}(\sqrt{2}) \rightarrow \mathbb{Q}(\sqrt{2})$ be an automorphism and $a,b \in \mathbb{Q}$.  If $x_0 \in \mathbb{Q}$ we know that $\varphi(x_0) = x_0$, so, for any $a + b\sqrt{2} = y_0 \in \mathbb{Q}(\sqrt{2})$ we have $\varphi(y_0) = a + b\varphi(\sqrt{2})$

Notice that $2 = \varphi(2) = \varphi(\sqrt{2}\sqrt{2}) = \varphi(\sqrt{2})\varphi(\sqrt{2}) = \varphi(\sqrt{2})^2$.  Thus $\varphi(\sqrt{2}) = \pm \sqrt{2}$ and the only two automorphisms on $\mathbb{Q}(\sqrt{2})$ are $\varphi(a+b\sqrt{2}) = a+b\sqrt{2}$ and $\varphi(a+b\sqrt{2}) = a-b\sqrt{2}$.
\item \emph{Find $Aut{\mathbb{R}}$.}

Any automorphism $\varphi$ on $\mathbb{R}$ must fix $\mathbb{Q}$.  Consider $a \in \mathbb{R}, a > 0$.  Then there exists a $b \in \mathbb{R}$ such that $b^2 = a$ and $\varphi(a) = \varphi(b^2) = \varphi(b)^2 > 0$.  Next, $a<b \Rightarrow \varphi(b-a)>0 \Rightarrow \varphi(a)<\varphi(b)$.

Let $y \in \mathbb{R} \backslash \mathbb{Q}$.  Assume for contradiction that $y<\varphi(y)$.  There exists $r \in \mathbb{Q}$ such that $y<r=\varphi(r)<\varphi(y)$.  But then $\varphi(y)<\varphi(r)$ by the fact that $\varphi$ is increasing, a contradiction.

Likewise, assume for contradiction that $\varphi(y)<y$.  There exists $r \in \mathbb{Q}$ such that $\varphi(y)<\varphi(r)=r<y$.  But then $\varphi(r)<\varphi(y)$ by the fact that $\varphi$ is increasing, a contradiction.

Therefore $\varphi(x)=x$ for all $x \in \mathbb{R}$. 
\item \emph{Given $r \in \mathbb{Q}$ and $|r|_p = 1$ find $k \in \mathbb{Z}$ such that $|r-k|_p \leq \frac{1}{p^m}$ for $m \in \mathbb{N}$.}

Let $r=\frac{a}{b}$.  We need to find $k \in \mathbb{Z}$ such that $|\frac{a}{b}-k|_p \leq \frac{1}{p^m}$.  We know that $p \not| b$, so $|b|_p=1$.  Hence we need to find $k$ such that $|\frac{a}{b}-k|_p = |\frac{a-bk}{b}|_p = \frac{|a-bk|_p}{|b|_p} = |a-bk|_p \leq \frac{1}{p^m}$ for any $m \in \mathbb{N}$.

First consider $a \cong kb (\mbox{mod p})$.  For $k=0,\ldots,p-1$ we see this forms a reduced residue system, and thus that there exists a $k$ which satisfies the above congruence.  For each $n=0,\ldots,m-1$ we then have $m$ solutions.  Iterating this for each $n$, that is, for each $a \cong kb (\mbox{mod p}^n)$, we arrive at a reduced residue system with one less solution than the previous.  Therefore, there is one integer such that $|a-bk|_p \leq \frac{1}{p^m}$ which we find by repeating this process.

\item \emph{Find a sequence in $\mathbb{Q}$ which is Cauchy but does not converge under the p-adic norm.}

Choose $a_n = \sum_{k=1}^{n}p^k$.  Assuming $n>m$ without a loss of generality, we have $|a_n - a_m|_p = \frac{1}{p^{n+1}}$, so $(a_n)$ is Cauchy.  Assume $(a_n)$ converges, then for every $\epsilon>0$ there exists $N \in \mathbb{N}$ such that $n \geq N \Rightarrow |a_n-L|_p < \epsilon$. If $p \not| L$ then $|a_n-L|_p = 1$ for all $n$ and we are done, so assume $p|L$ and write $L$ in base-$p$.  Even if $L$ is rational we can do this, since the numerator and denominator are integers and they can certainly be written in base-$p$.  Moreover, we can split apart the sum of the denominator to get a sum of rational numbers times a prime power.  $|a_n - \sum_{k=1}^mb_kp^k|_p = |\sum_{k=1}^nc_kp^k|_p = max_{1 \leq k \leq n}\{|c_kp^k|_p\} > 0$.  Therefore we can pick an $\epsilon >0$ (anything smaller than this number) for which any $N$ will fail, i.e., this sequence does not converge.


\item \emph{Let $V \subset X'$ be an open set.  Show that $f: X \rightarrow X'$ is continuous if $f^{-1}(V)$ is open.}

Let $x_0 \in X$ be arbitrary and $\epsilon > 0$ be fixed.  $B_\epsilon(f(x_o))$ is open, so $f^{-1}(B_\epsilon(f(x_o))$ is also open by hypothesis.  Since $f(x_0) \in B_\epsilon(f(x_0))$ we have that $x_0 \in f^{-1}(B_\epsilon(f(x_o))$, and hence that there exists $\delta > 0$ such that $B_\delta(x_0) \subset f^{-1}(B_\epsilon(f(x_o))$.

If $\rho(x,x_0) < \delta$ then $x \in B_\delta(x_0)$ and, moreover, $x \in f^{-1}(B_\epsilon(f(x_o))$.  Therefore, $f(x) \in B_\epsilon(f(x_o))$, or, $\rho(f(x),f(x_0)) < \epsilon$.  That is, $f$ is continuous.
\item \emph{Let $V \subset X'$ be a closed set.  Show that $f: X \rightarrow X'$ is continuous if $f^{-1}(V)$ is closed.}

Let $f^{-1}(V)$ be closed.  

\underline{Claim}: $f^{-1}(V^c) = f^{-1}(V)^c$

\begin{tabular}{lll}
\underline{Proof}:	$x \in f^{-1}(V^c)$ 	& $\Leftrightarrow$ & $f(x) \in V^c$\\
						& $\Leftrightarrow$ & $f(x) \notin V$\\
						& $\Leftrightarrow$ & $x \notin f^{-1}(V)$\\
						& $\Leftrightarrow$ & $x \in f^{-1}(V)^c$
\end{tabular}

Then $f^{-1}(V)^c = f^{-1}(V^c)$ is open.  By the previous problem, this correspondence between open sets and their preimage impluies $f$ is continuous.

\item
\begin{enumerate}
\item \emph{Show that the set of all homeomorphisms between a set and itself is a group under composition.}

We know that the set of all bijections from a set to itself forms a group under composition, so all we must show is that composition preserves continuity.  We will denote the set of all homeomorphisms from $X$ to itself as $\mathbb{H}$.

Let $f,g \in \mathbb{H}$.  $f$ is continuous at every point in $X$, so it is certainly continuous at $g(a)$.  Then, there exists a $\delta_1>0$ such that for all $y \in X,\rho(y,g(a))<\delta_1 \Rightarrow \rho(f(y),f(g(a)))<\epsilon$ for all $\epsilon>0$.  In particular, $y=g(x)$.  Moreover, for some $delta>0$ we have $\rho(x,a)<\delta \Rightarrow \rho(g(x),g(a))<\epsilon$ for all $\epsilon>0$.  

That is, for all $\epsilon>0$ there exists $\delta>0$ such that $\rho(x,a)<\delta \Rightarrow \rho(g(x),g(a))<\delta_1 \Rightarrow \rho(f(g(x)),f(g(a)))<\epsilon$.

As we already have that the inverses are continuous, we now have that their composition is, too.  Therefore composition preserves continuity and the set of all homeomorphisms from a set to itself forms a group under composition.

\item \emph{Show that the set of all isometries between a set and itself is a group under composition.}

Let $f,g$ be isometries.  Since $g$ is a bijection from a set to itself, we have that there exist $x,y$ for any $a,b$ such that $g(x)=a$ and $g(y)=b$.  Then $\rho((f \circ g)(x),(f \circ g)(y)) = \rho(f(a),f(b)) = \rho(g(x),g(y)) = \rho(x,y)$.  Therefore the composition of isometries is also an isometry, and, as above, we inherit all the other properties of the group of bijections from a set to itself.
\end{enumerate}
\end{enumerate}
\end{document}
