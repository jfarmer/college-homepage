\documentclass[11pt]{article}
\textwidth = 6.5 in
\textheight = 9 in
\oddsidemargin = 0.0 in
\evensidemargin = 0.0 in
\topmargin = 0.0 in
\headheight = 0.0 in
\headsep = 0.0 in
\parskip = 0.2in
\parindent = 0.0in
\usepackage{amsfonts}
\usepackage{amsmath}
\usepackage{amssymb}

\def\bddots{\mathinner{\mkern1mu\raise1pt\hbox{.}\mkern2mu
        \raise4pt\hbox{.}\mkern2mu\raise7pt\vbox{\kern7pt\hbox{.}}\mkern1mu}}

\newtheorem{theorem}{Theorem}[section]
\newtheorem{lemma}[theorem]{Lemma}
\newtheorem{proposition}[theorem]{Proposition}
\newtheorem{corollary}[theorem]{Corollary}

\title{MATH 209: Homework \#2}
\author{Jesse Farmer}
\date{12 April 2004}
\begin{document}
\maketitle
\begin{enumerate}
\item \emph{Show that a $\sigma$-algebra is closed under set difference and countable intersections.}

Let $M$ be a $\sigma$-algebra and $\{A_i\}$ be a countable collection of sets such that $A_i \in M$ for every $i \in \mathbb{N}$.  By de Morgan's law and the definition of a $\sigma$-algebra,
\[
\left(\bigcap_{i \in \mathbb{N}} A_i\right)^c = \bigcup_{i \in \mathbb{N}} A_i^c \in M
\]

But this implies
\[
\bigcap_{i \in \mathbb{N}} A_i \in M
\]

Then, from the definition of set difference, if $A,B \in M$ we know $A,B^c \in M$, so by above
\[
A \setminus B = A \cap B^c \in M
\]
\item \emph{Show that a collection $M$ of subsets of $X$ is a $\sigma$-algebra if $X \in M$ and $M$ is closed under complementation and countable intersection.}

Clearly, if $M$ is closed under complementation then $X \in M$ if and only if $\varnothing \in M$.  Let $\{A_i\}$ be a countable collection of sets such that $A_i \in M$ for every $i \in \mathbb{N}$ and assume $M$ is closed under countable intersection, then again by de Morgan's law,
\[
\bigcup_{i \in \mathbb{N}} A_i = \left(\bigcap_{i \in \mathbb{N}} A_i^c\right)^c \in M
\]

Therefore $M$ is a $\sigma$-algebra.

\item \emph{Think deeply about the $\sigma$-algebra in $\mathbb{R}$ generated by the open intervals.}

If every open interval is included in this $\sigma$-algebra, then every open set is contained in the $\sigma$-algebra.  Because it is closed under complementation, every closed set is contained in the $\sigma$-algebra.  I am not sure what else we are supposed to ``think deeply'' about, not that this was very deep thinking.

\item \emph{Find a $\sigma$-algebra with cardinality $\aleph_0.$}

Let $M$ be a $\sigma$-algebra and assume $M$ is infinite.  We will prove
\[
|M| > \aleph_0
\]

and hence that it is impossible to find such a $\sigma$-algebra.  Since $M$ is infinite there is some $A_n \in M$ such that $A_n$ contains infinitely many other members of $M$.  Let $B$ be any member contained in $A_n$, then either $B$ or $A_n \setminus B$ contains infinitely many other members of $M$.  Let $A_{n+1}$ be whichever is and define
\[
B_n = A_n \setminus A_{n+1}
\]

Since $M$ is closed under set difference by Problem 1 it follows that $B_n \in M$.  For every $A \in P(\mathbb{N})$ we have
\[
\bigcup_{i \in A} B_i \in M
\]

For each $A$ this element is unique because the $\{B_i\}$ are pairwise disjoint, and hence
\[
|M| > \aleph_0
\]

\item \emph{Show that if there does not exist a $k \in \mathbb{N}$ such that $A_k$ has finite measure, then it might not be true that if $A_{k+1} \subset A_k$ then $\mu\left(\cap A_i\right) = \lim_{i \rightarrow \infty} \mu\left(A_i\right)$.}

Let $\mu$ be the counting measure and consider the following countable collection of sets
\[
A_i = \{n \in \mathbb{N} \mid n \geq  i\}
\]

Clearly $A_{i+1} \subsetneq A_i$ and $\mu(A_i) = +\infty$ for every $i \in \mathbb{N}$. Assume for contradiction that there is some $n \in \cap_{i \in \mathbb{N}} A_i$. Then $n \in A_i$ for all $i \in \mathbb{N}$, but $n \notin A_k$ where $k > n$. Hence
\[
\bigcap_{i \in \mathbb{N}} A_i = \varnothing
\]

and therefore, by our choice of measure, 
\[
\mu\left(\bigcap_{i \in \mathbb{N}} A_i\right) = 0
\]

However, for every $i \in \mathbb{N}$ we have $\mu(A_i) = +\infty$, so
\[
\lim_{i \rightarrow \infty} \mu\left(A_i\right) = +\infty \neq 0 = \mu\left(\bigcap_{i \in \mathbb{N}} A_i\right)
\]


\item \emph{Let $M$ be a $\sigma$-algebra and $\{A_i\}$ such that $A_i \in M$.  Prove that if $A_{n+1} \subset A_n$ and $\mu(A_i)$ is eventually finite, then $\mu\left(\cap A_i\right) = \lim_{i \rightarrow \infty} \mu\left(A_i\right)$.}


If $\mu(A_i)$ is eventually finite then $\mu\left(\cap_i A_i \right)$ is finite since $\mu$ is monotonic.  Let $B_n = A_n^c$, then $B_n \subset B_{n+1}$.  We know that $B_n \in M$ and
\begin{equation}
\label{sigunlim}
\mu\left(\bigcup_{i \in \mathbb{N}} B_i\right) = \lim_{i \rightarrow \infty} \mu(B_i)
\end{equation}

If $B \in M$ and $\mu(B)$ is finite, then by additivity $\mu(B^c) = \mu(X) - \mu(B)$, since $\mu(X)$ is at most $\infty$ and the difference is undefined only in the case that we consider $\infty - \infty.$  Since $\mu$ is monotonic, $\mu(A_k)$ is eventually finite and we have
\begin{eqnarray*}
\lim_{i \rightarrow \infty} \mu(B_i) &=& \lim_{i \rightarrow \infty} \mu(A_i^c) \\
&=& \mu(X) - \lim_{i \rightarrow \infty} \mu(A_i)
\end{eqnarray*}

Also by de Morgan's law,
\begin{eqnarray*}
\mu\left(\bigcup_{i \in \mathbb{N}} B_i\right) &=& \mu\left[\left(\bigcap_{i \in \mathbb{N}} B_i^c\right)^c\right] \\
&=& \mu\left[\left(\bigcap_{i \in \mathbb{N}} A_i\right)^c\right] \\
&=& \mu(X) - \mu\left(\bigcap_{i \in \mathbb{N}} A_i\right)
\end{eqnarray*}

Therefore, by (\ref{sigunlim}),
\[
\mu\left(\bigcap_{i\in\mathbb{N}} A_i\right) = \lim_{i \rightarrow \infty} \mu\left(A_i\right)
\]

\item \emph{Show that if $R$ is a rectangle then $\mu^\ast(R) = \mu(R)$.}

By definition,
\[
R =  (a_1,b_1) \times (a_2, b_2) \times \cdots \times (a_n, b_n)
\]

We define
\[
R_\epsilon = (a_1-\epsilon,b_1+\epsilon) \times (a_2-\epsilon, b_2+\epsilon) \times \cdots \times (a_n-\epsilon, b_n+\epsilon)
\]

Clearly each $R_\epsilon$ is a rectangle which covers $R$, and hence is a subset of all coverings of $R$.  In particular this means
\[
\inf\{\mu(U_i) - \mu(R)\} \leq \inf_{\epsilon>0} \{\mu(R_\epsilon) - \mu(R)\}
\]

where the former is taken over the set of all countable coverings of $R$.  Therefore to show that $\mu^\ast(R) = \mu(R)$ it is sufficient to show that
\[
\inf_{\epsilon>0} \{\mu(R_\epsilon) - \mu(R)\} = 0
\]

We have
\begin{eqnarray*}
\inf_{\epsilon>0} \{ \mu(R_\epsilon) - \mu(R)\} &=& \inf_{\epsilon > 0} \left\{\prod_{i=1}^n(b_1 - a_i + 2\epsilon) - \prod_{i=1}^n(b_i - a_i)\right\} \\
&=& \inf_{\epsilon > 0}\left\{\sum_{i=1}^n\alpha_i\epsilon^i + \prod_{i=1}^n(b_i - a_i) - \prod_{i=1}^n(b_i - a_i)\right\} \\
&=& \inf_{\epsilon > 0}\left\{\sum_{i=1}^n\alpha_i\epsilon^i\right\} \\
&=& 0
\end{eqnarray*}

Therefore
\[
\mu^\ast(R) = \mu(R)
\]
\item \emph{Show that the Lebesgue outer measure of a face of a rectangle is $0$.}

A rectangle $R$ in the extended reals is a cross product of $n$ intervals,
\[
R = (a_1,b_1) \times (a_2, b_2) \times \cdots \times (a_n, b_n)
\]

For any $k \in \mathbb{N}$, consider the $k^{th}$ face of the rectangle $R'_k$.  For every $\epsilon > 0$ this face is contained in the rectangle
\[
R_k = (a_1,b_1) \times (a_2, b_2) \times \cdots \times (a_k - \epsilon, a_k+\epsilon) \times \cdots \times (a_n, b_n)
\]

but
\[
\mu^\ast(R_k) \leq (n-1)\infty \cdot 2\epsilon
\]

If $A \subset B$ then $\inf B \leq \inf A$, so
\[
\mu^\ast(R'_k) = \inf_{k \in \mathbb{N}} \mu(U_k) \leq \inf_{\epsilon > 0} \mu(R_k) = 0
\]

where the $U_k$ are an arbitrary covering of $R'_k$ by at most countably many rectangles.

\item \emph{Prove that if $\{A_i\}$ is a finite or countable collection of sets then $\mu^\ast(\cup_iA_i) \leq \sum_i \mu^\ast(A_i)$.}

Let
\[
A = \bigcup_{i=1}^\infty A_i
\]

If $\mu^\ast(A_i) = +\infty$ for some $i \in \mathbb{N}$ then the conclusion is trivial, so we assume that $\mu^\ast(A_i) < +\infty$ for all $i \in \mathbb{N}$.  Given $\epsilon > 0$ there are coverings $\{E_{i,j}\}$ for $j \in \mathbb{N}$ of $A_i$ by open rectangles such that
\[
\sum_{j=1}^\infty \mu(E_{i,j}) \leq \mu^\ast(A) + \frac{\epsilon}{2^i}
\]

Then
\[
\mu^\ast(A) \leq \sum_{i=1}^\infty \sum_{j=1}^\infty \mu(E_{i,k}) \leq \sum_{i=1}^\infty \mu^\ast(A_i) + \epsilon
\]

\item \emph{Prove that the Lebesgue outer measure is an outer measure}

See the previous problem.

\item \emph{Prove that $\mathfrak{M}$, the set of all measurable sets, is a $\sigma$-algebra.}

In class we proved everything except closure under countable union.  Since $\mathfrak{M}$ is closed under complementation and finite union it is sufficient to prove this statement for a countable union of \emph{disjoint} measurable sets.  Let $\{A_n\}$ be a countable collection of disjoint measurable sets and define
\[
A = \bigcup_{k=1}^\infty A_k
\]

Since the outer measure $\mu^\ast$ is subadditive, to prove $A$ is measurable we only need to show that for any set $E \subseteq \mathbb{R}^n$
\[
\mu^\ast(E) \geq \mu^\ast(E \cup A) + \mu^\ast(E \setminus A)
\]

\begin{lemma}
\label{minsect_lem}
If $E_1, E_2, \ldots, E_n$ are a finite collection of disjoint measurable sets and $A \subseteq \mathbb{R}^n$ then
\[
\mu^\ast(A \cap [E_1 \cap E_2 \cap \cdots \cap E_n]) = \mu^\ast(A \cap E_1) + \cdots + \mu^\ast(A \cap E_n)
\]
\end{lemma}

\textbf{\underline{Proof}:}  It is sufficient to show this for two disjoint measurable sets, $E$ and $F$, since the above would then follow trivially by induction.  We know that $F$ is measurable, therefore
\[
\mu^\ast(A \cap [E \cup F]) = \mu^\ast(A \cap [E \cup F] \cap F) + \mu^\ast(A \cap [E \cup F] \cap F^c)
\]

But since $E$ and $F$ are disjoint we have
\[
A \cap (E \cup F) \cap F = A \cap F \mbox{ and } A \cap (E \cup F) \cap F^c = A \cap E
\]

Hence, by induction,
\begin{equation}
\label{minsect}
\mu^\ast(A \cap [E_1 \cap E_2 \cap \cdots \cap E_n]) = \mu^\ast(A \cap E_1) + \cdots + \mu^\ast(A \cap E_n)
\end{equation}

We can now prove that $\mathfrak{M}$ is closed under countable union.  Define
\[
B_k = \bigcup_{i=1}^k A_i
\]

Each $B_k$ is measurable and since $A^c \subseteq B_k^c$,
\[
\mu^\ast(A^c) \leq \mu^\ast(B_k^c)
\]

Then by (\ref{minsect}),
\begin{eqnarray*}
\sum_{k=1}^n\mu^\ast(E \cap A_n) + \mu^\ast(E \cap A^c) &=& \mu^\ast(E \cap B_n) + \mu^\ast(E \cap A^c)\\
&\leq& \mu^\ast(E \cap B_n) + \mu^\ast(E \cap B_n^c) \\
&=& \mu^\ast(E)
\end{eqnarray*}

This is true for any $n \in \mathbb{N}$, so by the subadditivity of $\mu^\ast$
\begin{eqnarray*}
\mu^\ast(E \cap A) &\leq& \sum_{k=1}^\infty \mu^\ast(E \cap A_k) + \mu^\ast(E \cap A^c) \\
&\leq& \mu^\ast(E)
\end{eqnarray*}
\item \emph{Do the following:}
\begin{enumerate}
\item \emph{If $A \subset \mathbb{R}^n$ is contained in a hyperplane prove that $A$ is measurable.}

Let $E \subseteq \mathbb{R}^n$ be any set.  Then
\[
\mu^\ast(E \cup A) + \mu^\ast(E \setminus A) = 0 + \mu^\ast(E) = \mu^\ast(E)
\]

\item \emph{If $A$ is as above what is $m(A)$?}

Since any hyperplane $H$ has a measure of $0$ and $A \subseteq H$, $m(A) = 0$.

\item \emph{Show that any rectangle (open, half-open, etc.) is measurable and its measure doesn't change if you add or remove some of its faces.}

We showed in class that any open rectangle is measurable.  From above, since the measure of any face is zero, adding or removing it, i.e., changing the intervals to closed or half-open has no effect on the measure of the rectangle.

\item \emph{Prove that any open set in $\mathbb{R}^n$ is measurable.}

In $\mathbb{R}$ we know that any open set can be covered by the disjoint union of open intervals, and that open intervals are measurable.  It follows that any open set in $\mathbb{R}$ is measurable.  I do not know how to extend this to $\mathbb{R}^n$.
\end{enumerate}
\end{enumerate}
\end{document}
