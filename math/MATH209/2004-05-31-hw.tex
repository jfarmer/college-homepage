\documentclass[letterpaper,11pt]{article}
\textwidth = 6.5 in
\textheight = 9 in
\oddsidemargin = 0.0 in
\evensidemargin = 0.0 in
\topmargin = 0.0 in
\headheight = 0.0 in
\headsep = 0.0 in
\parskip = 0.2in
\parindent = 0.0in
\usepackage{amsfonts}
\usepackage{amsmath}
\usepackage{amssymb}
\usepackage{amsthm}

\def\bddots{\mathinner{\mkern1mu\raise1pt\hbox{.}\mkern2mu
        \raise4pt\hbox{.}\mkern2mu\raise7pt\vbox{\kern7pt\hbox{.}}\mkern1mu}}

\newtheorem{theorem}{Theorem}
\newtheorem{lemma}[theorem]{Lemma}
\newtheorem{proposition}[theorem]{Proposition}
\newtheorem{corollary}[theorem]{Corollary}
\newtheorem{problem}[theorem]{Problem}

\newcommand{\C}{\mathbb{C}}
\newcommand{\R}{\mathbb{R}}
\newcommand{\N}{\mathbb{N}}
\newcommand{\Q}{\mathbb{Q}}
\newcommand{\T}{\mathbb{T}}
\newcommand{\Z}{\mathbb{Z}}
\newcommand{\ceil}[1]{\left\lceil#1\right\rceil}
\newcommand{\ip}[2]{\left(#1\mid#2\right)}
\title{MATH 209: Homework \#9}
\author{Jesse Farmer}
\date{31 May 2004}
\begin{document}
\maketitle
\begin{problem} 
Show that $\widehat{\left(\Z_n,+\right)} \cong \left(\Z_n,+\right)$
\end{problem}

$\Z_n$ is compact, so all characters are unitary.  Moreover, if $0 \neq m \in \Z_n$ is arbitrary then there exists $0 \neq n \in Z_n$ such that $mn = 0$.  Let $\psi$ be a character of $\Z_n$, then
\[
1 = \psi(0) = \psi(mn) = \psi(m)^n
\]
i.e., $\chi(m)$ is an $n^{th}$ root of unity.  Define 
\[
\psi_k(m) = \psi(km) = e^{\frac{i 2 \pi km}{n}}
\]

Clearly $\big\{ \psi_k \mid k \in \{0,1,\ldots,n-1\}\big\} = \widehat{\Z}_n$, since, for any $\psi$ we simply pick a root of unity to which $\psi(1)$ will map, and $\psi(m)$ is then generated by $\psi(1)$ to powers of $k \in \{0,1,\ldots,n-1\}$.  Consider the map from $\Z_n$ to $\widehat{\Z}_n$, $k \mapsto \psi_k$.  From its definition this map is clearly surjective and homomorphic.

Let $\psi_k = \psi_j$.  Then
\[
\psi_{k-j} = \psi_k\psi_{-j} = \psi_k \psi_j^{-1} = 1 \Rightarrow e^{\frac{i2\pi(k-j)m}{n}} = 1 \Rightarrow k=j \mbox{(mod n)} \mbox{ or } m = 0
\]

However, $m$ is not identically zero, therefore $k=j$ and $k \mapsto \psi_k$ is an injection.

\begin{problem}
Show that $f \in \mathcal{S}(\R)$ if and only if $\widehat{f} \in \mathcal{S}(\R)$.
\end{problem}

To show sufficiency first note that the Schwartz condition is equivalent to
\[
\sup_x |x^k f^{(n)}(x)| \leq C_{k,n} < \infty
\]

for all $k,n \in \N$, where $C$ is some constant that may depend on $k$ and $n$.  Moreover, every Schwartz function is Lebesgue integrable, so
\begin{eqnarray*}
(t^k \widehat{f}^{(n)})(t) &=& \frac{1}{\sqrt{2\pi}} \int_\R t^k (-ix)^n e^{-itx} f(x) \,dx \\
&=&\frac{1}{\sqrt{2\pi}} \int_\R \frac{1}{(-i)^k} \left(e^{-itx}\right)^{(k)} (-ix)^nf(x) \,dx \\
&=& \frac{(-i)^k}{\sqrt{2\pi}} \int_\R e^{-itx} \left((-ix)^nf(x)\right)^{(k)}
\end{eqnarray*}

Since $f^{(n)} \in \mathcal{S}(\R)$, this implies
\[
\sup_t|t^k \widehat{f}^{(n)}(t)| \leq \frac{1}{\sqrt{2\pi}} \int_\R \left|(x^n f(x))^{(k)}\right| < \infty
\]

For the other direction, consider the inverse Fourier transform (which exists because $\widehat{f}$ is Schwartz)
\[
f(x) = \frac{1}{\sqrt{2\pi}} \int_\R \widehat{f}(x) e^{itx} \,dx
\]

If $\widehat{f}$ is Schwartz the same argument as above shows that $f$ is Schwartz.
\begin{problem}
Show that $f \mapsto \widehat{f}$ is a bijection of $\mathcal{S}(\R)$.
\end{problem}

From the previous problem's arguments it is clear that $f$ is Schwartz if and only if $\widehat{f}$ is Schwartz if and only if the inverse Fourier transform $\check{f}$ is Schwartz.  Then, given arbitrary $g \in \mathcal{S}(\R)$, there exists a $f \in \mathcal{S}(\R)$ such that $\widehat{f} = g$, namely, $f = \check{g}$.  Therefore $f \mapsto \widehat{f}$ is surjective.

For injectivity, let $\widehat{f} = \widehat{g}$.  Then
\[
f(x) = \int_\R \widehat{f}(n) e^{inx} = \int_\R \widehat{g}(n) e^{inx} = g(x)
\]

Therefore $f \mapsto \widehat{f}$ is also an injection, and hence a bijection.
\begin{problem}
Find $f \in L^1(\R)$ such that $\widehat{f} \notin L^1(\R)$.
\end{problem}

Let $f(x) = e^{-x}\chi_{(0,\infty)}(x)$.  Then $f$ is the density function of a standard exponential distribution, and hence
\[
\int_\R f(x) \,dx = \int_0^\infty e^{-x} \,dx = 1
\]

Moreover, 
\[
\widehat{f}(t) = \int_0^\infty e^{-x(1+it)} \,dx = \frac{1}{1+it}
\]

since $(1+it)e^{-x(1+it)}$ is the density function of an exponential distribution with parameter $(1+it)$.  This function is clearly not integrable on the real line.

\newpage 
\begin{problem}
Find the Fourier series of $\varphi(e^{i\theta}) = \theta$, for $-\pi < \theta < \pi$.
\end{problem}

Recall that
\begin{equation}
\label{eulers_formula}
e^{in\theta} = \cos \theta + i\sin \theta \mbox{ and } e^{-in\theta} = \cos \theta - i\sin \theta
\end{equation}

\begin{eqnarray*}
\widehat{f}(n) &=& \frac{1}{2\pi} \int_{-\pi}^{\pi} \theta e^{-in\theta} \,d\theta \\
&=& \frac{1}{2\pi} \left(\frac{i\theta}{n} e^{-in\theta} \bigg|_{-\pi}^{\pi} - \int_{-\pi}^{\pi} \frac{i}{n} e^{-in\theta} \,d\theta \right)\\
&=& \frac{1}{2\pi} \left( \frac{i\theta}{n} e^{-in\theta} + \frac{1}{n^2} e^{-in\theta} \bigg|_{-\pi}^{\pi} \right) \\
&=& \frac{1}{2\pi} \frac{e^{i n \theta} + in\theta e^{-in\theta}}{n^2} \bigg|_{-\pi}^{\pi} \\
&=& \frac{1}{2\pi} \frac{in\pi(e^{-in\pi} + e^{in\pi}) + e^{-in\pi} - e^{in\pi}}{n^2} \\
&=& \frac{i(n\pi\cos(n\pi) - \sin(n\pi))}{\pi n^2} \mbox{ by ($\ref{eulers_formula}$)}
\end{eqnarray*}

And $\widehat{f}(0) = 0$.  Therefore
\[
f(x) = \sum_{n \in \Z}  \frac{i(n\pi\cos(n\pi) - \sin(n\pi))}{\pi n^2} e^{inx}
\]

\begin{problem}
Show that if $f$ and $g$ are smooth functions then $f \ast g$ is also smooth.
\end{problem}

Let $f,g \in C^\infty$.  Then
\[
(f \ast g)'(x) = \left(\int_G f(y)g(y^{-1}x) \,dy\right)' = \int_G f(y)g'(y^{-1}x) \,dy = (f\ast g')(x) = (f' \ast g)(x)
\]

Inductively, if $f,g \in C^\infty$ then $(f \ast g)^{(n)}(x) = (f \ast g^{(n)})(x) = (f^{(n)} \ast g)(x)$.  Since differentiability implies continuity, and the convolution is differentiable $n$ times for any $n \in \N$, it follows that every derivative is continuous, i.e., $(f \ast g) \in C^\infty$.
\begin{problem}
Show that $f \ast g = g\ast f$.
\end{problem}

Let $G$ be a locally compact Abelian group with a Haar measure $dy$. Consider the map $y \mapsto y^{-1}x$.  Then
\begin{equation*}
(f \ast g)(x) = \int_G f(y)g(y^{-1}x)\,dy = \int_G f(y^{-1}x)g((y^{-1}x)^{-1}x)\,dy = \int_G f(y^{-1}x)g(y)\,dy = (g \ast f)(x)
\end{equation*}

\newpage
\begin{problem}
Show that if $f$ is a smooth function then $\tilde{f}$ is also smooth.
\end{problem}

\begin{problem}
Show that $\sqrt[3]{\sqrt{108} + 10} - \sqrt[3]{\sqrt{108} - 10} = 2$.
\end{problem}

Note that
\[
(\sqrt{3}-1)^3 = 6\sqrt{3} - 10 \mbox{ and } (\sqrt{3}+1)^3 = 6\sqrt{3} + 10
\]

Then
\begin{eqnarray*}
\sqrt[3]{\sqrt{108} + 10} - \sqrt[3]{\sqrt{108} - 10} &=& \sqrt[3]{6\sqrt{3} + 10} - \sqrt[3]{6\sqrt{3} - 10} \\
&=& \sqrt[3]{(\sqrt{3}+1)^3} - \sqrt[3]{(\sqrt{3}-1)^3} \\
&=& 2
\end{eqnarray*}
\end{document}
