\documentclass[10pt]{article}

\usepackage{amsfonts}
\usepackage{amsmath}
\usepackage{amssymb}
\usepackage{amsthm}
\usepackage{eucal}
\usepackage{enumerate}
\usepackage{geometry}

\geometry{letterpaper}

\textwidth = 6.5 in
\textheight = 9 in
\oddsidemargin = 0.0 in
\evensidemargin = 0.0 in
\topmargin = 0.0 in
\headheight = 0.0 in
\headsep = 0.0 in
\parskip = 0.2in
\parindent = 0.0in

\newcommand{\brac}[1]{
\left\langle #1 \right\rangle
}

\newcommand{\powset}[1]{
\wp\left(#1\right)
}

\newcommand{\Aut}{\text{Aut}}
\newcommand{\Sym}{\text{Sym}}
\newcommand{\Syl}{\text{Syl}}

\newcommand{\N}{\mathbb{N}}
\newcommand{\Z}{\mathbb{Z}}
\newcommand{\Q}{\mathbb{Q}}
\newcommand{\R}{\mathbb{R}}
\newcommand{\A}{\mathbb{A}}
\newcommand{\C}{\mathbb{C}}

\newcommand{\T}{\mathcal{T}}

\newcommand{\iso}{\cong}

\newtheorem{lemma}{Lemma}

\title{MATH 258: Homework \#5}
\author{Jesse Farmer}
\date{09 February 2005}
\begin{document}
\maketitle
\begin{enumerate}

\item \emph{Let $K$ be a field an $R = K[x]$.  Let $S = \{ f \in K[x] \mid f = \sum a_i x_i, a_1 = 0\}$.}
\begin{enumerate}
\item \emph{Show that $S$ is a subring of $R$.}

Any polynomial with $a_i = 0$ for $i > 0$ is in $S$, so it suffices to show that $S$ is closed under addition and multiplication.  Let $f,g \in S$. Then $f+ g = \sum (a_i+b_i)x^i$, so if $a_1,b_1 = 0$, then $a_1 + b_1 = 0$.  Similarly, $f \cdot g = \sum c_i x_i$ where $c_i = \sum{k=0}^i a_k b_{i-k}$.  If $i = 1$, then $c_1 = a_0b_1 + a_1b_0 = 0$.  Therefore $S$ is closed under addition and multiplication, and hence is a subring.

\item \emph{Let $I = \{f \in S \mid f(0) = 0\}$.  Show that $I = (x^2, x^3)$.  Show that $S$ is not a principal ideal domain by showing that $I$ is not a principal ideal.}

Let $f(x) = \sum_{i} a_i x^i$.  Then $f(0) = 0$, so $f \in I$ if and only if $a_0 = a_1 = 0$.  Clearly any element of $I$ is such that $a_0 = a_1 = 0$, since there are no terms of degree less than $2$ in the polynomial.  To show the converse it suffices to show that any $x^i$ with $i \neq 1$ can be written as $(x^2)^j(x^3)^k$ for some $j,k \in \N$, since then $f$ will be the finite sum of such terms with coefficients in $K$.  So consider $x^i$.  If $i$ is even then we are done since $i = 2k$ and $x^i = (x^2)^k$.  So assume $i$ is odd with $i \neq 1$.  Then $i = 2k+1 = 2(k-1) + 3$ and hence $x^i = (x^2)^{k-1}x^3$.  Therefore $I = (x^2, x^3)$.

To see that the idea is not principal assume for contradiction that there exists some $p \in S$ such that $(p(x)) = (x^2, x^3)$.  Then there exist $r,q \in I$ such that $x^2 = q(x)p(x)$ and $x^3 = r(x)p(x)$.  Since none of the polynomials are zero, $\deg q + \deg p = 2$, and hence $\deg p = 2$ since it is impossible that $\deg p$ is $1$ or $2$.  Moreover, $\deg r = 1 + \deg q$ since $\deg x^3 = 3$.  Then $\deg r = 1$, which is impossible by virtue of being in $I$.  Therefore $I$ cannot be principal.

\item \emph{Show that $S = K[x^2, x^3]$.}

$K[x^2, x^3] = \{\sum a_{ij} (x^2)^i (x^3)^j \mid a_{ij} \in K, i,j \in \N\}$.  But, from above, any $x^k$ with $k \neq 1$ can be generated by products of powers of $x^2$ and $x^3$, and no product of such powers has a degree of $1$ from the fact that there are no positive integral solutions to the equation $2k + 3j = 1$.  Therefore $S = K[x^2, x^3]$.

\end{enumerate}

\item \emph{Show that $R=\Z[2i]$ is not a principal ideal domain.}

From Theorem 14 on pp. 287 it suffices to show that $R$ is not a unique factorization domain.  This is obvious since $i \notin R$ and $4 = 2 \cdot 2 = (2i) \cdot (-2i)$, i.e., $4$ does not have a unique factorization.

Alternatively, consider the ideal $(2, 2i) = \{2a + 2bi \mid a,b \in \Z\}$.  Assume this ideal were principal, i.e., there exists some $a+2bi$ such that $(2,2i) = (a+2bi)$.  Then there would also exist $\alpha, \beta \in R$ with $2 = \alpha(a+2bi)$ and $2i = \beta(a+2bi)$.  Let $N$ be the associated field norm so that $4 = N(\alpha)(a^2 + 4b^2)$ and $4 = N(\beta)(a^2 + 4b^2)$.  Then $a^2 + 4b^2$ must be one of $1$, $2$, or $4$.  If it is $4$, then $N(\alpha) = N(\beta) = 1$ and $\beta = \pm 1$ and $\alpha = \pm 1$.  But clearly this is a contradiction since this implies $2 = \pm 2i$.  $a^2 + 4b^2$ cannot be $2$ since there are no integral solutions.  If $a^2 + 4b^2 = 1$ then $a+2bi = \pm 1$ and therefore $1 \in (2, 2i)$.  Then there exist $j,k$ such that $1 = 2k + 2ij$.  Multiplying by $-2i$ shows that $i$ must be a multiple of $2$, a contradiction.  Therfore $(2, 2i)$ is not principal and $R$ is not a principal ideal domain.

\item \emph{Let $A = \Z[\sqrt{2}]$.  Show that $A$ is a Euclidean Domain with norm $\nu:A \rightarrow \N$ defined by $$\nu(a+b\sqrt{2}) = |a^2 - 2b^2|$$}

Recall that a Euclidean Domain $A$ is an integral domain with a Euclidean norm defined such that, for any $\alpha, \beta \in A$ there exist $q,r \in A$ with $\alpha = q\beta + r$ and $N(r) < N(q)$.  Let $A = \Z[\sqrt{2}]$ and $\alpha, \beta \in \Z[\sqrt{2}]$.  Define $N(a + b\sqrt{2}) = |a^2 - 2b^2|$.

Then the above definition is clearly equivalent to the following:  For any $\delta \in \Q[\sqrt{2}]$ there exists a $\kappa \in \Z[\sqrt{2}]$ such that $N(\delta - \kappa) < 1$.  On $\Q[\sqrt{2}]$ we define $N$ in the obvious way: $N(a/b) = N(a)/N(b) \in \Q$.  Letting $\delta = r + s\sqrt{2}$ and $\kappa = x + y\sqrt{2}$, then we must choose $x,y$ such that
\[
|(r-x)^2 - 2(s-y)^2| < 1
\]

But this is easy to do: simply choose the integers closest to $r$ and $s$, so that $|r-x| \leq 1/2$ and $|s-y| \leq 1/2$.  Incidentally, this works also for any $\Z[\sqrt{m}]$ where $m < 5$, i.e., $\Z[\sqrt{3}]$ is also a Euclidean Domain.

\item \emph{Let $A = \{a + wb \mid a,b \in \Z\}$ where $w = \frac{-1 + i\sqrt{3}}{2}$.  Show that $A$ is a Euclidean Domain with norm $\nu:A \rightarrow \N$ defined by $$\nu(a+bw) = a^2 - ab + b^2$$}

As above, it is sufficient to find $\kappa \in \Z[w]$ such that for any $\delta \in \Q[w]$ we have $N(\delta - \kappa) < 1$.  Letting $\delta = r+sw$ and $\kappa = x + yw$, then
\[
N(\delta - \kappa) = (r-x)^2 - (r-x)(s-y) + (s-y)^2
\]

Choosing $x,y$ to be the cloest integers to $r,s$ so that $|r-x| \leq 1/2$ and $|s-y| \leq 1/2$ gives the estimate
\[
|N(\delta - \kappa)| = |(r-x)^2 - (r-x)(s-y) + (s-y)^2| \leq |r-x|^2 + |r-x||s-y| + |s-y|^2 \leq \frac{3}{4} < 1 
\]

Therefore $N$ is indeed a Euclidean norm on $\Z[w]$ and $\Z[w]$ is a Euclidean Domain.

\item \emph{Let $K$ be a field and $f \in K[x]$ with $f(x) = \prod_{i=1}^n (x-a_i)$ and $a_i \neq a_j$ for $j \neq i$.  Show that the map $K[x] / (K[x] \cdot f) \rightarrow \underbrace{K \times \cdots \times K}_{n~times}$ given by $\bar{g} \mapsto (g(a_1), \ldots, g(a_n))$ is a ring isomorphism.}

Let $\varphi(g) = (g(a_1), \ldots, g(a_n))$.  We will first show that this is a ring homomorphism with kernel $K[x] \cdot f$, and then that it is surjective, proving the statement.  That it is a homomorphism is almost too trivial for words, since
\[
\varphi(g) + \varphi(h) = (g(a_1), \ldots, g(a_n)) + (h(a_1), \ldots, h(a_n)) = ((g+h)(a_1), \ldots, (g+h)(a_n)) = \varphi(g+h)
\]

and \emph{mutatis mutandis} for multiplication.  $g \in \ker \varphi$ if and only if $g(a_i) = 0$ for $1 \leq i \leq n$.  But this is equivalent to saying that $x - a_i \mid g$ for each $a_i$, and hence $f \mid g$.  Therefore $\ker \varphi = K[x] \cdot f(x)$.

Showing that this map is surjective is more difficult, and requires (to my knowledge) some linear algebra.  We wish to find a function $g$ such that $g(x) = \sum_{i=1}^n g_i x^i$ and $g(a_i) = b_i$ for arbitrary $b_i \in K$.  But this is equivalent to the following matrix equation
\[
\left(
\begin{array}{ccccc} 
1 & x_1 & x_1^2 & \cdots & x_1^n \\
1 & x_2 & x_2^2 & \cdots & x_2^n  \\
\vdots & \vdots & \vdots & \ddots & \vdots  \\
1 & x_n & x_n^2 & \cdots & x_n^n 
\end{array}
\right)
\left(
\begin{array}{c}
g_1 \\ g_2 \\ \vdots \\ g_n
\end{array}
\right) = 
\left(
\begin{array}{c}
b_1 \\ b_2 \\ \vdots \\ b_n
\end{array}
\right)
\]

The determinant of this matrix is known to be $\prod_{i > j} (x_i - x_j)$.  In our case, we have $x_i = a_i$ and $a_i \neq a_j$ for $i \neq j$, so that this matrix (the Vandermonde matrix) is nonsingular and our homomorphism is surjective.

\item
\begin{enumerate}
\item \emph{Show that the ideal generated by ${2, x}$ in $\Z[x]$ is not a principal ideal.}

Assume there exists a nonzero polynomial $p \in \Z[z]$ such that $(p(x)) = (2,x)$.  Then there exist nonzero polynomials $q_1, q_2$ such that $2 = q_1(x)p(x)$ and $x = q_2(x)p(x)$.  But then $0 = \deg 2 = \deg q_1 + \deg p$, which implies $\deg p = 0$ and $\deg q_2 = 1$.  Then $x = \alpha x$ for some $\alpha \in \Z$ with $\alpha \neq 0$, which is a contradiction.  Therefore $(2,x)$ is not a principal ideal.

\item \emph{Let $K$ be a field and $A = K[x_1,x_2]$.  Show that the ideal generated by $x_1, x_2$ in $A$ is not a principal ideal.}

Assume $(x_1, x_2)$ is principal, then there exists some $g \in A$ such that $x_1 = g \cdot a$ and $x_1 = g \cdot b$ for some $a,b \in A$.  Since none of these are zero, it follows that the degree of $g$ in $x_1$ must be at least $1$.  But then the degree of $x_2 = g \cdot b$ in $x_1$ must be at least $1$, which is absurd.

\end{enumerate}

\item
\begin{enumerate}
\item \emph{Show that the ideal generated by $x^2 + 1$ in $\R[x]$ is a prime ideal.}

By Proposition 11 on pp. 284, a nonzero element of a PID is irreducible if and only if it is prime, and by Proposition 10 on pp. 308 a polynomial of degree two or three over a field $F$ is reducible if and only if it has a root in $F$.  Therefore, since $x^2+1$ has no real root, it is irreducible and therefore generates a prime ideal.

\item \emph{Decide if the ideal generated by $x^2 + 1$ in $\C[x]$ is a prime ideal.}

$x^2+1$ is reducible in $\C[x]$ by $x^2+1 = (x-i)(x+i)$.  By Proposition 10 on pp. 284, $(x^2+1)$ therefore is not a prime ideal.

\end{enumerate}

\item \emph{Let $K$ be a field and $A = K[x,y]$.  For $a,b \in K$ consider the ideal $M = A \cdot (x-a) + A \cdot (y-b)$.  Show that $M$ is a maximal ideal.}


Consider the map $\varphi: A \rightarrow K$ defined by $\varphi(f) = f(a,b)$.  This map is surjective since $f(x,y) = a$ maps to $a \in K$ for arbitrary $a$. Clearly $M \subset \ker \varphi$, so let $f \in \ker \varphi$.  Then there exists a $g \in K[x,y]$ and constant $r$ such that $f(x,y) = g(x,y)(x-a) + r(x,y)$.  Apply this again to $r(x,y)$, and we see that there exists a $h \in K[x,y]$ and constant $r'$ such that $$f(x,y) = g(x,y)(x-a) + h(x,y)(x-b) + r'(x,y)$$  Since $f(a,b) = 0$, $r = 0$, and therefore $f \in \ker \varphi$ and $M = \ker \varphi$.  By the first isomorphism theorm for rings, $A/M \iso K$, and hence $M$ is maximal in $A$.

\end{enumerate}
\end{document}
