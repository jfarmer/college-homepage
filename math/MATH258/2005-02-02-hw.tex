\documentclass[10pt]{article}

\usepackage{amsfonts}
\usepackage{amsmath}
\usepackage{amssymb}
\usepackage{amsthm}
\usepackage{eucal}
\usepackage{enumerate}
\usepackage{geometry}

\geometry{letterpaper}

\textwidth = 6.5 in
\textheight = 9 in
\oddsidemargin = 0.0 in
\evensidemargin = 0.0 in
\topmargin = 0.0 in
\headheight = 0.0 in
\headsep = 0.0 in
\parskip = 0.2in
\parindent = 0.0in

\newcommand{\brac}[1]{
\left\langle #1 \right\rangle
}

\newcommand{\powset}[1]{
\wp\left(#1\right)
}

\newcommand{\Aut}{\text{Aut}}
\newcommand{\Sym}{\text{Sym}}
\newcommand{\Syl}{\text{Syl}}

\newcommand{\N}{\mathbb{N}}
\newcommand{\Z}{\mathbb{Z}}
\newcommand{\Q}{\mathbb{Q}}
\newcommand{\R}{\mathbb{R}}
\newcommand{\A}{\mathbb{A}}
\newcommand{\C}{\mathbb{C}}

\newcommand{\T}{\mathcal{T}}

\newcommand{\iso}{\cong}

\newtheorem{lemma}{Lemma}

\title{MATH 258: Homework \#4}
\author{Jesse Farmer}
\date{02 February 2005}
\begin{document}
\maketitle
\begin{enumerate}

\item \emph{Let $p$ be prime.  Show that $p$ divides $\binom{p}{i}$ for $1 \leq i \leq p-1$.  Deduce that for $x,y$ elements of a commutative ring $A$ of characteristic $p$, $(x+y)^{p^n} = x^{p^n} + y^{p^n}$.}

\begin{lemma}\label{division_lemma} If $a,b,c$ are integers with $c \mid ab$ where $a$ and $c$ are relatively prime then $c \mid b$. \end{lemma}
\begin{proof}
Since $a$ and $c$  are relatively prime there exist integers $j$ and $k$ such that $$cj + ak = 1$$  And hence $$cbj + abk = b$$  By hypothesis there exists an integer $h$ such that $ab = hc$ and therefore $$c(bj + hk) = b$$
\end{proof}

Since $\binom{p}{i}$ is an integer and $$\binom{p}{i} = \frac{p(p-1)\cdots (p-i+1)}{i!}$$ we have $i! \mid  p(p-1) \cdots (p-i+1)$.  But $\text{g.c.d}(p,i!) = 1$ since $1 \leq i \leq p-1$.  From the above lemma, $i! \mid (p-1)\cdots (p-i+1)$, and hence
\[
\binom{p}{i} = p \cdot \frac{(p-1) \cdots (p-i+1)}{i!} = pk
\]

where $k$ is an integer.

Let $x,y \in A$ where $A$ is a commutative ring with $\operatorname{char}~A = p$ for some prime $p$.  Then $$(x+y)^p = \sum_{k=0}^p \binom{p}{k} \cdot x^k y^{p-k}$$

For $1 \leq k \leq p-1$ and some $j \in \Z$ $$\binom{p}{k} = pj = \underbrace{(1+1+\cdots+1)}_{p~times}j = 0$$  since $\operatorname{char}A = p$.  Hence $\binom{p}{k}x^ky^{p-k} = 0$ for all $1 \leq k \leq p-1$ and $(x+y)^p = x^p + y^p$.  Assume $(x+y)^{p^n} = x^{p^n} + y^{p^n}$.  Then $$(x+y)^{p^{n+1}} = \left((x+y)^{p^n}\right)^p = \left(x^{p^n} + y^{p^n}\right)^p = \left(x^{p^{n}}\right)^p + \left(y^{p^{n}}\right)^p = x^{p^{n+1}} + y^{p^{n+1}}$$

Therefore $(x+y)^{p^n} = x^{p^n} + y^{p^n}$ for all $n \in \N$.

\item \emph{Determine the ideals, prime ideals, and maximal ideals of $\Z/168\Z$.}

There is a one-to-one correspondence between ideals of $\Z/168\Z$ and ideals of $\Z$ that contain $168\Z$.  Since $\Z$ is a principal ideal domain, any ideal which contains $168\Z$ must be of the form $k\Z$ where $k \mid 168$.  The maximal ideals are those that correspond to the prime divisors of $168$.  The ideals are therefore all $(k\Z)/(168\Z)$ where $k \mid 168$, and the maximal ideals (and prime ideals) are all such $k$ that are prime, namely, $k=2,3,7$.

\item \emph{Let $p$ be a prime.  Show that $\Q(\sqrt{p})$ is a field.  Find all $q$ prime such that $\Q(\sqrt{p}) \iso \Q(\sqrt{q})$.}

Since addition is performed coordinate-wise $\Q(\sqrt{p})$ is clearly an abelian group with respect to addition.  Consider $(\Q(\sqrt{p}), \cdot)$.  We will treat $\Q(\sqrt{p})$ as a subset of $\Q \times \Q$ with multiplication defined by
\[
(a,b) \cdot (c,d) = (ac + pbd, ad+bc)
\]
\begin{eqnarray*}
(a,b)((c,d)(e,f)) &=& (a,b)(ce+pdf,cf+de) \\
&=& (ace+padf+pbcf+pbde,acf+ade+bce+pbdf) \\
&=& (ac+pbd,ad+bc)(e,f) \\
&=& ((a,b)(c,d))(e,f)
\end{eqnarray*}

The identity is $(1,0)$: $(a,b)(1,0) = (a + pb0, a0 + b) = (a,b)$, and the operation is commutative since addition and multiplication on $\Q$ are commutative.  The inverse of $(a,b) \neq 0$ is given by
\[
(a,b) \cdot \left(\frac{a}{a^2 - pb^2}, \frac{-b}{a^2 - pb^2}\right) = \left(\frac{a^2 - pb^2}{a^2 - pb^2}, \frac{ab-ab}{a^2 - pb^2}\right) = (1,0)
\]

since $a^2 = pb^2$ if and only if $a = \sqrt{p}b$ and hence $(a,b) = (0,0)$.  Distributivity is equally trivial:
\begin{eqnarray*}
(a,b)((c,d)+(e,f)) &=& (a,b)(c+e,d+f) \\
&=& (ac+ae+pbd+pbf,ad+af+bc+be) \\
&=&(ac+pbd,ad+bc)+(ae+pbf,af+be) \\
&=& (a,b)(c,d)+(a,b)(e,f)
\end{eqnarray*}

Hence $\Q(\sqrt{p})$ is a field.  Now let $f: \Q(\sqrt{p}) \rightarrow \Q(\sqrt{q})$ be an isomorphism of fields where $p,q$ are any two primes.  From the additive and multiplicative properties of $f$ is is fairly obvious that
\[
f(m,n) = (m,0)f(1,0) + (n,0)f(0,1)
\]

where $m,n \in \Q$. $f(1,0) = (1,0)$ from the fact that this is an isomorphism.  All that is left is to determine the value of $f(0,1)$.  Note that $f(0,1)^2 = f(p,0) = (p,0)$.  Hence we must find an $(a,b) \in \Q(\sqrt{q})$ such that $(a,b)^2 = (p,0)$.  But $(a,b)^2 = (a^2 + qb^2, 2ab)$.  If this equals $(p,0)$ then either $a = 0$ or $b = 0$.  If $b = 0$ then $a^2 = p$ and hence $a$ is not rational, so assume $a = 0$.  Then we must find a rational $b = m/n$ in lowest terms such that $pn^2 = qm^2$.  If $p = q$ then clearly $m=n=1$, so assume $p \neq q$.  Then $p \mid m^2$ and hence $p \mid m$, so that $m = kp$ and $n^2 = qpk^2$.  But then $p \mid n^2$ and $p \mid n$, so $m/n$ is not in lowest terms.  Hence there is no such $m/n \in \Q$ and $p = q$.  That is, $\Q(\sqrt{p}) \iso \Q(\sqrt{q})$ for $p,q$ primes if and only if $p = q$.

\item \emph{Let $A$ be a commutative ring and $R = A[U]$.  Show that if $f,g: R \rightarrow B$ are ring homomorphisms such that $f(x) = g(x)$ for all $x \in A \cup U$ then $f \equiv g$.}

Define $Z = \{ x \in C \mid f(x) = g(x)\}$, where $A \subset C$, and $U \subset C$ for some ring $C$.  Then for all $x,y \in Z$, $f(x-y) = f(x) - f(y) = g(x) - g(y) = g(x-y)$ and $f(xy) = f(x)f(y) = g(x)g(y) = g(xy)$.  Associativity and commutativity is inherited in the same way from $R$ and $B$, and therefore $Z$ is a subring of $C$.  Since $f$ and $g$ agree on $A \cup U$, $A \cup U \subset Z$.  Moreover, since $A[U]$ is by definition the smallest subring of $C$ containing $A \cup U$, it follows that $A[U] \subset Z$, and hence $f(x) = g(x)$ for all $x \in R$.

\item \emph{Find all the roots of $x^3 - x$ in $\Z_6[x]$.}

$x^3 - x = x(x^2 - 1)$, so that either if $x^3-x = 0$, $x = 0$, $x$ is its own inverse, or $x$ is a zero divisor.  Clearly $x=0$ and $x=1$ are roots.  $x=5$ is a root since $5^1 - 1 = 24 \equiv 0 \mod 6$.  $x=2$ is a root since $2(2^2-1) = 6 \equiv 0 \mod 6$.  $x = 3$ is a root since $3(3^2 - 1) = 24 \equiv 0 \mod 4$.  $x = 4$ is a root since $4(4^2-1) = 60 \equiv 0 \mod 6$.  So every element of $\Z_6$ is a root of this polynomial.

\item \emph{Let $F$ be a finite field.  Show that $\operatorname{char} F$ is prime and that $\prod_{a \in F^\times} a = -1$.  Deduce from this Wilson's Theorem: $(p-1)! \equiv -1 \mod p$ where $p$ is prime.}

The characteristic of a field $F$ is the smallest positive integer $p$ such that $$\underbrace{1+1+\cdots+1}_{p~times}$$ or $0$ is there is no such integer.  If $p$ is composite then $p=nk$ for some $n,k$ nonzero and less than $p$.  But then $$\underbrace{1+1+\cdots+1}_{nk~times} = \underbrace{(1+1+\cdots+1)}_{n~times}\underbrace{(1+1+\cdots+1)}_{k~times} = 0$$  Since $F$ is a field this means that one of $\underbrace{1+1+\cdots+1}_{n~times}$ or $\underbrace{1+1+\cdots+1}_{k~times}$ is zero, and hence that $p$ is not minimal.  Therefore, if $p$ is minimal, $p$ must be prime.

Now let $|F| = q < \infty$ so that $|F^\times| = q-1$.  Assume that $q > 2$ since for $q = 2$ then result is trivial: $1 = -1$ and $1$ is the only unit.  Consider $a \in F^\times$ such that $a^2 = 1$, then $a^1 - 1 = (a-1)(a+1) = 0$ and hence $a = \pm 1$.  Since $a$ is a unit if and only if $a^{-1}$ is a unit, $q-1$ is always even.  We can therefore pair each unit with its inverse, and, since $-1$ is always a unit, it follows that for $F^\times = \{a_1, \ldots, a_{q-1}\}$, letting $a_1 = 1$ and $a_2 = -1$,
\[
a_1a_2 \cdots a_{q-1} = 1 \cdot -1 \cdot (a_3a_3^{-1}) \cdots (a_{q-1}a_{q-1}^{-1}) = -1
\]

If $F = \Z/p\Z$ where $p$ is prime, then $F$ is a finite field of order $p$ and the $k$ such that $1 \leq k \leq p-1$ are precisely the units of $F$.  Therefore, by above, $(p-1)! \equiv -1 \mod p$.

\item \emph{Let $f: \Z[x] \rightarrow \C$ be the ring homomorphism defined by $f(x) = i$ and $f(n) = n$ for $n \in \Z$.  Show that $\ker f = \{g \cdot (x^2+1) \mid g \in \Z[x]\}$ and that this is the ideal generated by $x^2+1$ in $\Z[x]$.}

$f$ is defined by
\[
f\left(\sum a_k x^k\right) = \sum a_k i^k
\]

where $a_k \in \Z$.  So that if $\sum a_k i^k = 0$, $i$ is a root of the polynomial $\sum a_k x^k$.  There exist polynomials $q$ and $r$ such that for any $p \in \ker f$, $p(x) = q(x)(x^2+1) + r(x)$.  But as $p(i) = 0$, $r = 0$, and hence $p(x) = q(x)(x^2+1)$ for some polynomial $q \in \Z[x]$.  So $\ker f \subset (x^2+1)$.  Since $f(x^2+1) = 0$, $(x^2+1) \subset \ker f$, and therefore $\ker f = (x^2+1)$, the ideal generated by $x^2+1$.

\end{enumerate}
\end{document}
