\documentclass[letterpaper, 11pt]{article}
\textwidth = 6.5 in
\textheight = 9 in
\oddsidemargin = 0.0 in
\evensidemargin = 0.0 in
\topmargin = 0.0 in
\headheight = 0.0 in
\headsep = 0.0 in
\parskip = 0.2in
\parindent = 0.0in
\usepackage{amsfonts}
\usepackage{amsmath}
\usepackage{amssymb}
\usepackage{enumerate}

\newcommand{\brac}[1]{
\left\langle #1 \right\rangle
}

\newcommand{\powset}[1]{
\wp\left(#1\right)
}

\newcommand{\iso}{\cong}

\newcommand{\Aut}{\text{Aut}}
\newcommand{\Sym}{\text{Sym}}
\newcommand{\Syl}{\text{Syl}}

\newcommand{\N}{\mathbb{N}}
\newcommand{\Z}{\mathbb{Z}}
\newcommand{\Q}{\mathbb{Q}}
\newcommand{\R}{\mathbb{R}}
\newcommand{\A}{\mathbb{A}}
\newcommand{\C}{\mathbb{C}}

\newcommand{\T}{\mathcal{T}}

\title{MATH 258: Homework \#3}
\author{Jesse Farmer}
\date{26 January 2005}
\begin{document}
\maketitle
\begin{enumerate}

\item \emph{Let $R$ be a commutative ring.  Prove that $R$ is a field if and only if $0$ is a maximal ideal.}

Assume $0$ is a maximal ideal.  Then $R \iso R / \{0\}$ is a field.  Conversely, assume $R$ is a field and let $I$ be an ideal of $R$ containing a nonzero element $a$.  Then $1 = a^{-1}a \in I$, and hence $I = R$.  Therefore $0$ is a maximal ideal.

\item \emph{Let $R$ be an integral domain.  Prove that $(a) = (b)$ for some $a,b \in R$ if and only if there exists some unit $u \in R$ such that $a = ub$.}

Recall that $(a) = \{ra \mid r \in R\}$.  Assume $(a) = (b)$, then for any $r_1 \in R$ there exists an $r_2 \in R$ such that $r_1a = r_2b$.  In particular, there exist $u,v \in R$ such that $a = ub$ and $va = b$.  $u$ is a unit since 
\[
a = ub = u(va) = (uv)a
\]

and hence $uv = 1$ since $R$ is an integral domain.  Assume $a = ub$ for some unit $u$ of $R$.  $(a) \subset (b)$ since for any $ra \in (a)$, $ra = rub \in (b)$.  Conversely, $b = u^{-1}a$, so that for any $rb \in (b)$, $rb = ru^{-1}b \in (a)$, and hence $(a) = (b)$.

\item \emph{Let $R$ be the ring of all continuous function on $[0,1]$ and let $I$ be the collection of functions $f \in R$ with $f(1/3) = f(1/2) = 0$.  Prove that $I$ is an ideal of $R$ but is not a prime ideal.}

Let $g \in R$ and $f \in I$, then, for $x = 1/2, 1/3$,
\[
(gf)(x) = g(x)f(x) = g(x)0 = 0
\]

Similarly, for $g,f \in I$ and $x=1/2, 1/3$,
\[
(g+f)(x) = g(x) + f(x) = 0 + 0 = 0
\]

Therefore $I$ is an ideal.  To see that it is not a prime ideal, consider $f(x) = x - 1/3$ and $g(x) = x - 1/2$.  Then neither $f$ nor $g$ is in $I$, but $fg$ is.

\item \emph{Let $R$ be a commutative ring.  Let $I$ and $J$ ideals of $R$, and $P$ a prime ideal of $R$ that contains $IJ$.  Prove that either $I$ or $J$ is contained in $P$.}

Assume for contradiction that neither $I$ nor $J$ is contained in $P$.  Pick $a \in I$ and $b \in J$ not in $P$, then $ab \in IJ \subset P$.  But since $P$ is a prime ideal, one of $a$ or $b$ must be in $P$ -- a contradiction.  Hence $I$ or $J$ must be contained in $P$.

\item \emph{Let $R$ be a commutative ring and suppose $I$ and $J$ are two finitely generated ideals of $R$.  Prove that $IJ$ is finitely generated.}

Let $A$ and $B$ be finite subsets of $R$ and $I = (A)$ and $J = (B)$.  Let $K$ be the ideal generated by all $a_ib_i$ with $a_i \in A$ and $b_i \in B$.  Clearly $K \subset IJ$ since $A \subset (A)$ and $B \subset (B)$, so that if $x \in K$ then
\[
x = \sum_{i=1}^l r_ia_ib_i \in IJ
\]

since $r_ia_i \in (A)$ and $b_i \in B \subset (B)$.  The other inclusion is equally obvious, though more tedious.  It is sufficient to prove that $a'b' \in K$ for every $a' \in (A)$ and $b' \in (B)$ since every element in $IJ$ is a sum of $a'b'$.  So let $a' = \sum_{i=1}^\infty r_ia_i$ and $b' = \sum_{i=1}^\infty r'_ib_i$, where all but finitely many $r_i$ and $r'_i$ are zero.  Then
\[
a'b' = \sum_{i=1}^\infty c_i
\]

where $c_i = \sum_{k=0}^i r_{k,i-k} a_k b_{i-k}$ and $r_{k,i-k} = r_ir'_{i-k}$.  Then $c_i \in K$, and hence the sum of any number of $c_i$ is in $K$ since there are only finitely many nonzero $c_i$.

\item \emph{Let $\varphi: R \rightarrow S$ be a homomorphism of commutative rings.}
\begin{enumerate}
\item \emph{Prove that if $P$ is a prime ideal of $S$ then either $\varphi^{-1}(P) = R$ or $\varphi^{-1}(P)$ is a prime ideal of $R$.  Apply this to the special case where $R$ is a subring of $S$ and $\varphi$ is the inclusion homomorphism to deduce that if $P$ is a prime ideal of $S$ then $P \cap R$ is either $R$ or a prime ideal of $R$.}

The preimage of a subgroup is itself a subgroup, so take $r \in R$ and $x \in \varphi^{-1}(P)$, then $\varphi(rx) = \varphi(r)\varphi(x) \in P$ since $P$ is an ideal, and $rx \in \varphi^{-1}(P)$.  Let $ab \in \varphi^{-1}(P)$, then $\varphi(a)\varphi(b) = \varphi(ab) \in P$, which means $\varphi(a) \in P$ or $\varphi(b) \in P$, i.e., $a \in \varphi^{-1}(P)$ or $b \in \varphi^{-1}(P)$.  If $P$ is a prime ideal then $1_S \notin P$, but that means $\varphi(1_R) \notin P$, i.e., $1_R \notin \varphi(P)$.  Therefore $\varphi^{-1}(P)$ is a prime ideal.

Let $R$ be a subring of $S$ and $\varphi(r) = r$ be the inclusion map of $R$ into $S$.  Then $\varphi^{-1}(P) = P \cap R$ and hence $P \cap R$ is a prime ideal of $R$ if $P$ is a prime ideal of $S$.

\item \emph{Prove that if $M$ is a maximal ideal of $S$ and $\varphi$ is surjective then $\varphi^{-1}(M)$ is a maximal ideal of $R$.  Give an example to show that this need not be the case if $\varphi$ is not surjective.}

Let $\pi: S \rightarrow S/M$ be the natural projection from $S$ to $S/M$ and define $\psi: R \rightarrow S/M$ by $\psi = \pi \circ \varphi$.  Then $\ker \psi = \{r \in R \mid \varphi(R) \in M\} = \varphi^{-1}(M)$.  $\psi$ is surjective since it is the composition of two surjective functions and
\[
R/\varphi^{-1}(M) = R/\ker \psi \iso \psi(R) = S/M
\]

$M$ is maximal so $S/M$ is a field, and so is $R/\varphi^{-1}(M)$.  But this means $\varphi^{-1}(M)$ is maximal in $R$.

To show that surjectivity is necessary, consider the inclusion map $i: \Z \rightarrow \Q$.  Since $\Q$ is a field $0$ is a maximal ideal, but the preimage of $0$ is just $0$, which is not maximal in $\Z$.

\end{enumerate}

\item \emph{Let $R$ be a finite commutative ring with identity.  Prove that every prime ideal of $R$ is a maximal ideal.}

If $R$ is a commutative ring, recall that $R/I$ is a field if and only if $I$ is a maximal ideal.  If $P$ is a prime ideal then $R/P$ is a finite integral domain.  But every finite integral domain is a field, and therefore $P$ is also a maximal ideal.

\item \emph{Assume $R$ is a commutative ring such that for every $a \in R$ there exists an integer $n > 1$ such that $a^n = a$.  Prove that every prime ideal of $R$ is maximal.}

If $P$ is a prime ideal of $R$ then $R/P$ is an integral domain.  Let $a \in R$ with $a \neq 0$ and consider $a + P \in R/P$.  Since $a^n = a$,
\[
(a+P)(a^{n-1} + P) = (a+P)(1+P)
\]

So that $a^{n-1} + P = 1 + P$, since $R/P$ is an integral domain.  But then, as $n \geq 2$,
\[
(a + P)(a^{n-2} + P) = 1 + P
\]

So $a+P$ has an inverse in $R/P$, i.e., $R/P$ is a field and hence $P$ is a maximal ideal.

\item \emph{Let $R$ be a nonzero ring.  Show that if $e$ is an idempotent element of the center of $R$ then $Re$ and $R(1-e)$ are two-sided ideals of $R$ and that $R \iso Re \times R(1-e)$.  Show that $e$ and $1-e$ are identities for the subrings $Re$ and $R(1-e)$, respectively.}

Let $e$ be any idempotent element of the center of $R$, and let $r \in R$.  Then $$Re\cdot r = R \cdot er = R \cdot re = Rr \cdot e = Re$$ so $Re$ is a right ideal.  It is obviously a left ideal, so $Re$ is a two-sided ideal.  Moreover, for any $re \in Re$, $$e \cdot re = re \cdot e = re$$ so that $(Re, +, \cdot)$ is a ring, though not a subring as the book says (unless it happens that $e = 1$)  Note that here $e$ was an arbitrary idempotent central element, and hence this applies to any such element.

Consider $(1-e)$.  $1-e$ is in the center of $R$ since $$r(1-e) = r-re = r-er = (1-e)r$$ and is also idempotent since $$(1-e)^2 = 1 - 2\cdot e + e^2 = 1 - 2\cdot e + e= 1-e$$  Hence, from above, $R(1-e)$ is a two-sided ideal of $R$.  It also follows from above that $1-e$ is the identity of the ring $R(1-e)$.

Define $\varphi: R \rightarrow Re \times R(1-e)$ by $r \mapsto (re, r(1-e))$.  Then $\ker \varphi = 0$ since $(re, r(1-e)) = (0,0)$ implies that $re = 0$ and $r = re$, and hence $r = 0$.  Therefore $\varphi$ is injective.  To see that it is surjective, let $(re, s(1-e))$ be an arbitrary element of $Re \times R(1-e)$ and note that since $e$ is idempotent, $e(1-e) = (1-e)e = 0$.  Then $(re + s(1-e))e = re^2 + s(1-e)e = re$ and $(re + s(1-e))(1-e) = re(1-e) s(1-e)^2 = s(1-e)$.  Hence $\varphi$ is surjective.  It is obviously a homomorphism and therefore $R \iso Re \times R(1-e)$.

Note also that we could use the Chinese Remainder Theorem by showing that $R/Re \iso R(1-e)$ and $R/R(1-e) \iso Re$.  The two ideals are obviously comaximal and their intersection is trivial, so the result follows.

\item \emph{Let $R$ and $S$ be rings.  Prove that every ideal of $R \times S$ is of the form $I \times J$ for $I$ an ideal of $R$ and $J$ an ideal of $S$.}

Let $I \times J$ be an ideal of $R \times S$ and let $(x_1, y_1), (x_2, y_2)$ be arbitrary elements of $I \times J$.  Then $(x_1+x_2, y_1+y_2) \in I \times J$, which implies $x_1 + x_2 \in I$ and $y_1 + y_2 \in J$.  Similarly, for any $(r,s) \in R \times S$, $(r,s)(x,y) = (rx, sy) \in I \times J$, which implies $rx \in I$ and $sy \in J$.  Since all these points were arbitrary, $I$ and $J$ are ideals of $R$ and $S$, respectively.

\item \emph{Prove that if $R$ and $S$ are nonzero rings then $R \times S$ is never a field.}

Pick any nonzero element $r \in R$ and $s \in S$.  Then $$(r,0)(0,s) = (0,0)$$ but neither $(r,0)$ nor $(0,s)$ is $(0,0)$.  Hence $R \times S$ is never an integral domain, and therefore never a field.

\item \emph{Let $n_1, n_2, \ldots, n_k$ be integers such that $(n_i,n_j) = 1$ for $i \neq j$.}
\begin{enumerate}
\item \emph{Show that the Chinese Remainder Theorem implies that for any $a_1, \ldots, a_k \in \Z$ there is a solution $x \in \Z$ to the simultaneous congruences $x \equiv a_1 \mod n_1, \ldots, x \equiv a_k \mod n_k$.}

If $(n_i, n_j) = 1$ for $i \neq j$ then their respective ideals $n_i\Z$ and $n_j\Z$ are comaximal, i.e., $n_i\Z + n_j\Z = \Z$.  By the Chinese remainder theorem, letting $n = n_1n_2 \cdots n_k$,
\[
\Z/n\Z \equiv \Z/n_1\Z \times \cdots \times \Z/n_k\Z
\]

Hence there is a surjective homomorphism $\varphi$ from $\Z$ to $\Z/n_1\Z \times \cdots \times \Z/n_k\Z$.  Let $\overline{a_i} \in \Z / n_i \Z$.  Then there exists an $x \in \Z$ such that $\varphi(x) = (\overline{a_1}, \ldots, \overline{a_k})$, i.e., $x$ is congruent to $a_i$ modulo $n_i$ for each $1 \leq i \leq k$.  This $x$ is unique modulo $n$ since $n\Z$ is the kernel of this homomorphism.

\item \emph{Show that this solution $x$ from (a) is given by $x = a_1t_1n'_1 + \cdots + a_kt_kn'_k \mod n$ where $n'_i = n/n_i$ and $t_i$ is the inverse of $n'_i$ modulo $n_i$.}

Since $(n'_i, n_i) = 1$ there exists such a $t_i$.  Consider $n_j$ and $a_it_in'_i$ such that $i \neq j$.  Then $n_j \mid a_it_in'_i$ since $\frac{n}{n_in_j} = \prod_{h = 1}^k n_h$ where $h \neq i$ and $h \neq j$.  Consider $x - a_j$, where $x$ is as above.  Then
\[
x - a_j = \sum_{i=1, i \neq j}^k a_i t_i n'_i + a_j(t_jn'_j - 1)
\]

Hence $n_j \mid x - a_j$ since $n_j$ divides all the summands in the sum on the left, and $t_jn'_j = 1$ modulo $n_j$ by construction.  Therefore $x$ satisfies all the desired congruences.  Modulo $n$, this is the unique solution.

\item \emph{Solve the simultaneous system of congruences $$x \equiv 1 \mod 8,\, x \equiv 2 \mod 25,\, x \equiv 3 \mod 81$$ and $$y \equiv 5 \mod 8,\, y \equiv 12 \mod 25, y \equiv 47  \mod 81$$}

The smallest positive integral solutions to the above equations are $x = 4377$ and $y = 15437$.
 
\end{enumerate}

\end{enumerate}
\end{document}
