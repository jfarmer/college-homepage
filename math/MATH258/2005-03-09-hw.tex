\documentclass[10pt]{article}

\usepackage{amsfonts}
\usepackage{amsmath}
\usepackage{amssymb}
\usepackage{amsthm}
\usepackage{eucal}
\usepackage{enumerate}
\usepackage{geometry}

\geometry{letterpaper}

\textwidth = 6.5 in
\textheight = 9 in
\oddsidemargin = 0.0 in
\evensidemargin = 0.0 in
\topmargin = 0.0 in
\headheight = 0.0 in
\headsep = 0.0 in
\parskip = 0.2in
\parindent = 0.0in

\newcommand{\brac}[1]{
\left\langle #1 \right\rangle
}

\newcommand{\powset}[1]{
\wp\left(#1\right)
}

\newcommand{\Aut}{\text{Aut}}
\newcommand{\Sym}{\text{Sym}}
\newcommand{\Syl}{\text{Syl}}
\newcommand{\Hom}{\text{Hom}}
\newcommand{\End}{\text{End}}
\newcommand{\Ann}{\text{Ann}}

\newcommand{\tor}{\text{tor}}

\newcommand{\N}{\mathbb{N}}
\newcommand{\Z}{\mathbb{Z}}
\newcommand{\Q}{\mathbb{Q}}
\newcommand{\R}{\mathbb{R}}
\newcommand{\A}{\mathbb{A}}
\newcommand{\C}{\mathbb{C}}

\newcommand{\F}{\mathbb{F}}

\newcommand{\T}{\mathcal{T}}

\newcommand{\iso}{\cong}

\newtheorem{lemma}{Lemma}

\title{MATH 258: Homework \#9}
\author{Jesse Farmer}
\date{09 March 2005}
\begin{document}
\maketitle
\begin{enumerate}

\item \emph{Let $A$ be a commutative ring and define $E = \{ f \in A[x] \mid \deg f \leq n-1, f \text{monic}\}$.  Let $f_0, \ldots, f_{n-1} \in E$ with $\deg f_i = i$ for $0 \leq i \leq n-1$.  Show that $\{f_0, \ldots, f_{n-1}\}$ form a basis for $E$ over $A$, and hence that $E$ is a free $A$-module of rank $n$.}

We prove this by induction on $n$.  It is obviously true for $n=1$, so assume it for the $n-1$ case.  Without loss of generality assume that for $f \in E$ we have $\deg f = n$, since otherwise the $n-1$ case applies.  Then we can write
\[
f(x) = x^n + \sum_{i=0}^{n-1} \lambda_i f_i(x)
\]

for some $\lambda_i \in A$.  Write $f_n(x) = x^n + a_{n-1} + \cdots + a_0$.  Then $$f(x) - f_n(x) = \sum_{i=0}^{n-1} (\lambda_i - a_i)f_i(x)$$ so that $$f(x) = \sum_{i=0}^{n-1} (\lambda_i - a_i)f_i(x) + f_n(x)$$  Since the $\{f_i\}$ are monic they must be linearly independent.  This follows since if $\lambda x^k = 0$ for some $k \in \N$ then $\lambda = 0$, so this applies equally well to sums of such components.  Therefore $E$ is a free $A$-module of rank $n$.

\item \emph{Let $K$ be a field and $f \in K[x]$ with $\deg f = n > 0$.  Show that $V = K[x]/(f \cdot K[x])$ is a vector space of dimension $n$ over $K$.}

Any quotient of an $A$-module, where $A$ is a ring, field, etc., is always an $A$-module, so we must just verify that the dimension is $n$.  Consider the set $\{\bar{1}, \ldots, \bar{x}^{n-1}\}$.  Writing $\bar{x}$ as $x$ for now, this set spans since we have the following:
\[
f(x) = \sum_{x=0}^n a_i x_i \equiv 0
\]

Hence we can write $x^m = \sum_{i=0}^{m-1} b_i x_i$ for all $m \geq n$.  Any $g \in K[x] / (f \cdot K[x])$ can therefore be written as a linear combination of $\{1, \ldots, x^{n-1}\}$ since any power of $x$ greater than $n$ can be successively reduced by using the above equality until it is a polynomial of degree less than $n$.

\item \emph{Let $K$ be a field and $V, V'$ finite-dimensional vector spaces over $K$.  Let $f: V \rightarrow V'$ be a $K$-linear map.  Show that $\dim V = \dim(\ker f) + \dim f(V)$.}

From class, for any subspace $W$ of a vector space $V$, $\dim V = \dim W + \dim V/W$.  By the first isomorphism theorem it follows that $\dim V = \dim \ker f + \dim f(V)$.

\item \emph{Let $K$ be a field and $V,V'$ be finite-dimensional vector spaces over $K$.  Suppose that $\dim V = \dim V' = n$.  Show that for a $K$-linear map $f: V \rightarrow V'$ being an isomorphism, being injective, and being surjective are all equivalent.}

It is sufficient to show that a $K$-linear map is surjective if and only if it is injective.  Recall that $\dim V = \dim \ker f + \dim f(V)$.  If $f$ is surjective then $f(V) = V'$ and hence $\dim \ker f = 0$.  But then $\ker f = 0$ and hence $f$ is injective.  Similarly, if $f$ is injective then $\dim f(V) = n$, so that $f(V) \iso V'$ (two finite-dimensional vector spaces are isomorphic if and only if they have the same dimension), and hence $f$ is surjective.

\item \emph{Let $V = K^n$ where $K$ is a field and $(\lambda_1, \ldots, \lambda_n) \in K^n$ with not all $\lambda_i = 0$.  Define $$W = \left\{(a_1, \ldots, a_n) \in K^n \mid \sum_{i=1}^n \lambda_i a_i = 0\right\}$$  Show that $W$ is a subspace of $V$.  Compute $\dim W$.}

Since the $0$-vector is in $W \subset V$, it suffices to check that $x + ky \in W$ for all $x,y \in W$ and $k \in K$.  But this is fairly obvious as $x + ky = (x_1 + ky_1, \ldots, x_n + ky_n)$ and hence
\[
\sum \lambda_i(x_i + ky_i) = \sum \lambda_ix_i + k \sum \lambda_i y_i = 0
\]

Consider $K$ as a one dimensional vector space over itself and define $\varphi: V \rightarrow K$ by
\[
\varphi(a_1, \ldots, a_n) = \sum_{i=1}^n \lambda_i a_i
\]

Since not all $\lambda_i$ are zero and $K$ is a field this map is surjective. $\ker \varphi = W$, so that
\[
\dim W = \dim V - \dim K = n - 1
\]

\item \emph{Let $A$ be a commutative ring and $E$ an $A$-modules.  Let $\{e_1, \ldots, e_n\}$ generate $E$.  Show that $E$ is a free $A$-module with basis $\{e_1, \ldots, e_n\}$ if and only if for all $A$-modules $M$ and $x_1, \ldots, x_n \in M$ there exists an $A$-linear map $f: E \rightarrow M$ such that $f(e_i) = x_i$.  Is such an $f$ unique?}

Let $\{e_1, \ldots, e_n\}$ be a basis for $E$ and let $M$ be an $A$-module with $x_1, \ldots, x_n \in M$.  Define
\[
f\left(\sum a_i e_i\right) = \sum a_i x_i
\]

Clearly this satisfies the condition that $f(e_i) = x_i$.  It is linear since
\[
f(x+y) = f\left(\sum (a_i+b_i) e_i \right) = \sum (a_i +b_i) x_i = \sum a_i x_i  + \sum b_i x_i = f(x) + f(y)
\]

and similarly for the ring action on $E$ and $M$.  For the converse, let $M = A^n$ and fix the standard basis $\{x_1, \ldots, x_n\}$ of $M$, i.e., $x_i$ is $0$ in every coordinate except the $i^{th}$ where it is $1$.  Then any $A$-linear map satisfying $f(e_i) = x_i$ is obviously surjective.  To see injectivity, recall that the $\{e_i\}$ generate $E$.  Then if $f(x) = f(\lambda_1e_1 + \cdots + \lambda_ne_n) = 0$ it follows that $\lambda_ie_i = 0$ for every $\lambda_i, e_i$, and hence $x=0$.  Therefore $\ker f = 0$ and $f$ is injective.

\item \emph{Let $A$ be a nonzero commutative ring and $I \subset A$ an ideal.  Show that any two elements of $I$ are linearly dependent.  Deduce that every nonzero ideal of $A$ is a free $A$-module if and only if $A$ is a principal ideal domain.}

Let $a = y$ and $b = -x$, then for any $x,y \in I$, $ax + by = 0$ even though $a,b \neq 0$.  If $A$ is a PID then every ideal is generated by a single element, and a single element in a PID is always linearly independent.  If every nonzero ideal of $A$ is free then, from the first part, every ideal must be generated by one element (since otherwise no set of generators could be a basis).

\item
\begin{enumerate}
\item \emph{Let $A$ be an integral domain and $a \in A$.  Let $E$ be an $A$-module and define $$E(a) = \{x \in E \mid a^rx = 0, \mbox{ for some $r \geq 0$}\}$$  Show that $E(a)$ is a submodule of $E$.}

Since $0 \in E(a)$ for any $a$, $E(a) \neq \emptyset$.  Let $x,y \in E(a)$ and $b \in A$, then there exist $r,s \in \Z_+$ such that $a^sx = 0$ and $a^ry = 0$.  Therefore
\[
a^{r+s}(x + by) = a^{r+s}x + ba^{r+s}y = 0
\]

Hence $x+by \in E(a)$ and therefore $E(a)$ is a submodule.
\item \emph{Let $A$ be as above and let $E$ be a finitely generated torsion module.  Show that $\Ann(E)$ is a nonzero ideal.}

$\Ann(E)$ is an ideal since it is the kernel of the homomorphism from $A$ to the sub-modules of $E$ given by $a \mapsto aE$.  Recall that $E$ is a torsion module if $E = \tor(E) = \{x \in E \mid \exists a \in A \text{ s.t. } ax = 0\}$.  Assume $E = \brac{F}$ where $F$ is finite.  Then for every $x \in F$ there exists some nonzero $a_x$ such that $a_xx = 0$.  Let $$a = \prod_{x \in F} a_x$$  Then $a \in \Ann(E)$, since every element of $E$ can be written as an $A$-linear combination of elements of $F$.

\end{enumerate}

\item \emph{Let $A$ be a PID and $E \neq 0$ a finitely generated torsion module.  Let $I = \Ann(E)$, and, say, $I = Aa$ where $$a = \prod_{i=1}^r p_i^{m_i}$$ for $m_i > 0$ and $p_i$ a prime element of $A$ with $Ap_i \neq Ap_j$ for $i \neq j$.}
\begin{enumerate}
\item \emph{Show that the sub-module $E(p_1) + \cdots + E(p_r) = E'$ is a direct sum of the sub-modules $E(p_1), \ldots, E(p_r)$.}

First, it cannot be the case that there exist $p,q$ prime elements such that $p^nx = q^mx = 0$ for $x \neq 0$ since if this were the case then $(p^n - q^m)x = 0$, and hence $p \mid q$ or $q \mid p$, a contradiction.  Therefore, since $E$ is a torsion module, it follows that $E(p_i) = \{x \in E \mid p_i^{m_i} x = 0 \}$.  This is because if $aE = 0$ (as it does, by assumption) then if $x \in E(p_i)$, $p_i^r x = p_i^{m_i} x = 0$ for some $p_i$.  But by construction $r \leq m_i$.

This condition, viz., that no two prime distinct elements have powers which annihilate a given element of $E'$, also guarantees that $E'$ is in fact the direct sum of the $E(p_i)$.
\item \emph{Show that $E = E'$.}

This follows from the Chinese Remainder Theorem by noting that the ideals $p_i^{m_i}A$ are comaximal and that $E/(p_i^{m_i}A)E \iso E(p_i)$ (it is the kernel of the homomorphism defined by $x \mapsto a_i x$ where $a_i = \frac{a}{p_i^{m_1}}$). Therefore
\[
E \iso \frac{E}{\{0\}} \iso \frac{E}{(a)E} \iso E(p_1) \times \cdots \times E(p_n)
\]

and, from the previous part, the last expression is isomorphic to the direct sum of the $E(p_i)$.

\item \emph{Show that $E(p_i) = a_iE$, and hence that $p_i^{m_i}E(p_i) = 0$.}

This was essentially shown already, but, since every element $x \in E$ is annihilated by $a$, then it must be annihilated by some prime power dividing $a$.  From the first part there is only one such prime, $p_i$, and hence the power to do this is $m_i$.  Therefore $E(p_i)$ consists of precisely those elements that are divisible by other primes dividing $a$.  By the first part, again, this means that $E(p_i) = a_iE$, and hence $p_i^{m_i}E(p_i) = aE = 0$.
\end{enumerate}

\item \emph{Let $A$ be a commutative ring and $R = A[x]$.  Let $E$ be an $A$-module and $\alpha \in \End_A(E)$.  Show that $E$ acquires an $R$-module structure if, for $x \in E$, we define $$f(x) \cdot z = f(\alpha) \cdot z = \sum_{i=1}^r a_i \alpha^i(z)$$ where $f(x) = \sum_{i=1}^r a_i x^i$.}

Note that $\alpha^i \in \End_A(E)$ for any $i \in \N$, where this means not ``to the power of'' but rather ``$i$-fold composition.''  We take addition on $E$ as it is as an $A$-module, and multiplication as defined above.  Fix $\alpha \in \End_A(E)$.  Let $f,g \in R$ and $z \in E$, then
\[
(f + g) \cdot z = \sum (a_i+b_i) \alpha^i(z) = \sum a_i \alpha^i(z) + \sum a_i \alpha^i(z) = f \cdot z + g \cdot z
\]

Let $f,g \in R$ and $z \in E$, then
\[
f \cdot (g \cdot z) = f \cdot \left(\sum_i b_i \alpha^i(z)\right) = \sum_k a_k \alpha^k\left(\sum_i b_i \alpha^i(z)\right) = \sum_k a_k \sum_i b_i \alpha^{k+i}(z) = \sum_j c_j \alpha^j(z)
\]

where $c_j = \sum_{l} a_l b_{j-l}$.  But this last expression is equal to $(fg) \cdot z)$.  Now let $y,z \in E$ and $f \in R$, then
\[
f \cdot (y + z) = \sum a_i \alpha^i(y+z) = \sum a_i \left(\alpha^i(y) + \alpha^i(z)\right) = \sum a_i \alpha^i(y) + \sum a_i \alpha^i(z) = f \cdot y + f \cdot z
\]

That $1 \cdot z = z$ for all $z \in E$ is obvious, and hence $E$ can be extended from an $A$-module to an $A[x]$-module, given some $\alpha \in \End_A(E)$.

\item \emph{Let $A$ be a commutative ring and $E,F$ be $A$-modules.  Let $\alpha \in \End_A(E)$ and $\beta \in \End_A(F)$.  Show that an $A$-lienar map $\eta: E \rightarrow F$ is an $A[x]$-linear map if $f \circ \alpha = \beta \circ f$. }

Let $E_\alpha$ and $E_\beta$ be as defined in the previous problem. Let $f \in A[x]$ and $z \in E$.  Assume $\eta \circ \alpha = \beta \circ \eta$, then
\begin{eqnarray*}
\eta(f \cdot z) &=& \eta\left(\sum a_i \alpha^i(z)\right) \\
&=&  \sum a_i \eta\left(\alpha^i(z)\right) \\
&=& \sum a_i \beta^i\left(\eta(z)\right) \\
&=& f \cdot \eta(z)
\end{eqnarray*}

The additive properties of $\eta$ certainly still hold as a function on $E_\alpha$, so it follows that $\eta$ is an $A[x]$-linear map.
\end{enumerate}
\end{document}
