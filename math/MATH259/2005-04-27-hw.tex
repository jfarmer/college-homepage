\documentclass[10pt]{article}

\usepackage{homework}
\usepackage{enumerate}
\usepackage{geometry}

\geometry{letterpaper}

\textwidth = 6.5 in
\textheight = 9 in
\oddsidemargin = 0.0 in
\evensidemargin = 0.0 in
\topmargin = 0.0 in
\headheight = 0.0 in
\headsep = 0.0 in
\parskip = 0.2in
\parindent = 0.0in


\title{MATH 259: Homework \#4}
\author{Jesse Farmer}
\date{27 April 2005}
\begin{document}
\maketitle
\begin{enumerate}

\item \emph{Suppose $L/E$ and $E/F$ are field extensions.  Let $\alpha \in L$ be algebraic over $F$.  Prove or disprove that $[E(\alpha):E]$ divides $[F(\alpha):F]$.}

I am almost certain this is false, but don't know how to show it.

\item \emph{Find a splitting field $F/\Q$ for each of the following polynomials.  Also find the degree and a primitive element for each extension.}
\begin{enumerate}
\item \emph{$x^4 - 5x^2 + 6$}

This polynomial is reducible into $(x^2-2)(x^2-3)$, so that $\Q(\sqrt{3}, \sqrt{2})$ is a splitting field over $\Q$.  From previous assignments it follows that $[\Q(\sqrt{3}, \sqrt{2}):\Q] = 6$ and $\Q(\sqrt{3}, \sqrt{2}) = \Q(\sqrt{3} + \sqrt{2})$ so that $\sqrt{2}+\sqrt{3}$ is a primitive element.

\item \emph{$x^4-5$}

This polynomial factors over $\C$ as $(x-\sqrt[4]{5})(x+\sqrt[4]{5})(x-i\sqrt[4]{5})(x+i\sqrt[4]{5})$.  Hence the splitting field is $\Q(i, \sqrt[4]{5})$ since any strictly smaller field extension of $\Q$ will not contain one of the generators, and the two generators are linearly independent over $\Q$.  Since the minimal polynomial of $i$ over $\Q(\sqrt[4]{5})$ is still $x^2+1$, the degree of the extension is $8$.

From one of the following exercises it follows that $\Q(i, \sqrt[4]{5}) = \Q(i + \sqrt[4]{5})$, since over the splitting field we have $i - \alpha_i \neq \sqrt[4]{5} - \beta_j$ for all $i,j$ where $\alpha_i, \beta_j$ are roots of generators' respective minimal polynomials, except for the case where $\alpha_i = i$ and $\beta_j = \sqrt[4]{5}$.

\end{enumerate}

\item \emph{Find a splitting field $E$ of $x^3 - 5$ over $\F_7$, $\F_{11}$, and $\F_{13}$.  In each case determine $|E|$.}

Since any finite field extension of $\F_p$ is of the form $\F_{p^n}$ for some $n$, and these fields are constructed as the splitting field of the polynomial $x^{p^n} - x$, it suffices to find a field extension of $\F_p$ such that every root of $x^3 - 5$ is a root of $x^{p^n} - x$ where $n$ is the smallest such positive integer.  Reducing this polynomial it therefore suffices to find a smallest $n \in \N$ such that $x^{p^n - 1} - 1 = 0$ modulo $p$.  In all the cases below equality denotes congruence modulo $p$, for the respective primes in consideration.  Let $f(x) = x^3 - 5$.

For $\F_7$ there are no roots contained in the field itself since every cube modulo $7$ is congruent to either $1$ or $6$.  Let $\alpha$ be a root of $f$ so that $\alpha^3 = 5$.  Then for $n=2$, $\alpha^{48} = 5^{16} = 2$ modulo $7$.  However, for $n=3$, this becomes $\alpha^{342} = 5^{114} = 1$ modulo $7$.  Hence the splitting field of $f$ over $\F_7$ is $\F_{343}$.

For $\F_{11}$ there is a root in the field, namely $\alpha = 3$, but no other roots.  Consider $n=2$.  Then if $\alpha$ is a root of $f$, $\alpha^{120} = 5^{40} = 1$ modulo $11$.  Hence the splitting field of $f$ over $\F_{11}$ is $\F_{121}$.

For $\F_{13}$ there are two roots in the field, namely, $\alpha=7$ and $\alpha=11$, but no other roots.  Again, consider $n=2$.  Then if $\alpha$ is a root of $f$, $\alpha^{168} = 5^{56} = 1$ modulo $13$.  Hence the splitting field of $f$ over $\F_{13}$ is $\F_{169}$.

\item \emph{Let $F$ be a field and $f,g \in F[x]$ with $\deg f, \deg g > 0$.  Show that $\gcd(g,f) \neq 1$ if and only if $f$ and $g$ have a common root $\alpha$, with $\alpha \in E$ for some field extension $E/F$.}

If $f$ and $g$ have a common root $\alpha$ in some field extension $E/F$ then the minimal polynomial of $\alpha$, $h$, has $\deg h > 0$ and $h \mid g$ and $h \mid f$.  Hence $h$ is a common divisor, and $\gcd(g,f) \neq 1$.

Conversely, if $\gcd(f,g) \neq 1$ then there exists a polynomial $h \in F[x]$ with $\deg h > 0$ which divides both $f$ and $g$.  Let $E$ be the splitting field of $h$ over $F$.  Then for any root $\alpha \in E$ of $h$, $f(\alpha) = g(\alpha) = h(\alpha) = 0$, i.e., $f$ and $g$ share a common root.

\item \emph{Let $F(\alpha)/F$ be a simple extension with $\alpha$ separable over $F$.  Suppose $\ch F = p > 0$.  Show that $F(\alpha) = F(\alpha^p)$.}

Since $\alpha$ is separable, $F(\alpha)/F$ is a separable extension.  Consider $F(\alpha)/F(\alpha^p)$.  Let $f$ be the minimal polynomial of $\alpha$ over $F(\alpha^p)$.  Then $\alpha$ is a root of $g(x) = x^p-\alpha^p$.  Since $\ch F = p$, $g(x) = (x - \alpha)^p$ over $F(\alpha)$.  But then $f \mid g$, and $f$ has no multiple roots since $F(\alpha)/F(\alpha^p)$ is also a separable extension.  Hence $f(x) = x - \alpha \in F(\alpha^p)[x]$ and $\alpha \in F(\alpha^p)$.

\item{Let $F$ be a field and $x^p - a$, $x^p - b$, $p$ a prime, be two irreducible polynomials in $F[x]$.  Suppose that $\ch F \neq p$.  Let $E = F(\alpha, \beta)$ with $\alpha^p = a$, $\beta^p = b$, and $[E:F] = p^2$.  Show that $\alpha + \beta$ is a primitive element of $E/F$.}

Let $F_1 = F(\alpha)$.  Then $F_1(\beta) = F_1(\alpha+\beta) = E$ and $p^2 = [E:F] = [E:F_1][F_1:E]$ which implies $[E:F_1] = p$.  Assume for contradiction that $F(\alpha + \beta) \neq E$.  Then $\alpha + \beta \notin F$ since that would imply $\alpha + \beta \in F_1$ and hence $\beta \in F_1$, contradicting that $[E:F_1] = p$.

Then $[E:F(\alpha,\beta)] > 1$ so that $p^2 = [E:F(\alpha,\beta)][F(\alpha,\beta):F]$ implies that $[F(\alpha,\beta):F] = p$.  Let $f$ be the minimal polynomial of $\alpha + \beta$ over $F$.  Since $[F_1(\alpha+\beta):F_1] = p$, it is also the minimal polynomial of $\alpha + \beta$ over $F_1$.  Define $g(x) = (x-\alpha)^p - b$.  Then $g(\alpha+\beta) = 0$ and $f \mid g$.  Write $f = gh$ for some $h \in F_1[x]$.  Since $\deg g = \deg f$, $\deg h = 1$, but as $g$ is monic it must be that $h = 1$.  Hence $g = f$, which means that, in fact, $g \in F[x]$.  But the coefficient of $x^{p-1}$ in $g$ is $-p\alpha$ by calculation.  Since $\ch F \neq p$, it follows that $\alpha \in F$, contradicting the fact that $x^p - a$ is irreducible.

Therefore $F(\alpha+\beta) = F(\alpha, \beta)$.

\item \emph{Let $E/F$ be a field extension and $\alpha, \beta \in E$ be algebraic over $F$ with minimal polynomials $f,g$ of degree $m$ and $n$, respectively.  Write $f = \prod_{i=1}^m(x - \alpha_i)$ and $g = \prod_{i=1}^n(x - \beta_i)$ with $\alpha_1 = \alpha$ and $\beta_1 = \beta$.  If $\alpha - \alpha_i \neq \beta - \beta_j$ for all $i,j$ show that $F(\alpha, \beta) = F(\alpha+\beta)$.}

From the proof that separable extensions are simple, we know that if $c \in F$ satisfies $g(c(\alpha-\alpha_i) + \beta)$ for all $2 \leq i \leq m$ then $F(\alpha, \beta) = F(\alpha + \beta)$.  But for $c = 1$ this becomes $\alpha-\alpha_i + \beta \neq \beta_j$ for any $j$ with $1 \leq j \leq n$, and $i$ with $1 \leq i \leq m$.

However, the hypothesis in the statement of the exercise seems to be backwords.  Namely, we want $\alpha - \alpha_i \neq \beta_j - \beta$, rather than the condition given.

\item \emph{Deduce from the previous exercise that $\Q(\sqrt[3]{p}, \sqrt[3]{q}) = \Q(\sqrt[3]{p} + \sqrt[3]{q})$ where $p,q$ are prime.}

The minimal polynomial of $\sqrt[3]{p}$ is $x^3 - p$, which splits into
\[
(x - \sqrt[3]{p})\left(\frac{\sqrt[3]{p} + i\sqrt{3}\sqrt[3]{p}}{2}\right)\left(\frac{\sqrt[3]{p} - i\sqrt{3}\sqrt[3]{p}}{2}\right)
\]

For $p,q$ distinct primes, these factors satisfy the conditions of the previous exercise (as no cube root of two distinct primes will ever be rational multiples of one another) and therefore $\Q(\sqrt[3]{p}, \sqrt[3]{q}) = \Q(\sqrt[3]{p} + \sqrt[3]{q})$.

\item \emph{Let $F$ be a field and $E/F$ the splitting field of $x^n-1$.  Define $\mu_n = \set{\zeta \in E \mid \zeta^n = 1}$.  If $\ch F = 0$ or $\ch F = p \not| n$ show that $\mu_n$ is a cyclic subgroup of $E^\ast$ of order $n$.}

In general $\mu_n$ is a cyclic group (of some order) since if $\alpha, \beta \in \mu_n$ then $(\alpha\beta)^n = \alpha^n\beta^n = 1$, and it is known that any subgroup of the multiplicative group of a field is cyclic.  Consider the polynomial $x^n - 1$.  Its derivative is $nx^{-1}$, which has a zero only at $x=0$ for fields of characteristic $0$ or fields of prime characteristic $p$ which do not divide $n$.  Hence, for such fields, every root of $x^n-1$ is distinct and the splitting field of $x^n-1$ has order $n$.  That is, $\mu_n$ is a cyclic subgroup of $E^\ast$ of order $n$.

\end{enumerate}
\end{document}
