\documentclass[10pt]{article}

\usepackage{amsfonts}
\usepackage{amsmath}
\usepackage{amssymb}
\usepackage{amsthm}
\usepackage{eucal}
\usepackage{enumerate}
\usepackage{geometry}

\geometry{letterpaper}

\textwidth = 6.5 in
\textheight = 9 in
\oddsidemargin = 0.0 in
\evensidemargin = 0.0 in
\topmargin = 0.0 in
\headheight = 0.0 in
\headsep = 0.0 in
\parskip = 0.2in
\parindent = 0.0in

\newcommand{\brac}[1]{
\left\langle #1 \right\rangle
}

\newcommand{\powset}[1]{
\wp\left(#1\right)
}

\newcommand{\Aut}{\text{Aut}}
\newcommand{\Sym}{\text{Sym}}
\newcommand{\Syl}{\text{Syl}}
\newcommand{\Hom}{\text{Hom}}
\newcommand{\End}{\text{End}}
\newcommand{\Ann}{\text{Ann}}

\newcommand{\tor}{\text{tor}}

\newcommand{\N}{\mathbb{N}}
\newcommand{\Z}{\mathbb{Z}}
\newcommand{\Q}{\mathbb{Q}}
\newcommand{\R}{\mathbb{R}}
\newcommand{\A}{\mathbb{A}}
\newcommand{\C}{\mathbb{C}}

\newcommand{\F}{\mathbb{F}}

\newcommand{\T}{\mathcal{T}}

\newcommand{\iso}{\cong}

\newtheorem{lemma}{Lemma}

\title{MATH 259: Homework \#1}
\author{Jesse Farmer}
\date{04 April 2005}
\begin{document}
\maketitle
\begin{enumerate}

\item \emph{Show that $p(x) = x^3 + 9x + 6$ is irreducible in $\Q[x]$.  Let $\theta$ be a root of $p(x)$.  Find the inverse of $1+\theta$ in $\Q(\theta)$.}

$p$ is irreducible by Eisenstein's theorem.  Note that $1+\theta$ is a root of
\[
p(1-x) = (1-x)^3 + 9(1-x) + 6 = x^3 - 3x^2 + 12x - 4
\]

Therefore, by Example $5$ on pp. 516, it follows that
\[
(1+\theta)^{-1} = \frac{(\theta+1)^2 - 3(\theta+1) + 12}{4} = \frac{\theta^2 + \theta - 10}{4}
\]

\item \emph{Show that $x^3 - 2x - 2$ is irreducible over $\Q$ and let $\theta$ be a root.  Compute $(1+\theta)(1+\theta+\theta^2)$ and $\frac{1+\theta}{1+\theta+\theta^2}$ in $\Q(\theta)$.}

Let $p(x) = x^3 - 2x - 2$.  Since $\deg p = 3$, it is reducible if and only if it has a rational root, but neither $\pm 2, \pm 1$ are roots.  So, by the rational roots theorem $p$ has no roots and is therefore irreducible.  Any root $\theta$ of $p$ satisfies $\theta^3 = 2\theta + 2$, so
\[
(1+\theta)(1+\theta+\theta^2) = 1 + 2\theta + 2\theta^2 + \theta^3 = 3 + 4\theta + 2\theta^2
\]

\item \emph{Prove directly that the map $a + b\sqrt{2} \mapsto a - b\sqrt{2}$ is an isomorphism of $\Q(\sqrt{2})$.}

This map is a bijection since it is its own left and right inverse.  Denote this map by $\varphi$, then
\[
\varphi(a+b + (c+d)\sqrt{2}) = a+b - (c+d)\sqrt{2} = a - c\sqrt{2} + b-d\sqrt{2} = \varphi(a+b\sqrt{2}) + \varphi(c+d\sqrt{2})
\]
and
\[
\varphi(ac + 2bd + (ad+bc)\sqrt{2}) = ac +2bc - (ad+bc)\sqrt{2} = (a-b\sqrt{2})(c-d\sqrt{2}) = \varphi(a+b\sqrt{2})\varphi(c+d\sqrt{2})
\]

Therefore $\varphi$ is, indeed, an isomorphism.

\item \emph{Show that if $\alpha$ is a root of $a_nx^n + \cdots + a_1x + a_0$ then $a_n\alpha$ is a root of the monic polynomial $x^n + a_{n-1}x^{n-1} + a_na_{n-2}x^{n-2} + \cdots + a_n^{n-2}a_1x + a_n^{n-1}a_0$.}

Let the latter polynomial be denoted by $f(x)$ and the former by $g(x)$ then
\[
f(a_n\alpha) = (a_n \alpha)^n + \sum_{i=0}^{n-1} a_n^{n-1-i} a_i (a_n \alpha)^i = a_n^n \alpha^n + \sum_{i=0}^{n-1} a_n^{n-1}a_i\alpha^i = a_n^{n-1}g(\alpha) = 0
\]

\item \emph{Suppose the degree of the extension $K/F$ is a prime $p$.  Show that any subfield $E$ of $K$ containing $F$ is either $K$ or $F$.}

In this case, $F \subset E \subset K$.  Let $[K:F] = p$ be prime, and $[K:E] = n$, $[E:F] = m$.  Then we know that $p = mn$, which implies that either $m=1$ or $n=1$, i.e., either $K = E$ or $F = E$.

\item \emph{Prove that if $[F(\alpha) : F]$ is odd then $F(\alpha) = F(\alpha^2)$.}

$\alpha^2 \in F(\alpha)$, so $F(\alpha^2)$ is a field extension of $F(\alpha)$.  Hence we have $F \subset F(\alpha^2) \subset F(\alpha)$.  Then if $[F:F(\alpha)]$ is odd both $[F(\alpha):F(\alpha^2)]$ and $[F(\alpha^2):F]$ must be odd.  Assume for contradiction that $\alpha \notin F(\alpha^2)$.  Then the polynomial $x^2 - \alpha^2$ is irreducible over $F(\alpha^2)$, and is therefore the minimal polynomial for $\alpha$ over $F(\alpha)^2$.  However, this implies that $[F(\alpha):F(\alpha^2)] = 2$, contradicting the fact that $[F(\alpha):F]$ is odd.  Hence $\alpha \in F(\alpha^2)$, and $F(\alpha^2) = F(\alpha)$.

\item \emph{Determine the degree of $\Q(\sqrt{3 + 2\sqrt{2}})$ over $\Q$.}

Note that $(1+\sqrt{2})^2 = 3 + 2\sqrt{2}$, so that $\sqrt{3 + 2\sqrt{2}} = 1 + \sqrt{2}$.  This has a minimal polynomial of degree two, viz., $x^2 - 2x - 1$, so $\Q(\sqrt{3 + 2\sqrt{2}})$ has degree $2$ over $\Q$.

\item \emph{Suppose $F = \Q(\alpha_1, \ldots, \alpha_n)$ where $\alpha_i^2 \in \Q$ for $1 \leq i \leq n$.  Prove that $\sqrt[3]{2} \notin F$.}

If $\sqrt[3]{2} \in F$ then $\Q(\sqrt[3]{2})$ would be a subfield of $F$.  However $[F:\Q]$ is a power of $2$, since, if any minimal polynomial over $\Q$ becomes reduced in once of the larger field extensions, its degree would simply become $1$.  This is a contradiction since $[\Q(\sqrt[3]{2}):\Q] = 3$ and certainly cannot divide any power of $2$.

\item \emph{Let $f$ be an irreducible polynomial of degree $n$ over a field $F$ and $g \in F[x]$.  Prove that every irreducible factor of the composite polynomial $f \circ g$ has degree divisible by $n$.}

Let $\alpha$ be any root of $f \circ g$ (in its splitting field, say).  It is sufficient to prove that $[F(\alpha):F]$ is divisible by $n$ for all such $\alpha$.  Since $f$ is irreducible and $g(\alpha)$ is a root of $f$, it follows that $[F((g(\alpha)):F] = n$.  But $F(g(a))$ is a subfield of $F(\alpha)$, and hence $n \mid [F(\alpha):F]$ since
\[
[F(\alpha):F] = [F(\alpha):F(g(\alpha))]\cdot[F(g(\alpha)):F]
\]

\item \emph{Let $E/F$ be a field extension and $\alpha_i \in E$ algebraic over $F$ for $1 \leq i \leq n$.  Show that $$[F(\alpha_1, \ldots, \alpha_n) : F] \leq \prod_{i=1}^n m_i$$ where $m_i$ is the degree of the minimal polynomial of $\alpha_i$ over $F$.}

We can write
\[
[F(\alpha_1, \ldots, \alpha_n) : F] = \prod_{i=1}^n [F(\alpha_1, \ldots, \alpha_i):F(\alpha_1, \ldots, \alpha_{i-1})]
\]

where $F(\alpha_0)$ is defined as $F$.  Each $[F(\alpha_1, \ldots, \alpha_i):F(\alpha_1, \ldots, \alpha_{i-1})]$ is bounded by $m_i$, since $\alpha_i$ has a minimal polynomial of degree $m_i$ over $F$, and so that same polynomial has $\alpha_i$ as a root when its coefficients are in $F(\alpha_1, \ldots, \alpha_{i-1})$.  It could, of course, be the case that what was the minimal polynomial is now reducible over $F(\alpha_1, \ldots, \alpha_{i-1})$, which is why strict equality might not hold.  The inequality follows immediately.

\item \emph{Suppose $[E:F] = p$, where $p$ is a prime.  Show that for every $\alpha \in E \setminus F$, $E = F(\alpha)$.}

Let $\alpha \in E \setminus F$, then $F \subset F(\alpha) \subset E$.  Note that it is sufficient to show that $[E:F(\alpha)] = 1$, since any finite extension is algebraic and hence is an extension by an element with degree $1$.  If $[E:F]$ is prime, then $[E:F(\alpha)][F(\alpha):F]$ is also prime.  Since $\alpha \notin F$, $F$ is strictly contained in $F$, and hence $[E:F(\alpha)] = 1$, so that $E = F(\alpha)$.

\item \emph{Let $[E:F] = n$ and $\alpha \in E$.  Show that $\deg f \mid n$ where $f$ is the minimal polynomial of $\alpha$ over $F$.}

Note that $F(\alpha) \subset E$ since $\alpha \in E$.  Then  $[F(\alpha) : F] = \deg f$, and therefore $n = m \cdot \deg f$, where $m = [E:F(\alpha)]$.

\item \emph{Let $E = F(\alpha)$ where $\alpha$ is algebraic over $F$.  Let $F'$ be a subfield of $E$ contained in $F$, and let $g$ be the minimal polynomial of $\alpha$ over $F'$, and $f$ the minimal polynomial of $\alpha$ over $F$.  Let $[E:F] = n$ and $m = \deg g$.  Show that $f = gh$ for some $h \in F'[x]$ and $n$ is a multiple of $m$.}

We can treat $f$ as a polynomial in $F'[x]$ since $F \subset F'$.  It follows then that there exists some $h$ such that $f = gh + r$ where $\deg r < \deg g$.  Since $f(\alpha) = g(\alpha) = 0$, it follows that $r(\alpha) = 0$.  But as $g$ is the minimal polynomial of $\alpha$ over $F'$, $r$ must be identically $0$, and hence $f = gh$.  That $n$ is a multiple of $m$ follows from the equality $n = [E:F] = [E:F'][F':E] = m[F':E]$.

\item \emph{Let $E/F$ be a field extension and $\alpha \in E$ be algebraic over $F$.  Let $F'$ be a subfield of $E$ contained in $F$.  Let $f \in F[x]$ be an irreducible polynomial such that $f(\alpha) = 0$ and $\deg f = n$.  Show that $f$ is irreducible in $F'[x]$ if and only if $[F'(\alpha) : F'] = n$.}

Let $g$ be the minimal polynomial of $\alpha$ over $F'$.  Then $[F'(\alpha) : F'] =  \deg g$.  If $f$ is irreducible over $F'$, then $f = c \cdot g$ for some constant $c \in F'^\ast$, so that $\deg f = \deg g = n$.  Conversely, if $[F'(\alpha) : F'] = n$, then $\deg g = n$.  If $f$ were reducible over $F'$ then there would exist an irreducible polynomial of degree less than $n$ with $\alpha$ as a root, contradicting the minimality of $g$.

\item \emph{Let $F, F', E$ be subfields of a field $L$.  Suppose $[E:F] = n$ and $F \subset F'$ with $[F': F] = m$.  Suppose $m,n$ are relatively prime.  Let $EF'=F'E$ denote the subfield of $L$ generated by $F' \cup E$.  Show that $[F'E: F] = mn$.  Compute $[F'E: F']$.}

Write $F' = F(\beta_1, \ldots, \beta_s)$ and $E=F(\alpha_1, \ldots, \alpha_r)$.  Then $$EF' = F(\alpha_1, \ldots, \alpha_r, \beta_1, \ldots, \beta_s)$$.  In particular one sees that,
\[
[F(\alpha_1, \ldots, \alpha_r, \beta_1, \ldots, \beta_{s-i}):F(\alpha_1, \ldots, \alpha_r, \beta_1, \ldots, \beta_{s-i-1})] \leq [F(\beta_1, \ldots, \beta_{s-i}) : F(\beta_1, \ldots, \beta_{s-i-1})]
\]

This is because any polynomial irreducible over $F(\alpha_1, \ldots, \alpha_r, \beta_1, \ldots, \beta_{s-i-1})]$ is certainly irreducible over $F(\beta_1, \ldots, \beta_{s-i-1})$.  However, as previously, the product of all the elements on the left-hand side of the inequality is equal to $[EF':E]$, while the product of those on the right-hand side is equal to $[F':F] = m$.  Hence $[EF':F] = [EF':E][E:F] \leq mn$.  Writing 
\[
[EF':F] = [EF':E][E:F] = [EF':F'][F':F]
\]

shows that both $m$ and $n$ divide $[EF':F]$.  Since $m$ and $n$ are relatively prime it follows that $mn \mid [EF':F]$, and therefore that $[EF:F] = mn$.  It follows that $[EF':F'] = n$ and $[EF':E] = m$.

\item \emph{In the above, let $E = F(\alpha)$ and $[E:F] = n$.  Let $f \in F[x]$ be the minimal polynomial of $\alpha$ over $F$.  Show that $f$ is irreducible over $F'$.}

In this case, $EF' = F'(\alpha)$, and from a previous problem we know $f$ is irreducible over $F'$ if and only if $[F'(\alpha):F] = n$.  But as $[F':F]$ is relatively prime to $n$ by hypothesis, exactly that follows from the previous problem.  Hence $f$ is irreducible over $F'$.

\item \emph{Let $E = F(\alpha)$ be a subfield of $L$ and $F'$ a subfield of $L$ such that $F \subset F'$ with $[F':F] = 2$.  Let $f \in F[x]$ be an irreducible sextic with $f(\alpha) = 0$.  Show that $f$ is irreducible over $F'$ or $f$ is a product of irreducible cubics in $F'[x]$.}

\end{enumerate}
\end{document}
