\documentclass[10pt]{article}

\usepackage{homework}
\usepackage{enumerate}
\usepackage{geometry}

\geometry{letterpaper}

\textwidth = 6.5 in
\textheight = 9 in
\oddsidemargin = 0.0 in
\evensidemargin = 0.0 in
\topmargin = 0.0 in
\headheight = 0.0 in
\headsep = 0.0 in
\parskip = 0.2in
\parindent = 0.0in


\title{MATH 259: Homework \#7}
\author{Jesse Farmer}
\date{18 May 2005}
\begin{document}
\maketitle
\begin{enumerate}

\item \emph{Let $K/F$ be a Galois extension with $\Gal(K/F) \iso \Z/2\Z \times \Z/12\Z$.  How many intermediate fields $L$ are there such that:}
\begin{enumerate}
\item \emph{$[L:F] = 4$}

By FTG it is sufficient and necessary to find subgroups $H$ of $\Gal(K/F)$ with index $4$, i.e., subgroups of order $6$.  Let $\Z/2\Z \times \Z/12\Z = \brac{x, y | x^2 = y^{12} = 1, xyx^{-1}y^{-1} = 1}$.  The only subgroups of order $6$ are therefore $\brac{y^6}$ and $\brac{xy^4}$.  Therefore there exist two intermediate fields $L$ such that $[L:F] = 4$.
\item \emph{$[L:F] = 9$}

As $9 \nmid 24$, there are no such fields.

\item \emph{$\Gal(K/L) \iso \Z/4\Z$}

Using the same presentation as in (a), it is easy to see that the only possible subgroups of order $4$ are $\brac{y^3}, \brac{x, y^6}$, and $\brac{xy^3}$.  Only the first and last are cyclic, and therefore are isomorphic to $\Z/4\Z$.  Hence there are two such intermediate fields.

\end{enumerate}

\item
\begin{enumerate}
\item \emph{Let $F$ be a field with $\ch F = p > 0$, $f = x^p - x + a \in F[x]$.  Let $E/F$ be a field extension with $\alpha \in E$ where $f(\alpha) = 0$.  Show that $\alpha, \alpha+1, \ldots, \alpha+(p-1)$ are all roots of $f$.}

Let $1 \leq n \leq p -1$ and assume $f(\alpha) = 0$.  Then
\[
f(\alpha + n) = (\alpha + n)^p - \alpha - n + a = \alpha^p + n^p - \alpha - n + a = n^p - n = 0
\]

Since $p \mid n^p - n$ by Fermat's Little Theorem.

\item \emph{With $F$ and $f$ as above, show that $f$ is irreducible over $F$ if and only if $f$ has no roots in $F$.}

By the first part if $f$ has any root in $F$ then $f$ has all of its roots in $F$, and therefore $f$ is reducbile.  Indeed, it splits completely into linear factors over $F$.

Assume $f$ has no roots in $F$.  Let $E/F$ be the splitting field of $f$ and suppose for contradiction that $f = gh$, $g,h \in F[x]$ with $\deg g = r < p$, i.e., $f$ is reducible.  Hence $gh = \prod_{i=0}^{p-1} (x - \alpha - i)$ by the first part.  Write $g(x) = x + cx^{r-1} + \cdots c_0$.  Calculating $c$ gives $c = r\alpha + b$ for some $b \in F$, and hence $\alpha \in F$, a contradiction.

\item \emph{With $F$ and $f$ as above, suppose $f$ has no roots in $F$.  Let $E/F$ be the splitting field of $f$ over $F$.  Show that $E=F(\alpha)$ for all $\alpha \in E$ such that $f(\alpha) = 0$ and $E/F$ is cyclic of degree $p$.   Exhibit the elements of $\Gal(E/F)$.}

Let $\alpha \in E$ such that $f(\alpha) = 0$.  Then from the first part $\alpha+1, \cdots, \alpha+(p-1)$ are also roots, each of which is clearly in $E$.  Since $\deg f = p$ these are all the roots and therefore $E = F(\alpha)$.

Let $E = F(\alpha)$.  Then $p = [E:F] = \abs{\Gal(E/F)}$.  Define $\sigma \in \Gal(E/F)$ by $\sigma(\alpha) = \alpha+1$.  Then $|\sigma| = p$ and hence $\Gal(E/F) = \brac{\sigma}$, i.e., $\Gal(E/F)$ is cyclic of degree $p$.

\end{enumerate}

\item \emph{Determine the Galois group of $x^4 + x^2 + 4$ over $\Q$.}



\item
\begin{enumerate}
\item \emph{Suppose $H \unlhd S_4$ and $S_3 \leq H$.  Show that $H = S^4$.}

Let $H$ be such that $S_3 \leq H \unlhd S_4$.  Since $H$ is normal it contains all conjugates of transpositions in $S_3$.  But it is easy to see that this requires $H$ to contain all transpositions in $S_4$ since for $(a b) \in S_3$ and $(c d) \in S_4$, distinct transpositions, conjugating $(a b)$ by $(a c b d)$ gives $(c d)$.  Hence $H = S_4$.

\item \emph{Let $E/F$ be a Galois extension with $G = \Gal(E/F) \iso S_4$ via $\eta: S_4 \rightarrow G$.  Let $H = \eta(S_3)$.  Let $F'$ be the fixed field of $H$.  Compute $[F':F]$.  Show that for any intermediate field $L$ either $L=F'$ or $L=F$.}

First, $[F':F] = |S_4:S_3| = 4$.  Assume $F \subsetneq L \subsetneq F'$, since otherwise we are done.  By the Galois correspondence $H$, the elements in $\Gal(E/F)$ fixing $L$ is isomorphic to a proper subgroup of $S_4$ properly containing $S_3$.  But if this is so then $|H| = 12$ and $|\Gal(E/F):H| = 2$.  Hence $H$ is normal, and by the first part $H \iso S_4$.  It then follows that $L = F$.

\end{enumerate}


\item \emph{Given any monic polynomial $f(x) \in \Z[x]$ of degree at least one show that there are infinitely many distinct prime divisors of the integers $f(1), f(2), \ldots, f(n), \ldots$.}

Assume for contradiction that there exist a finite number of primes $p_1, \ldots, p_k$ which divide each of $f(1), \ldots, f(n)$.  Let $N \in \Z$ such that $f(N) = a \neq 0$ and let $\beta = a p_1 p_2 \cdots p_k$.  Define
\[
g(x) = a^{-1}f\left(N + \beta x\right)
\]

Every term in $f\left(N + \beta x\right)$ has a coefficient containing $\beta$ except the constant term, which is exactly
\[
N^n + a_{n-1}N^{n-1} + \cdots + a_0 = f(N) = a
\]

Hence each term is divisible by $a$ and $g(x) \in \Z[x]$.  Furthermore, since each term containing $\beta$ is congruent to $0 \mod p_1 \cdots p_k$, and the constant term of $g$ is just $1$, it must be the case that $$g(n) \equiv 1 \mod p_1 p_2 \cdots p_k$$ for $n \in \Z_+$.

If $g(b) = 1$ for all $n \in \Z_+$ then $f$ would be the constant polynomial, contradicting the hypothesis that $\deg f \geq 1$.  So there exists an $m$ with $g(m) \neq 1$.  Since $g(m) \equiv 1 \mod p_1 \cdots p_k$, $g(m) \equiv 1 \mod p_i$ for all $1 \leq i \leq k$.  In particular this means that none of the $p_i$ divide $g(m)$.  Since $g(m) \neq 1$ it must be divisible by a prime number not among the $p_i$ and therrefore $f\left(N + \beta m\right)$ has a prime factor not among the $p_i$, also.

\item \emph{Let $p$ be an odd prime not dividing $m$ and let $\Phi_m(x)$ be the $m^{th}$ cyclotomic polynomial.  Suppose $a \in \Z$ satisfies $\Phi_m(a) \equiv 0 \mod p$.  Prove that $a$ is relatively prime to $p$ and that the order of $a$ in $(\Z/p\Z)^\times$ is precisely $m$.}

Write
\[
x^m - 1 = \prod_{d \mid m} \Phi_d(x) = \Phi_m(x) \prod_{\substack{d \mid m \\ d < m}} \Phi_d(x)
\]


If $\Phi_m(a) = 0$ then $a^m \equiv 1 \mod p$.  The only possible divisors of $p$ are $1$ and $p$, so if $\gcd(a,p) \neq 1$ then $\gcd(a,p) = p$.  But then $a^m \equiv 0 \mod p$ so that $1 \equiv 0 \mod p$, which is absurd.

Assume for contradiction that the order of $a$ is less than $m$, so that there exists a $d < m$ with $a^d \equiv 1 \mod p$.  Then $a$ would be a root of $\Phi_m(x)$ and $\Phi_d(x)$, which would mean that $x^m - 1$ is not separable -- a contradiction since $p \nmid m$.

\item \emph{Let $a \in \Z$.  Show that $p$ is an odd prime dividing $\Phi_m(a)$ then either $p \mid m$ or $p \equiv 1 \mod m$.}
\item \emph{Prove there are infinitely many primes $p$ with $p \equiv 1 \mod m$.}

There are infinitely many odd primes, and only finitely many primes dividing $m$.  From the previous exercises we know that there are infinitely many primes dividing $\Phi_m(a)$ for $a \in \Z_+$, and since there are only finitely many primes dividing $m$, there are infinitely many primes not dividing $m$ which do divide $\Phi_m(a)$.  Hence there are infinitely many such that primes $p$ such that $p \equiv 1 \mod m$.

\item \emph{Deduce that if $G$ is any finite abelian group then there exists a Galois extension $E/\Q$ such that $\Gal(E/\Q) \iso G$.}

By the fundamental theorem of finite abelian groups 
\[
G \iso \prod_{i=1}^r \Z/n_i\Z
\]

for some integers $n_1, \ldots, n_r$.  By the previous exercise there exist $r$ distinct primes such that $p_i \equiv 1 \mod n_i$.  Let $L/\Q$ be the splitting field of $x^m - 1$ for $m = p_1p_2 \cdots p_k$.  Then
\[
\Gal(L/\Q) \iso (\Z/m\Z)^\times \iso \prod_{i=1}^r (\Z/p_i\Z)^\times
\]

Thus $\Gal(L/\Q) \iso \prod_{i=1}^r G_i$ where $G_i$ is a cyclic group of order $p_i - 1$.  For each $n_i \mid (p_i - 1)$ there exists $H_i \unlhd G_i$ with $|H_i| = \frac{p_i - 1}{n_i}$.  Hence $G_i / H_i$ is cyclic of order $n_i$.  Recall from group theory that for groups $\set{G_i}$ and normal subgroups $\set{H_i \leq G_i}$ that
\[
\frac{\prod_{i=1}^k G_i}{\prod_{i=1}^k H_i} \iso \prod_{i=1}^k G_i/H_i
\]

by consideration of the first isomorphism theorem.  Let $H = \prod_{i=1}^r H_i$ and $E$ the fixed field of $H$. Then by the above
\[
\Gal(E/\Q) \iso \frac{\prod_{i=1}^r G_i}{\prod_{i=1}^r H_i} \iso \prod_{i=1}^r G_i/H_i \iso \prod_{i=1}^r \Z/n_i\Z \iso G
\]
\end{enumerate}
\end{document}
