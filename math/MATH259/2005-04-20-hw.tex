\documentclass[10pt]{article}

\usepackage{homework}
\usepackage{enumerate}
\usepackage{geometry}

\geometry{letterpaper}

\textwidth = 6.5 in
\textheight = 9 in
\oddsidemargin = 0.0 in
\evensidemargin = 0.0 in
\topmargin = 0.0 in
\headheight = 0.0 in
\headsep = 0.0 in
\parskip = 0.2in
\parindent = 0.0in


\title{MATH 259: Homework \#3}
\author{Jesse Farmer}
\date{20 April 2005}
\begin{document}
\maketitle
\begin{enumerate}

\item \emph{Let $E/F$ be a field extension with $\alpha, \beta \in E$ such that $[F(\alpha):F] = m$ and $[F(\beta):F] = p$, where $p$ is prime and $1 \leq p < m$.}
\begin{enumerate}
\item \emph{Show that $[F(\alpha, \beta):F] = mp$.}

Since $F(\alpha)$ and $F(\beta)$ are subfields of $F(\alpha, \beta)$,
\[
kp = [F(\alpha, \beta):F(\beta)][F(\beta):F] = [F(\alpha, \beta):F] = [F(\alpha, \beta):F(\alpha)][F(\alpha):F] = jm
\]

for some $j,k \in \N$.  Since $p < m$, $p$ cannot divide $m$ it must be that so $p \mid j$.  Moreover, $m < p$ implies $k < j$ so that $j \mid p$ and hence $j = p$.  Therefore $jm = mp = [F(\alpha, \beta):F]$.

\item \emph{Suppose $\ch F \neq p$.  Show that $F(\alpha, \beta) = F(\alpha + \beta)$.}

We have $F(\alpha, \beta) = F(\alpha)(\alpha+\beta) = F(\alpha)(\beta)$.  From the previous part $[F(\alpha, \beta):F(\alpha)] = p$, so that $[F(\alpha)(\alpha+\beta):F(\alpha)] = p$, also.

\item \emph{For distinct primes $p,q$ compute $[\Q(\sqrt[q]{p}, \sqrt[p]{q}):\Q]$ and show that $\Q(\sqrt[q]{p}, \sqrt[p]{q}) = \Q(\sqrt[q]{p}+\sqrt[p]{q})$.}

The minimal polynomials of $\sqrt[q]{p}$ and $\sqrt[p]{q}$ over $\Q$ are $x^q - p$ and $x^p -q$, which are easily seen to be irreducible by Eisenstein.  Therefore $[Q(\sqrt[q]{p}:\Q] = q$ and $[Q(\sqrt[p]{q}:\Q] = p$.  Since $p \neq q$ either $p < q$ or $q < p$, so by the first part $[\Q(\sqrt[q]{p}, \sqrt[p]{q}): \Q] = pq$.  The second statement follows directly from $(b)$ since $\ch \Q = 0$.

\item \emph{Show that $\Q(\sqrt[p]{2}, \sqrt[q]{2}) = \Q(\sqrt[p]{2} + \sqrt[q]{2})$ when $p$ and $q$ are distinct primes.}

The minimal polynomials of $\sqrt[p]{2}$ and $\sqrt[q]{2}$ are $x^p-2$ and $x^q - 2$, which are irreducible by Eisenstein.  Since $p \neq q$ it must be the case that $p < q$ or $q < p$, so by the first part $[\Q(\sqrt[p]{2}, \sqrt[q]{2}):\Q] = pq$.  Then, again, by the second part, $\Q(\sqrt[p]{2}, \sqrt[q]{2}) = \Q(\sqrt[p]{2} + \sqrt[q]{2})$.

\end{enumerate}

\item \emph{Let $E/\Q$ and $E'/\Q$ be field extensions and $\sigma: E \rightarrow E'$ a ring homomorphism.  Show that $\sigma \mid_\Q = 1_\Q$, i.e., $\sigma$ is a $\Q$-monomorphism.}

This follows directly from known properties of ring homormorphisms, namely, that they fix both the multiplicative and additive identities and preserve multiplicative and additive inverses.  Since $E$, $E'$, and $\Q$ share the same identities by hypothesis it follows that $\sigma(1) = 1$ and $\sigma(-1) = -1$ so that $\sigma(m) = m$ for all $m \in \Z$ (since $\Z$ is generated by $1$ or $-1$).  But then $1 = \sigma(1) = \sigma(nn^{-1}) = n\sigma(n^{-1})$, for $n \in \Z$.  Hence
\[
\sigma\left(\frac{p}{q}\right) = \frac{p}{q}
\]

for all $p,q \in \Z$, $q \neq 0$.  Again, since $\sigma(0) = 0$ it follows that $\sigma$ fixes all of $\Q$, i.e., $\sigma$ is a $\Q$ monomorphism.

\item \emph{Let $E/F$ be a finite normal field extension and $F'$ a field such that $F \subset F' \subset E$.  Let $g \in F'[x]$ be irreducible with $g(\alpha) = 0$ for some $\alpha \in E$.  Show that $g = c\prod(x-\alpha_i)$ for $c \in F'^\ast$, $\alpha_i \in E$, $1 \leq i \leq d = \deg g$.}

Let $f$ be the minimal polynomial of $\alpha$ over $h$.  Then $f$ splits into linear factors over $E$.  Since $\alpha$ is a root of $g$ there exists some $h \in F'[x]$ such that $f = gh$.  But then since $f$ splits into linear factors in $E$ and $E$ is a UFD, $gh$ must split and, in particular, $g$ must split times perhaps a leading coefficient in $F'^\ast$.

\item \emph{Let $M$ be a field and $E_1, E_2, F$ subfields of $M$ with $F \subset E_1$ and $F \subset E_2$.  Assume $E_1/F$ and $E_2/F$ are both normal extensions.  If they are also finite show that $(E_1E_2) / F$ and $(E_1 \cap E_2)/F$ are finite normal extensions.}

If $E_1/F$ and $E_2/F$ are finite and normal then there exist polynomials $f_1, f_2 \in F[x]$ such that $E_1/F$ and $E_2/F$ are the splitting fields of $f_1$ and $f_2$, respectively.  Then the product of these two polynomials splits in $E_1E_2$, and no smaller field can do so since $E_1E_2$ is by definition the smallest field containing both $E_1$ and $E_1$.  Hence $E_1E_2$ is a splitting field for $f_1f_2$, and is therefore normal.

If $f$ is an irreducible polynomial with a root in $E_1 \cap E_2$ then by the normality of $E_1$ all of its roots are in $E_1$.  Similarly, all of its roots are in $E_2$, also, and hence all of its roots are in $E_1 \cap E_2$.  But then $f$ certainly splits over $E_1 \cap E_2$ so that $(E_1 \cap E_2)/F$ is a normal extension.

\item \emph{Let $E/F$ be an algebraic (not necessarily finite) extension and let $\sigma: E \rightarrow E$ be an $F$-monomorphism.  Show that $\im \sigma = E$.}

Let $Z(f) = \set{\alpha \in E \mid f(\alpha) = 0}$.  Since $E/F$ is algebraic every $x \in E$ is in at least one $Z(f)$, namely, $Z(g)$ where $g$ is the minimal polynomial of $x$ over $F$.  Then $E = \bigcup_{f \in F[x]} Z(f)$.  To show that $\sigma(E) = E$ it therefore suffices to show that $\sigma(Z(f)) = Z(f)$.  Let $\alpha \in Z(f)$, then there exists some constants in $F$ such that
\[
\alpha^n + \cdots + a_1\alpha + a_0 = 0
\]

Applying $\sigma$ to this gives
\[
\sigma(\alpha)^n + \cdots + a_1\sigma(\alpha) + a_0 = 0
\]

since $\sigma$ is an $F$-monomorphism.  Therefore $f(\sigma(\alpha)) = 0$ and $\sigma(\alpha) \in Z(f)$.  Since $\sigma$ is injective it has a left inverse, and the same argument applies to show that $\sigma^{-1}(Z(f)) \subset Z(f)$ so that $Z(f) \subset \sigma(Z(f))$.  Then

\[
\sigma(E) = \sigma\left(\bigcup_{f \in F[x]} Z(f)\right) = \bigcup_{f \in F[x]} \sigma(Z(f)) = \bigcup_{f \in F[x]} Z(f) = E
\]

\item \emph{Let $E/F$ be a splitting field of $f \in F[x]$ where $\deg f = n \geq 1$.  Show that $[E:F] \mid n!$.}

This is obviously true for $n=1$, so consider the $n-1$ case.  First assume $f$ is irreducible.  Then let $\alpha \in E$ be a root of $f$.  Since $f$ is irreducible it has the same degree as the minimal polynomial of $\alpha$ over $F$, and therefore $[F(\alpha):F] = n$.  But by the inductive hypothesis $[E:F(\alpha)] \mid (n-1)!$ so that $$[E:F] = [E:F(\alpha)][F(\alpha):F] \mid n(n-1)! = n!$$

If $f$ is reducible then write $f = f_1f_2$ where $f_1$ is irreducible over $F$ and $\deg f_1 = m$.  Let $E'$ be the splitting field of $f_1$ over $F$.  Then $E' \subset E$.  Then $[E':F] \mid m!$ and $[E:E'] \mid (n-m)!$ by the inductive hypothesis, so that $[E:F] \mid m!(n-m)! \mid n!$.

\item
\begin{enumerate}
\item \emph{Compute $\Aut(\Q(\sqrt[3]{2}))$.}

Any element of $\Q(\sqrt[3]{2})$ can be written as $a + b\sqrt[3]{4} + c\sqrt[3]{2}$ for $a,b,c \in \Q$.  Since any homomorphism of $\Q(\sqrt[3]{2})$ fixes $\Q$ it must be that, for $\sigma \in \Aut(\Q(\sqrt[3]{2}))$,
\[
\sigma(a + b\sqrt[3]{4} + c\sqrt[3]{2}) = a + b\sigma(\sqrt[3]{4}) + c\sigma(\sqrt[3]{2})
\]

Since $\sigma(\sqrt[3]{2}^3) = \sigma(2) = 2$, $\sigma(\sqrt[3]{2})^3 = 2$.  But the only real number that satisfies this, and hence the only number in $\Q(\sqrt[3]{2})$, is $\sqrt[3]{2}$.  Therefore $\sigma(\sqrt[3]{2}) = \sqrt[3]{2}$.  From this $\sigma(\sqrt[3]{4})$ is completely determined as $2 = \sigma(\sqrt[3]{2})\sigma(\sqrt[3]{4})$, so that $\sigma(\sqrt[3]{4}) = \sqrt[3]{4}$.  Therefore the only automorphism of $\Q(\sqrt[3]{2})$ is the identity map.

\item \emph{Is $\Q(\sqrt[3]{2})/\Q$ normal?}

This extension is not normal because the irreducible polynomial $x^3-2$ has a root in $\Q(\sqrt[3]{2})$ but does not split over $\Q(\sqrt[3]{2})$, viz.,
\[
x^3 - 2 = (x-\sqrt[3]{2})(x^2 + \sqrt[3]{2}x + \sqrt[3]{4})
\]

The quadratic factor on the right hand side is irreducible over $\Q(\sqrt[3]{2})$.

\item \emph{Show that if $[E:F] = 2$ then $E/F$ is normal.}

If $[E:F] = 2$ then there exists some $\alpha \in E$ such that $E = F(\alpha)$, and the minimal polynomial of $\alpha$ has degree $2$.  Call this polynomial $f$.  Then $f$ has at least one linear factor, viz., $x-\alpha$, and the remaining factor must also be linear by degree considerations.  Therefore $E/F$ is a splitting field for $f$, and hence normal.

\item \emph{Let $F = \Q$, $F' = \Q(\sqrt{2})$, and $E = \Q(\sqrt[4]{2})$.  Show that $E/F'$ and $F'/F$ are finite normal, but $E/F$ is not.}

Both $E/F$ and $F'/F$ are obviously finite.  $F'/F$ is normal since it is the splitting field of $x^2 - 2$ over $\Q$.  Similarly, $E/F'$ is normal since it is the splitting field of $x^2 - \sqrt{2}$ over $\Q(\sqrt{2})$. $E/F$, however, is not normal, as $x^4-2$ has two roots in $E$, but does does not split over $E$.  That is,
\[
x^4 - 2 = (x^2 - \sqrt{2})(x^2+\sqrt{2}) = (x-\sqrt[4]{2})(x+\sqrt[4]{2})(x^2+\sqrt{2})
\]

and $x^4 - 2$ cannot be factored further.
\end{enumerate}

\item
\begin{enumerate}
\item \emph{Let $E/F$ be a finite extension and $E = F(\alpha_1, \ldots, \alpha_r)$.  Let $f_i$ be the minimal polynomial of $\alpha_i$ and define $f = \prod f_i$.  Let $N/E$ be a splitting field of $f$ over $E$.  Show that $N/F$ is a normal extension.}

Since $N/E$ is a finite extension which splits $f$ it is also normal.  Any irreducible polynomial $g \in F[x]$ is also a polynomial in $E[x]$, irreducible over $F$.  If $g$ is reducible over $E$ then it can be factored into the product of irreducible polynomials over $E$, each of which splits in $N$ by hypothesis.  Since everything in consideration is a UFD, it follows that $g$ splits into linear factors and therefore $N/F$ is normal.

\item \emph{With $E$ as above, let $M/F$ be a finite normal extension with $F \subset E \subset M$.  Let $f = \prod f_i$ so that $f = \prod_{i=1}^n (x - \alpha_i)$, $n \geq r$.  Let $N = F(\alpha_1, \ldots, \alpha_n)$.  Show that $\abs{\HomAlg_F(E,M)} = \abs{\HomAlg_F(E,N)}$.}

Since each $f_i$ has a root in $M$ and $M$ is normal, each $f_i$ splits and hence their product, $f$, must split, too.  But as $N$ is the smallest field over which $f$ splits, it must be that $N \subset M$.  Hence any $F$-monomorphism from $E$ into $N$ is also an $F$-monomorphism into $M$ by inclusion.

If $\sigma: E \into M$ is an $F$-monomorphism then $f(\sigma(\alpha_i)) = 0$, as in the fifth exercise.  Hence $\sigma(\alpha_i) \in N$ and $\sigma(E) = F(\sigma(\alpha_1), \ldots, \sigma(\alpha_r)) \subset N$.  So any such $\sigma$ is in fact a monomorphism into $N$.  Therefore $\abs{\HomAlg_F(E,M)} = \abs{\HomAlg_F(E,N)}$ and, in fact, the sets are equal.

\item \emph{With $E/F$ and $f$ in $(a)$, let $N/E$ and $N'/E$ be splitting fields of $f$ over $E$.  Show that \\ $\abs{\HomAlg_F(E,N)} = \abs{\HomAlg_F(E,N')}$}

Between two splitting fields $N/E$ and $N'/E$ of the same polynomial $f$ there exists an isomorphism, say, $\tau$, which fixes $F$. Then define a map by $\sigma \mapsto \tau \circ \sigma$.  This is surjective since for any $\sigma: E \rightarrow N'$, $\tau^{-1} \circ \sigma: E \rightarrow N$ maps to $\sigma$.  It is injective since if $\tau \circ \sigma_1 = \tau \circ \sigma_2$ then $(\tau \circ sigma_1)(x) = (\tau \circ \sigma_1)(x)$ for all $x$, which implies that $\sigma_1(x) = \sigma_2(x)$ for all $x$ by the injectivity of $\tau$.

Therefore $\abs{\HomAlg_F(E,M)} = \abs{\HomAlg_F(E,N)}$.

\item \emph{With $E/F$ as in $(a)$, let $M/F$ and $M'/F$ be finite normal extensions with $F \subset E \subset M$ and $f \subset E \subset M'$.  Show that $\abs{\HomAlg_F(E,M)} = \abs{\HomAlg_F(E,M')}$.}

By above there exist $N/E$, $N'/E$ splitting fields of $f$ with $N \subset M$ and $N' \subset M'$.  Using the previous parts, it follows that
\[
\abs{\HomAlg_F(E,M)} = \abs{\HomAlg_F(E,N)} = \abs{\HomAlg_F(E,N')} = \abs{\HomAlg_F(E,M')}
\]
\end{enumerate}
\end{enumerate}
\end{document}
