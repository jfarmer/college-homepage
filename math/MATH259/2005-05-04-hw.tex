\documentclass[10pt]{article}

\usepackage{homework}
\usepackage{enumerate}
\usepackage{geometry}

\geometry{letterpaper}

\textwidth = 6.5 in
\textheight = 9 in
\oddsidemargin = 0.0 in
\evensidemargin = 0.0 in
\topmargin = 0.0 in
\headheight = 0.0 in
\headsep = 0.0 in
\parskip = 0.2in
\parindent = 0.0in


\title{MATH 259: Homework \#5}
\author{Jesse Farmer}
\date{04 May 2005}
\begin{document}
\maketitle
\begin{enumerate}

\item \emph{Let $F$ be a field with $\ch F \neq 2$.}
\begin{enumerate}
\item \emph{If $K = F(\sqrt{d_1}, \sqrt{d_2})$ where $d_1, d_2 \in F$ have the property that none of $d_1, d_2,$ or $d_1d_2$ is a square in $F$, prove that $K/F$ is a Galois extension with $\Gal(K/F)$ isomorphic to the Klein $4$-group.}

$K$ is the splitting field of the polynomial $(x^2 - d_1)(x^2 - d_2)$, which is irreducbile as none of $d_1, d_2$, or $d_1d_2$ are squares in $F$, so that $K/F$ is a normal extension.  Since $\ch F \neq 2$, this polynomial is also separable, so, in fact, $K/F$ is Galois.  Consider $\Gal(K/F)$.   Any automorphism fixing $F$ is determined completely by its action on the generators $\sqrt{d_1}$ and $\sqrt{d_2}$, which must be mapped to to $\pm \sqrt{d_1}$ and $\pm\sqrt{d_2}$, respectively.  S

ince $[K:F] = 4$, $|\Gal(K/F)| = 4$, so these are, in fact, all the automorphisms.  Define $\sigma \in \Gal(K/F)$ by $\sqrt{d_1} \mapsto - \sqrt{d_1}$ and $\sqrt{d_2} \mapsto \sqrt{d_2}$.  Similarly, define $\tau \in \Gal(K/F)$ by $\sqrt{d_1} \mapsto \sqrt{d_1}$ and $\sqrt{d_2} \mapsto -\sqrt{d_2}$.  Any element of $k \in K$ can be written as
\[
k = a + b\sqrt{d_1} + c\sqrt{d_2} + d\sqrt{d_1d_2}
\]

Simple calculation shows that $\sigma(\sqrt{d_1d_2}) = -\sqrt{d_1d_2} = \tau(\sqrt{d_1d_2})$.  Furthermore, from their definitions, it is clear that $\sigma^2 = \tau^2 = 1$.  Then $\sigma\tau(\sqrt{d_1}) = - \sqrt{d_1}$ and $\sigma\tau(\sqrt{d_2}) = - \sqrt{d_2}$.  Hence $\sigma\tau$ is an element of order $2$ distinct from $\sigma$ and $\tau$.  But this is a characterization for $V_4$, the Klein $4$-group: a group of order $4$ whose nonidentity elements each have order $2$.  Hence $\Gal(K/F) \iso V_4$.

\item \emph{Conversely, suppose that $K/F$ is a Galois extension with $\Gal(K/F)$ isomorphic to the Klein $4$-group.  Prove that $K=F(\sqrt{d_1}, \sqrt{d_2})$ where $d_1, d_2 \in F$ have the property that none of $d_1, d_2$, or $d_1d_2$ is a square in $F$.}

Note that because $K/F$ is Galois, $[K:F] = 4$.  Every subgroup of $V_4$ is normal, and therefore there exist three distinct intermediary fields $E$ between $K$ and $F$.  Furthermore, since each element in $V_4$ has order $2$, it follows that $[E:F] = 2$ for all such $E$.  That is, every intermediary field is a quadratic extension of $F$.  In particular, there exist $d_1, d_2$ such that $E_1 = F(\sqrt{d_1})$ and $E_2 = F(\sqrt{d_2})$ where neither $d_1$ nor $d_2$ are squares.  Hence $K = F(\sqrt{d_1}, \sqrt{d_2})$ since $E_1$ and $E_2$ are distinct.  $\sqrt{d_1d_2} \notin F$ since, otherwise, $[K:F]$ would be $3$.

\end{enumerate}

\item
\begin{enumerate}
\item \emph{Prove that $x^4 - 2x^2 - 2$ is irreducible over $\Q$.}

This polynomial is irreducible since it is Eisenstein at $2$.

\item \emph{Show the roots of this quartic are of the form
\begin{align*}
\alpha_1 = \sqrt{1+\sqrt{3}} & \quad \alpha_3 = -\sqrt{1+\sqrt{3}} \\
\alpha_2 = \sqrt{1-\sqrt{3}} & \quad \alpha_4 = -\sqrt{1-\sqrt{3}}
\end{align*}}

Obviously $\alpha_1$ is a root if and only if $\alpha_3$ is, and similarly for $\alpha_2$ and $\alpha_4$.  But
\[
\alpha_1^4 = 4 + 2\sqrt{3} \mbox{ and } \alpha_1^2 = 1 + \sqrt{3}
\]

so $\alpha_1^4 - 2\alpha_1^2 - 2 = 0$.  It is exactly the same for $\alpha_2$.

\item \emph{Let $K_1 = \Q(\alpha_1)$ and $K_2 = \Q(\alpha_2)$.  Show that $K_1 \neq K_2$, and $K_1 \cap K_2 = \Q(\sqrt{3}) := F$.}

Clearly $K_1 \neq K_2$ since the former is a subfield of the reals and the latter is not.  Then since $\Q(\sqrt{3}) \subset K_1 \cap K_2$,
\[
4 = [K_1: \Q] = [K_1:K_1 \cap K_2][K_1 \cap K_2:\Q(\sqrt{3})][\Q(\sqrt{3}):\Q]
\]

But $[\Q(\sqrt{3}):\Q] = 1$ and $[[K_1:K_1 \cap K_2] > 1$ since $K_1 \neq K_2$, so that $[[K_1 \cap K_2:\Q(\sqrt{3})] = 1]$.

\item \emph{Prove that $K_1, K_2$, and $K_1K_2$ are Galois over $F$ with $\Gal(K_1K_2/F)$ the Klein $4$-group.  Write out the elements of $\Gal(K_1K_2/F)$ explicitly.  Determine all the subgroups of the Galois group and give their corresponding fixed subfields of $K_1K_2$ containing $F$.}

Over $F$, the minimal polynomials of $\alpha_1$ and $\alpha_2$ are $x^2 - \sqrt{3} - 1$ and $x^2 + \sqrt{3} - 1$, respectively, both of which split.  Hence $K_1$ and $K_2$ are Galois.  $K_1K_2$ is the splitting field of $x^4 - 2x^2 - 2$ over $F$, and is therefore also Galois.

$\Gal(K_1K_2/F) \iso V_4$, which can be seen by writing out the element of the automorphism group explicitly.  First, any automorphism must send $\alpha_1$ to $\pm \alpha_1$, and similarly, must send $\alpha_2$ to $\pm \alpha_2$.  Moreover, an automorphism is uniquely defined by its action on the generators.  Define $\sigma \in \Gal(K/F)$ by $\alpha_1 \mapsto - \alpha_1$ and $\alpha_2 \mapsto \alpha_2$.  Similarly define $\tau \in \Gal(K/F)$ by $\alpha_1 \mapsto \alpha_1$ and $\alpha_2 \mapsto - \alpha_2$.  Then $\sigma^2 = \tau^2 = 1$, and furthermore, $\sigma\tau$ acts by sending both $\alpha_1$ and $\alpha_2$ to their additive inverses.  Hence $(\sigma\tau)^2 = 1$ and $\sigma\tau$ is the fourth element.  There are no more automorphisms since we know $\abs{\Gal(K/F)} = 4$, and therefore $\Gal(K/F) \iso V_4$.

\item \emph{Prove that the splitting field of $x^4 - 2x^2 - 2$ over $\Q$ is of degree $8$ with dihedral Galois group.}

Let $K/F$ be the splitting field of $x^4 - 2x^2 - 2$ over $\Q$.  From the previous part we know that $K = F(\alpha_1, \alpha_2)$.  Furthermore, from the previous parts, it follows that $[K:F] = 8$ since $[\Q(\sqrt{3}):\Q] = 2$.

Define an automorphism by the cycle $(\alpha_1\,\alpha_2\,\alpha_3\,\alpha_4)$, and a second automorphism by $(\alpha_1\,\alpha_2)(\alpha_3\,\alpha_4)$.  But the group generated by these two permutations is precisely the dihedral group $D_8$, which can be seen by treating the former as a rotation of a square with vertices labeled clockwise from the northwest corner as $\alpha_1, \alpha_2, \alpha_3$, and $\alpha_4$, and the latter as a reflection about the vertical axis of symmetry.
\end{enumerate}

\item \emph{Let $F$ be a field and $f = x^p - a$, $a \in F$ and $p$ a prime.  Show that $f$ is reducible in $F[x]$ if and only if $f$ has a root in $F$.}

That this condition is necessary is obvious, since if $f$ has a root then it factors into $(x-r)g$ for some $r \in F$ and $g \in F[x]$.

The converse was already shown for $\ch F = p$, so assume $\ch F \neq p$.  Then $x^p-a$ only has simple roots since the derivative $px^{p-1}$ has only one $(p-1)$-fold root: zero.  Let $E/F$ be the splitting field of $f$ over $F$ so that $E=(\alpha, \zeta)$, where $\zeta$ is a primitive $n^{th}$ root of unity and $\alpha \in E$ is some root of $x^p-a$.  Write
\[
f(x) = \prod_{i=0}^{p-1}(x - \zeta^i\alpha)
\]

Suppose $f$ is reducible.  If $\alpha \in F$ then we are done, so assume $\alpha \in E \setminus F$.  Let $g$ be the minimal polynomial of $\alpha$ over $F$ and $\deg g = r$.  Then $1 < r < p$ and
\[
g(x) = \prod_{k=1}^r (x - \zeta^{i_k}\alpha)
\]

where $\set{i_k}$ is just some subset of $\set{1, \ldots, p-1}$.  From a quick calculation we see that the constant term in $g$ must be $\pm \zeta^l\alpha^r \in F$ for some $l \geq 0$.  Then, in either case, $\zeta^l\alpha^r \in F$.  Since $(r,p) = 1$ and $\zeta \in \mu_p$, a group of order $p$, there exists some $\zeta' \in \mu_p$ such that $\zeta = \zeta'^r$.  Then $\zeta'^{rl}\alpha^r = (\zeta'^l\alpha)^r \in F$.  Also $(\zeta'^l\alpha)^p = \alpha^p = a \in F$.  From these two equations it follows that $\zeta'^l\alpha \in F$.  But $\zeta'\alpha$ is a root of $f$, and we are done.

\item \emph{Let $E/F$ be a finite extension.  Show that $E/F$ is a simple extension if and only if the number of intermediary subfields between $F$ and $E$ are finite.}

Suppose $E$ is simple with $E = F(\alpha)$ and $[E:F] = n < \infty$.  Let $f$ be the minimal polynomial of $\alpha$ over $F$.  Let $T$ be the set of all intermediary subfields between $F$ and $E$, and let $T' = \set{g \in E[x] \mid g \mbox{ monic and } g \mid f}$.  Then $T'$ is finite (in fact, of cardinality less than or equal to $2^n$).  For $K \in T$ let $f_K$ be the minimal polynomial of $\alpha$ over $K$.  Define a function $T \rightarrow T'$ by $K \mapsto f_K$.  We claim this is injective.  Assume $f_K = f_L = g$ for $K,L \in T$.  Then the coefficients of $g$ reside in $K \cap L$, so $g$ is also the minimal polynomial of $\alpha$ over $K \cap L$.  But since $(K \cap L)(\alpha) = E$, it follows that $[K: K \cap L] = 1$ and $[L : K \cap L] = 1$ and hence $K = L$.  Therefore $|T| \leq |T'| \leq 2^n$.

To prove the converse, assume $T$ from above is finite.  We may assume without loss of generality that $F$ is infinite, since a finite extension of a finite field is simple.  Let $\alpha \in E$ such that $[F(\alpha):F]$ is maximal among all simple subextensions of $E/F$.  Assume for contradiction that $F(\alpha) \neq E$, and choose $\beta \in E \setminus F(\alpha)$.  Consider all subfields of the form $F(\alpha + c\beta)$ for $c \in F$.  Since $F$ is infinite, it follows from the pigeonhole principle that there exist distinct $c,c' \in F$ such that $F(\alpha + c\beta) = F(\alpha + c'\beta)$.  Immediately we see that this implies $\beta \in F(\alpha + c\beta)$ (since $(c-c')\beta$ is an element), which in turn implies $\alpha \in F(\alpha + c\beta)$.  Hence $F(\alpha, \beta) = F(\alpha + c\beta)$.  But $[F(\alpha, \beta):F] > [F(\alpha):F]$, contradicting the maximality of the latter.  Therefore $E = F(\alpha)$.

\item \emph{Let $p$ be a prime number and $E = F(\alpha)$ where $\alpha^p = a \in F^\ast$ and $x^p - a$ irreducible.  Suppose $\ch F \neq p$, and let $\beta \in E$.  Show that $\beta^p = b \in F$ if and only if $\beta = c \alpha^i$ for some $c \in F$ and $i \geq 0$.}

\item \emph{Compute $[\Q(\sqrt[p]{2}, \sqrt[p]{3}): \Q]$, where $p$ is a prime.}

\item \emph{Show that $\Q(\sqrt[p]{2}, \sqrt[p]{3}) = \Q(\sqrt[p]{2} + \sqrt[p]{3})$.}

\item \emph{Let $E/\Q$ be the splitting field of $x^p - 3$, where $p$ is a prime.  Show that $\Gal(E/\Q)$ has a normal cyclic subgroup of order $p$ with quotient an abelian group of order $p-1$.}
\end{enumerate}
\end{document}
