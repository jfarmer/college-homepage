\documentclass[10pt]{article}

\usepackage{homework}
\usepackage{enumerate}
\usepackage{geometry}

\geometry{letterpaper}

\textwidth = 6.5 in
\textheight = 9 in
\oddsidemargin = 0.0 in
\evensidemargin = 0.0 in
\topmargin = 0.0 in
\headheight = 0.0 in
\headsep = 0.0 in
\parskip = 0.2in
\parindent = 0.0in


\title{MATH 259: Homework \#6}
\author{Jesse Farmer}
\date{11 May 2005}
\begin{document}
\maketitle
\begin{enumerate}

\item \emph{Let $E/F$ be a field extension with $|F| < \infty$ and $[E:F] = n$.  Show that there exists a subfield $E' \subset E$ with $[E':F] = d$ if and only if $d \mid n$, and that this $E'$ is unique.  Conclude that there exists a subfield $E'$ of $E$ with $|E'| = p^d$ if and only if $d \mid n$.}

I assume that $|F| = p$, $p$ a prime number, since otherwise the rest of the exercise does not work.  Then $\Gal(E/F)$ is cyclic of order $n$, and from group theory we know that $d \mid n$ if and only if there exists a subgroup $H$ of $\Gal(E/F)$ of order $d$, and that any such subgroup is unique.  Writing $G = \Gal(E/F)$, the fixed field $E'$ of $H$ is therefore the unique field such that $[E':F] = |G:H| = \frac{n}{d} = a \in \Z$.  Of course, this suffices, as $\frac{n}{d} = a$, so that $d \mid n$ if and only if $a \mid n$.

Let $E$ be as above, then $|E| = p^n$.  From above $[E':F] = d$ if and only if $d \mid n$, which is true if and only if $|E| = p^d$, since $E'$ is the splitting field of $x^{p^d} - x$.

\item \emph{Let $F$ be a finite field with $|F| = q$ and $[E:F] = n$.  Let $R_d(F)$ be the set of all monic irreducible polynomials of degree $d$ in $F[x]$.  Show that $$x^{q^n} - x = \prod_{d \mid n}\left(\prod_{f \in R_d(F)} f\right)$$}

If $\alpha \in E$ is a root of $x^{q^n} - x$ then its minimal polynomial over $F$ must have degree $d$ dividing $n$, and hence is in $R_d(F)$ for the appropriate $d$.

To show the converse, note that any two monic irreducible factors of the same degree in the right-hand polynomial must have the same splitting field by the uniqueness derived in the first question.  Since at least one polynomial of degree $d \mid n$ splits, then, it follows that every polynomial of degree $d \mid n$ splits in $E$.  Moreover, from the construction of $\F_{q^n}$, any polynomial which splits in $E$ must have linear factors that are also factors of $x^{q^n} - x$.  Hence the right-hand polynomial divides $x^{q^n} - x$.

\item \emph{Let $E/F$ be a Galois extension with $[E:F] = n$.  Suppose $p$ is a prime such that $p \mid n$.  Show that there is a subfield $F'$ of $E$ with $F \subset F' \subset E$ and $[F':F] = \frac{n}{p}$.  Furthermore, show that if $n = p^rm$ where $p \nmid m$ then there exists a subfield $F'$ as above with $[E:F'] = p^r$.}

Consider $G = \Gal(E/F)$.  $|G| = n$, so by Cauchy's theorem for every $p \mid n$ there exists an element of order $p \in G$.  In particular, the subgroup generated by this element has order $p$, so there exists $H \leq G$ with $|H| = p$.  Let $K$ be the fixed field of $H$.  Then the fundamental theorem says $\frac{n}{p} = |G:H| = [K:F]$, so $K$ is precisely the field for which are are looking.

Now assume that $n = p^rm$ where $p \nmid m$.  Then from Sylow's theorem we know there exists a Sylow $p$-subgroup $H$ of $G = \Gal(E/F)$ of order $p^r$.  Let $K$ be the fixed field of this subgroup.  Then $p^r = |H| = [E:K]$, so, again, $K$ is precisely the field for which are are looking.

\item \emph{Let $E/\Q$ be a finite normal extension.  Suppose $\sqrt[3]{p} \in E$ where $p$ is prime.  Show that $\Gal(E/\Q)$ is not abelian.}

To show the contrapositive assume that $\Gal(E/\Q)$ is abelian.  Then every subgroup is normal, and each corresponding fixed field is a Galois extension.  However, $\Q(\sqrt[3]{p})$ is easily seen to not be Galois since the splitting field of $x^3 - p$, which is monic irreducible since it is Eisenstein at $p$, contains one real root and two purely complex roots and $\Q(\sqrt[3]{p})$ is a subset of the reals.  Hence it is not a normal extension, and the corresponding subgroup of $\Gal(E/\Q)$ is not normal.  Therefore $\sqrt[3]{p} \notin E$.

\item \emph{Let $E/F$ be a cyclic extension of degree $n$.  Show that for all $d \mid n$, $d > 0$, there exists a cyclic extension $F'/F$ with $F' \subset E$ such that $[F':F] = d$.}

Let $G = \Gal(E/F) \iso \Z/n\Z$.  By the fourth isomorphism theorem for groups, for $d \mid n$,
\[
d\Z/n\Z \unlhd \Z/n\Z
\]

By the fundamental theorem of Galois theory there exists a cyclic Galois extension $F'$ of $F$ with $[F':F] = \abs{d\Z/n\Z : \Z/n\Z} = d$.

\item \emph{Let $E/F$ and $E'/F$ be field extensions with $|E| = |E'| < \infty$.  Prove or disprove that there is an $F$-isomorphism $\sigma: E \rightarrow E'$.}

Let $f$ and $f'$ be the polynomials of which $E$ and $E'$ are splitting fields, respectively, which exist because any finite extension of a finite field is normal.  Consider the splitting field $K$ of $ff'$.  Then $[K:E], [K:E'] \mid n$ and $[K:E] = [K:E']$.  Hence, by the uniqueness part of the first exercise, $E = E'$ and there certainly exists an $F$-isomorphism between these two fields.

\item \emph{Let $E = \Q(\sqrt[4]{p})$, $p$ a prime.  Let $\beta \in E$ be such that $\beta^4 \in \Q$.  Show that $\beta = c(\sqrt[4]{p})^i$ for some $c \in \Q$ and $i \geq 0$.}

\item \emph{Compute $\Q(\sqrt[4]{2}, \sqrt[4]{3})$ and $\Q(\sqrt[4]{2} + \sqrt[4]{3})$.}

I don't know what this question means.  Does it mean compute the Galois group?  Or perhaps compute the degree of the extension?

\item \emph{Let $p$ be a prime and $E/\Q$ the splitting field of $x^p-1$.  Show that $E/\Q$ is a cyclic extension of degree $p-1$.}

$x^p-1$ is reducible over $\Q$ into $(1-x)(1 + \cdots x^{p-1})$.  Hence the splitting field of $x^p-1$ is of degree $p-1$.  Furthermore, $\Gal(E/\Q) \iso (\Z/p\Z)^\times$, which for $p$ prime is cyclic of order $p-1$.

\item \emph{Let $E_n$ be the splitting field of $x^n-1$ over $\Q$.  Show that $\sqrt[3]{2} \notin E_n$ for all $n \geq 1$.}

Write $E_n = \Q(\zeta_n)$ where $\zeta_n$ is a primitive $n^{th}$ root of unity.  Then by Theorem 26 in chapter 14 of Dummit and Foote, $\Gal(E_n/\Q) \iso (\Z/n\Z)^\times$, the multiplicative group of units of $Z/n\Z$.  In particular, $E_n/\Q$ is a finite normal abelian extension.  Since it is abelian $\sqrt[3]{2} \notin E_n$ by fourth exercise.

\item \emph{Let $E_n/\Q$ be as above.  Let $\sigma \in \Gal(E_n/\Q)$, and write $E = \Q(\zeta)$ where $\zeta$ is a primitive $n^{th}$ root of unity.  Suppose $\sigma(\zeta) = \zeta^a$.  Show that for all $\eta \in \mu_n$, $\sigma(\eta) = \eta^a$.  Furthermore, $a$ and $n$ are relatively prime.}

Since $\mu_n = \brac{\zeta}$, write $\eta = \zeta^b$ for some $b < n$.  Then 
\[
\sigma(\eta) = \sigma(\zeta^b) = \sigma(\zeta)^b = \zeta^{ba} = \left(\zeta^b\right)^a = \eta^a
\]

Assume $\gcd(a,n) = d > 1$.  Then $\sigma(1) = 1$ and
\[
\sigma(\zeta^{\frac{n}{d}}) = \zeta^{\frac{na}{d}} = 1
\]

since $\frac{a}{d} = k \in \Z$.  Hence $\sigma$ is not injective, and therefore certainly not a bijection.

\item \emph{Let $E$ be the splitting field of $x^{15} - 1$ over $\Q$.  Determine $[E:\Q]$ and $\varphi_{15}(x)$, the minimal polynomial of $\zeta$ over $\Q$, where $E = \Q(\zeta)$ and $\zeta^{15} = 1$.}

$[E:\Q] = \abs{\Gal(E/\Q)} = \varphi(15)$, where $\varphi$ is Euler's totient function.  But $\varphi(15) = 8$.  Factoring $x^{15} - 1$ over $\Q$ gives
\[
x^{15} - 1 = (x-1)(1 + x + x^2)(1 + x + x^2 + x^3 + x^4)(1 - x + x^3 - x^4 + x^5 - x^7 + x^8)
\]

hence $\varphi_{15} = 1 - x + x^3 - x^4 + x^5 - x^7 + x^8$.

\end{enumerate}
\end{document}
