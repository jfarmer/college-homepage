\documentclass[10pt]{article}

\usepackage{amsfonts}
\usepackage{amsmath}
\usepackage{amssymb}
\usepackage{amsthm}
\usepackage{eucal}
\usepackage{enumerate}
\usepackage{geometry}

\geometry{letterpaper}

\textwidth = 6.5 in
\textheight = 9 in
\oddsidemargin = 0.0 in
\evensidemargin = 0.0 in
\topmargin = 0.0 in
\headheight = 0.0 in
\headsep = 0.0 in
\parskip = 0.2in
\parindent = 0.0in

\newcommand{\brac}[1]{
\left\langle #1 \right\rangle
}

\newcommand{\powset}[1]{
\wp\left(#1\right)
}

\newcommand{\Aut}{\text{Aut}}
\newcommand{\Sym}{\text{Sym}}
\newcommand{\Syl}{\text{Syl}}
\newcommand{\Hom}{\text{Hom}}
\newcommand{\End}{\text{End}}
\newcommand{\Ann}{\text{Ann}}

\newcommand{\tor}{\text{tor}}
\newcommand{\ch}{\text{ch}}

\newcommand{\N}{\mathbb{N}}
\newcommand{\Z}{\mathbb{Z}}
\newcommand{\Q}{\mathbb{Q}}
\newcommand{\R}{\mathbb{R}}
\newcommand{\A}{\mathbb{A}}
\newcommand{\C}{\mathbb{C}}

\newcommand{\F}{\mathbb{F}}

\newcommand{\T}{\mathcal{T}}


\newcommand{\iso}{\cong}

\newtheorem{lemma}{Lemma}

\title{MATH 259: Homework \#2}
\author{Jesse Farmer}
\date{13 April 2005}
\begin{document}
\maketitle
\begin{enumerate}

\item \emph{Let $E/F$ be a field extension with $f,g \in F[x]$, both irreducible over $F$.  Let $\alpha, \beta \in E$ be such that $f(\alpha) = g(\beta) = 0$.  Show that $f$ is irreducible in $F(\beta)[x]$ if and only if $g$ is irreducible in $F(\alpha)[x]$.}

By the symmetry of the proposition it is sufficient to prove this statement in one direction only.  Let $n = \deg f$ and $m = \deg g$.  If $g$ is irreducible over $F(\alpha)$ then $[F(\alpha,\beta):F(\alpha)] = m$ and $[F(\alpha,\beta): F] = mn$.  But then $mn = [F(\alpha, \beta):F] = [F(\alpha,\beta):F(\beta)][F(\beta):F] = [F(\alpha,\beta):F(\beta)]m$, so that $[F(\alpha,\beta):F(\beta)] = n$ and therefore $f$ is irreducible over $F(\beta)$.

\item \emph{Let $E/F$ be a field extension with $[E:F] = p$, a prime.  Show that for all $\alpha \in E \setminus F$, $F(\alpha) = E$.}

Since $\alpha \in E \setminus F$ we have $F \subsetneq F(\alpha) \subseteq E$, so that $[F(\alpha):F] \neq 1$.  Then $$p = [E:F] = [E:F(\alpha)][F(\alpha):F]$$  Since $[F(\alpha):F] \neq 1$ and $p$ is prime it follows that $[E:F(\alpha)] = 1$ and therefore $E = F(\alpha)$.

\item \emph{Compute the minimal polynomial of $\sqrt{2} + \sqrt{3}$ over $\Q$.}

The minimal polynomial is $x^4 - 10x + 1$, which has $\sqrt{2} + \sqrt{3}$ and is irreducible by applying Eisentein to the polynomial at $x = y+1$.

\item \emph{Let $p,q$ be primes.  Show that $\Q(\sqrt{p}, \sqrt{q}) = \Q(\sqrt{p} + \sqrt{q}) = \Q(\sqrt{p} + 2\sqrt{q})$.}

The polynomial $x^4 - 2(p+q) + (p-q)^2$ is a minimal polynomial for $\sqrt{p} + \sqrt{q}$ over $\Q$.  Since $\Q(\sqrt{p} + \sqrt{q}) \subseteq \Q(\sqrt{p}, \sqrt{q})$ and they have the same degree over $\Q$, it follows that they must be equal. Similarly, the minimal polynomial of $\sqrt{p} + 2\sqrt{q}$ is $x^4 - 2(p+4q) + (p-4q)^2$.  This is a subfield of $\Q(\sqrt{p}, \sqrt{2})$, also, and has degree $4$ over $\Q$.  Therefore all three quadratic fields are equal.

\item 
\begin{enumerate}
\item \emph{Let $E/F$ be a quadratic extention of $F$ and suppose $\ch(F) \neq 2$.  Show that there exists an $\alpha \in F$ such that $\alpha^2 = d \in F$ and $\alpha \notin F$ and $E = F(\alpha)$.}

Pick some $\alpha \in E \setminus F$, which is possible since $[E:F] = 2$.  Since $E/F$ is a finite extension it is also algebraic, and therefore $\alpha$ is a root of the polynomial $$f(x) = x^2 + bx + c$$ for some $b,c \in F$.  We know from previous lectures that the quadratic formula is defined for fields with $\ch(F) \neq 2$.  That is, $$\alpha = \frac{-b \pm \sqrt{b^2 - 4c}}{2}$$  Since $\ch(F) \neq 2$, it follows that $4c = 0$ if and only if $c = 0$ and so $\sqrt{b^2-4c}$ is a number whose square is in $F$, but which is not in $F$ itself.  To see that $F(\alpha) = F' := F(\sqrt{b^2 - 4c})$ is clear: $F(\alpha) \subset F'$ from the quadratic equation, and the opposite inclusion is true since $\sqrt{b^2 - 4c} = \pm(b + 2\alpha)$.  From the second problem it follows that, in fact, $F' = E = F(\sqrt{b^2 - 4c})$.

\item \emph{Let $E/F$ be a quadratic extension with $\ch(F) \neq 2$.  Let $E = F(\alpha) = F(\beta)$ with $\alpha^2 = d \in F$ and $\beta^2 = h \in F$.  Then $\beta = \alpha \cdot c$ for some $c \in F^\ast$.  Consversely, if $\beta = \alpha \cdot c$ for $c \in F^\ast$ then $F(\beta) = F(\alpha) = E$.}

The converse is immediate as it implies that $\alpha = \beta \cdot c^{-1} \in F(\beta)$ and $\beta = \alpha \cdot c \in F(\alpha)$.  To show the opposite implication write $\alpha = x\beta  + y$ for some $x,y \in F$.  $x \in F^\ast$ since, if $x = 0$ then $\alpha \in F$.  So it is sufficient to show that $y = 0$.  But $\alpha^2 = (x\beta)^2 + 2xy\beta + y^2$, so that $2xy\beta \in F$.  As $\beta \in E \setminus F$ and $x \neq 0$, the only way this is possible is if $y = 0$, and hence $\alpha = \beta \cdot c$, or $\beta = \alpha \cdot c^{-1}$.

\item \emph{Let $F/\Q$ be a quadratic field with $F \subset \C$.  Show that $F = \Q(\sqrt{n})$ whewre $n = p_1\cdots p_n$, $p_i \neq p_j$ are prime if $F \subset \R$.  Otherwise, if $F \not\subset \R$, then $F = \Q(\sqrt{-n})$ for $n$ as above.}

From the first part it follows that $\Q(\sqrt{\frac{m}{n}}) = \Q(\sqrt{mn})$ since $\sqrt{\frac{m}{n}} \cdot n = \sqrt{mn}$.  Hence it suffices to consider the case of $\Q(\sqrt{n})$ where $n \in \Z$.  If $F \subset \R$ then clearly $n \in \Z_+$.  Assuming it is not a perfect square, since then $F = \Q$, we can reduce the powers of any prime dividing $n$ to $1$ since $p^{\lfloor \frac{k}{2} \rfloor} \sqrt{p^{k - 2\lfloor \frac{k}{2} \rfloor}} = \sqrt{p^k}$, where $k - 2\lfloor \frac{k}{2} \rfloor = 1$ if $k$ is odd and $0$ otherwise.  Hence $\Q(\sqrt{n}) = \Q(\sqrt{p_1 \cdots p_j})$ where each $p_i$ is a prime divisor of $n$ and $p_i \neq p_j$ if $i \neq j$.
\end{enumerate}

\begin{enumerate}
\item \emph{Let $A = \{p_1, \ldots, p_n\}$ be distinct primes.  Let $E_i = \Q(\sqrt{p_1}, \ldots, \sqrt{p_i})$.  Show for any two such subsets $B = \{p_{i_1}, \ldots, p_{i_s}\}$ and $C = \{p_{j_1}, \ldots, p_{j_r}\}$ of $A$ that $$\Q(\sqrt{p_{i_1} \cdots p_{i_s}}) = \Q(\sqrt{p_{j_1} \cdots p_{j_r}})$$ if and only if $B = C$.  Show that if $M_n$ is the set of all quadratic fields of this form, where $p_{i_k} < p_{i_{k+1}}$ (i.e., we discount permutations of the primes) then $|M_n| = 2^n -1$.}

Obviously if $B = C$ then the two quadratic fields are equal.  If $B \neq C$ then we can write $$n\sqrt{p_{i_1} \cdots p_{i_s}} = m\sqrt{p_{j_1} \cdots p_{j_r}}$$ for some $m,n \in \Z_+$ by the previous part.  Squaring both sides and cancelling any common prime numbers among $B$ and $C$ leaves us with $\sqrt{p_{k_1} \cdots p_{k_t}} = \frac{m}{n}$, which is impossible if $t > 0$.  It must therefore be the case that $\frac{m}{n} = 1$ and that $B=C$.

So see that $|M_n| = 2^n - 1$, encode the membership of the various $p_i$ as a binary number, with a $1$ in the $i^{th}$ position if $p_i$ is among the $p_{j_k}$ in $C$.  Each $n$-digit binary number represents a unique quadratic extension by the above, and hence $|M_n| = 2^n -1$, which is the number of $n$-digit binary numbers.
\item \emph{With notation as above, show that the number of quadratic subfields of $E_n$ is $2^n-1$, i.e., $M_n$ includes all the quadratic subfields.}

The same technique works here, after noting that if $E_i = E_j$ then $j = i$, since the square root of no prime is a rational multiple of another.  Hence there is a bijection between subfields of $E_k$ and $k$-digit binary numbers.  In particular, the number of subfields of $E_n$ is $2^n - 1$.
\end{enumerate}

\item \emph{Deduce from the previous exercise that $[\Q(\sqrt{p_1}, \ldots, \sqrt{p_n}) : \Q] = 2^n$.}

This follows immediately since $[E_i : E_{i-1}] = 2$ for $1 \leq i \leq n$, where $E_0 = \Q$.

\item \emph{Determine the splitting field and its degree over $\Q$ for $x^4 - 2$.}

The splitting field for this polynomial is $\Q(i, \sqrt[4]{2})$.  The degree is computed in exactly the same as the following exercise.

\item \emph{Determine the splitting field and its degree over $\Q$ for $x^4 + 2$.}

The splitting field of this polynomial is $\Q(i, \sqrt[4]{2})$ and it has degree $8$.  This can be seen as $\pm \sqrt[4]{2}$ are clearly a root of this polynomial, factoring this into two degree $2$ polynomials over $\Q(\sqrt[4]{2})$.  Adjoining $i$, which has a minimal polynomial of degree $2$ over $\Q(\sqrt[4]{2})$, gives roots to these two polynomials and hence this is the splitting field, with degree $4 \cdot 2 = 8$ over $\Q$.
\item \emph{Determine the splitting field and its degree over $\Q$ for $x^4+x^2+1$.}

The splitting field of this polynomial over $\Q$ is $\Q\left(\frac{1 + i\sqrt{3}}{2}\right)$, which has a minimal polynomial of degree $2$ over $\Q$ and therefore the splitting field has degree $2$.  Note that this polynomial is reducible over $\Q$ already since $x^4+x^2+1 = (x^2+x+1)(x^2+x-1)$.

\item \emph{Determine the splitting field and its degree over $\Q$ for $x^6 - 4$.}

Similarly, adjoining $\sqrt[3]{2} \zeta$ where $\zeta$ is a primitive third root of unity to $\Q$ splits this polynomial, which itself already factors over $\Q$ into $x^3+2$ and $x^3-2$.  As $\sqrt[3]{2} \zeta$ has a minimal polynomial of degree $3$, so does the splitting field over $\Q$.

\end{enumerate}
\end{document}
