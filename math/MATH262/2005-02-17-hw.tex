\documentclass[10pt]{article}

\usepackage{amsfonts}
\usepackage{amsmath}
\usepackage{amssymb}
\usepackage{amsthm}
\usepackage{eucal}
\usepackage{enumerate}
\usepackage{geometry}

\geometry{letterpaper}

\textwidth = 6.5 in
\textheight = 9 in
\oddsidemargin = 0.0 in
\evensidemargin = 0.0 in
\topmargin = 0.0 in
\headheight = 0.0 in
\headsep = 0.0 in
\parskip = 0.2in
\parindent = 0.0in

\newcommand{\brac}[1]{
\left\langle #1 \right\rangle
}

\newcommand{\powset}[1]{
\wp\left(#1\right)
}

\newcommand{\Aut}{\text{Aut}}
\newcommand{\Sym}{\text{Sym}}
\newcommand{\Syl}{\text{Syl}}

\newcommand{\N}{\mathbb{N}}
\newcommand{\Z}{\mathbb{Z}}
\newcommand{\Q}{\mathbb{Q}}
\newcommand{\R}{\mathbb{R}}
\newcommand{\A}{\mathbb{A}}
\newcommand{\C}{\mathbb{C}}

\newcommand{\T}{\mathcal{T}}

\newcommand{\iso}{\cong}

\newtheorem{lemma}{Lemma}

\title{MATH 262: Homework \#5}
\author{Jesse Farmer}
\date{17 February 2005}
\begin{document}
\maketitle
\begin{enumerate}

\item
\begin{enumerate}
\item \emph{Show that no two of the spaces $(0,1)$, $(0,1]$, and $[0,1]$ are homeomorphic.}

Consider the separating properties of these sets.  Removing any point from $(0,1)$ separates it, but it is possible to remove a point from $(0,1]$ and $[0,1]$ so that they both remain connected, viz., $\{1\}$.  Likewise, the removal of any two points separates $(0,1]$, but it is possible to remove two points from $[0,1]$ and still remain connected, viz., $\{0,1\}$.  Therefore none of these spaces are homeomorphic to any of the others.

\item \emph{Suppose that there exist imbeddings $f: X \rightarrow Y$ and $g: Y \rightarrow X$.  Show by means of an example that $X$ and $Y$ need not be homeomorphic.}

Let $f(x) = \frac{1}{2}x + \frac{1}{4}$, $X = [0,1]$, and $Y = (0,1)$.  Then $f(X) = \left[\frac{1}{4}, \frac{3}{4}\right] \subset Y$.  $f$ is clearly an imbedding of $X$ in $Y$.  Similarly, let $g(x) = x$, then $g(Y) = (0,1) \subset [0,1]$ is an imbedding of $Y$ in $X$.  From the previous part $X$ and $Y$ are not homeomorphic.

\item \emph{Show that $\R^n$ and $\R$ are not homeomorphic if $n > 1$.}

The removal of any one point of $\R$ separates it, but, this is not the case with $\R^n$ for $n > 1$.  Indeed, not only is $\R^n \setminus \{a\}$ still connected, but it is path-connected.  For any two points $x,y \in \R^n \setminus \{a\}$, if $a$ is not on the line connecting $x$ and $y$ then that line connects $x$ and $y$.  Otherwise, if $a$ is on that line, pick another point $z \in \R^n \setminus \{a\}$ distinct from $x$ and $y$.  Then the line connecting $x$ to $z$ joined with the line connecting $z$ to $y$ connects $x$ and $y$.  Hence $\R^n$ and $\R$ are not homeomorphic for $n > 1$.

\end{enumerate}

\item \emph{Let $f: X \rightarrow X$ be continuous.  Show that if $X = [0,1]$ there is a point $x$ such that $f(x) = x$,  What happens if $X$ equals $[0,1)$ or $(0,1)$?}

If $f(0) = 0$ or $f(1) = 1$ then we are done, so assume $f(0) > 0$ and $f(1) < 1$.  Define $g(x) = f(x) - x$.  Then $g(1) < 0 < g(0)$, and by the intermediate value theorm there exists some $x$ such that $g(x) = 0$, which implies that $f(x) = x$.

If $X$ is $[0,1)$ or $(0,1)$ then this is not necessarily true because neither of these sets are compact.  For example, $x \mapsto \frac{1}{2}(x+1)$ is a function which is continuous on $X$ but has no fixed point there.

\item
\begin{enumerate}
\item \emph{Let $X$ and $Y$ be ordered sets in the order topology.  Show that if $f: X \rightarrow Y$ is order preserving and surjective then $f$ is a homeomorphism.}

If $f$ is order preserving then $x < y$ implies $f(x) < f(y)$.  $f$ is an injection since if $f(x) = f(y)$ then it cannot be the case that either $x < y$ or $y < x$, since then $f(x) \neq f(y)$.  Similarly, if $f(x) < f(y)$ then $x < y$, by the same argument \emph{mutatis mutandis}.  Hence $f$ is a bijection and its inverse is also order-preserving.  It is therefore sufficient to show that $f$ is an open map, since it then follows that $f^{-1}$ is also an open map and hence that $f$ is a homeomorphism.  This is fairly obvious since the order-preserving property of $f$ guarantees that the image of a basis element $(x,y)$ under $f$ is $(f(x), f(y))$, which is still a basis element.  $f$ preserves unions since it is a bijection, and therefore the image of an open set under $f$ is an open set.


\item \emph{Let $X = Y = \bar{\R}_+$.  Given a positive integer $n$, show that the function $f(x) = x^n$ is order preserving and surjective.  Conclude that its inverse is continuous.}

If $x^n$ is order-preserving and surjective then by the previous part it is a homeomorphism, which by definition means that its inverse is continuous.  If $x < y$ where $x > 0$, then $x^2 < xy < y^2$.  Inductively it follows that $x^n < y^n$ for all positive integers $n$.  This map is obviously surjective as $\sqrt[n]{x}$ is a well-defined real number for all $x > 0$ and $n \in \Z_+$.

\item \emph{Let $X$ be the subspace $(-\infty, -1) \cup [0, \infty)$ of $\R$.  Show that the function $f: X \rightarrow \R$ defined by setting $f(x) = x+1$ if $x < -1$ and $f(x) = x$ if $x \geq 0$ is an order-preserving surjection.  Is $f$ a homeomorphism?  Compare with $(a)$.}

Let $A = (-\infty, -1)$ and $B = [0, \infty)$.  If $x,y \in A$ then $f(x) = x+1 < y+1$, and similarly for $B$.  If $x \in A$ and $y \in B$ then $x < y$ and $f(x) < 0$ and $f(y) \geq 0$, so that $f(x) < f(y)$.  Let $a \in \R$.  If $a \geq 0$ then $f(a) = a$.  Otherwise, $f(a-1) = a$, so $f$ is surjective.  $f$ cannot be a homeomorphism because $A \cup B$ is disconnected, but this is precisely what makes the implication fail, i.e., there is no guarantee that $(a)$ holds if $X$ is not connected.

\end{enumerate}

\item \emph{What are the components and path components of $\R_l$?  What are the continuous maps $f: \R \rightarrow \R_l$?}

\begin{lemma}\label{lower_lim_disconnected}$\R_l$ is totally disconnected.
\end{lemma}
\begin{proof}
Let $A \subset \R_l$ be a connected subset such that there are at least two distinct points $x,y \in A$ and assume without loss of generality that $x < y$.  $(-\infty, y)$ and $[y, \infty)$ are both closed and open in $\R_l$.  $A \setminus (-\infty, y)$ and $A \setminus [y, \infty)$ therefore separate $A$, i.e., $A$ is not connected.
\end{proof}

From this lemma is follows directly that the only components and therefore path-components are the singletons of $\R_l$.  If $f$ is continuous then $f(\R)$ must be connected since $\R$ is connected.  Since the only connected sets are the singletons, it follows that $f$ must be constant.  Every constant function is continuous, and therefore $f: \R \rightarrow \R_l$ is continuous if and only if it is constant.

\item
\begin{enumerate}
\item \emph{What are the components and path components of $\R^\omega$ in the product topology?}

\begin{lemma}\label{prod_path_connected} Let $\{X_\alpha\}$ be a family of path-connected spaces indexed by an arbitrary set $J$.  Then the Cartesian product of all these spaces is also path-connected under the product topology.
\end{lemma}
\begin{proof}Let $X = \prod_{\alpha \in J} X_\alpha$.  Let $x = (x_\alpha)$ and $y = (y_\alpha)$ be two points of $X$.  By hypothesis there exists a continuous function $\gamma_\alpha: [0,1] \rightarrow X_\alpha$ for each $\alpha \in J$ such that $\gamma_\alpha(0) = x_\alpha$ and $\gamma_\alpha(1) = y_\alpha$.  Define $\gamma = (\gamma_\alpha)_{\alpha \in J}$.  This function is continuous from the properties of the product topology, and by construction connects $x$ and $y$.
\end{proof}

Since $\R$ is path-connected it follows directly from the above lemma that $\R^\omega$ is path-connected when endowed with the product topology, and hence the only path-component (and component) is $\R^\omega$.

\item \emph{Consider $\R^\omega$ with the uniform topology.  Show that $\vec{x}$ and $\vec{y}$ lie in the same component if and only if the sequence $$\vec{x} - \vec{y} = (x_1-y_1, x_2-y_2,  \ldots)$$ is bounded.}

\begin{lemma}\label{uniform_disconnected}Let $\R^\omega$ have the uniform topology.  Let $A$ be the set of all bounded sequences and let $B$ be the set of all unbounded sequences.  Then $A$ and $B$ separate $\R^\omega$.
\end{lemma}
\begin{proof}Let $\overline{\rho}$ be the uniform metric on $\R^\omega$.  Let $\vec{x} = (x_1, x_2, \ldots) \in A$ and $\vec{y} = (y_1, y_2, \ldots) \in B$  There exists some real $R > 0$ such that $|x_n| < R$ for all $n \in \N$ by hypothesis, and, since $y$ is unbounded, there must exist some $k \in \N$ such that $|y_k| > R + 1$.  But then $|x_n - y_k| > 1$ and $\overline{\rho}(\vec{x},\vec{y}) = 1$.  Therefore, for any $r < 1$, the ball of radius $r$ about a point in one of $A$ or $B$ does not contain any elements of the other set and hence both are open.  Moreover, since $A = \R^\omega \setminus B$, it follows that $A$ and $B$ separate $\R^\omega$.
\end{proof}

It is sufficient to consider the case where $\vec{y} = 0$ since $\vec{x}$ and $\vec{y}$ lie in the same component if and only if $\vec{x}-\vec{y}$ and $0$ lie in the same component, and $\vec{x} \mapsto \vec{x} - \vec{y}$ is a homeomorphism of $\R^\omega$ for fixed $\vec{y}$

Assume $\vec{x}$ is bounded, then $|x_n| < R$ for some real $R > 0$ and all $n \in \N$.  Define $$\gamma(t) = t\vec{x} = (tx_1, tx_2, \ldots)$$  For any $\epsilon > 0$ let $\delta = \frac{\epsilon}{R}$.  Then $\gamma(B_\delta(t)) \subset B_\epsilon(\gamma(t))$, and hence $\gamma$ is continuous.  Thus $\gamma$ connects $0$ and $\vec{x}$, so that they must lie in the same path component and hence the same component.

Conversely, if $\vec{x}$ is unbounded then by Lemma \ref{uniform_disconnected} it is in a different component from $0$.

\item \emph{Consider $\R^\omega$ with the box topology.  Show that $\vec{x}$ and $\vec{y}$ lie in the same component of $\R^\omega$ if and only if the sequence $\vec{x} - \vec{y}$ is eventually zero.}

As in the previous exercise we may assume that $\vec{y} = 0$.  If $\vec{x}$ is eventually $0$ then there exists some $N \in \N$ such that $\vec{x} \in \R^N \times \{0\} \times \{0\} \times \cdots \subset \R^\omega$, which is homeomorphic to $\R^N$.  Since we know $\R^N$ is connected, it follows that $\vec{x}$ and $0$ are in the same component.

Let $\vec{x}$ have infinitely many non-zero terms.  Define $h: \R^\omega \rightarrow \R^\omega$ by $$h_n(\vec{z}) = \begin{cases} s_n z_n & x_n \neq 0 \\ z_n & x_n = 0 \end{cases}$$ where $s_n$ is a sequence of real numbers such that $s_n|x_n| \rightarrow \infty$ as $n \rightarrow \infty$, e.g., $s_n = \frac{n}{|x_n|}$.  This map is a bijection, and also a homeomorphism.  Consider the open set $U_1 \times U_2 \times \cdots$.  Then $f_n(U_n)$ is either $U_n$ if $x_n = 0$, or $s_nU_n$ if $x_n \neq 0$, but both of these are open.  It is the same for $f^{-1}$, so that $h$ is a homeomorphism.

Moreover, $h(0) = 0$ and $h(\vec{x})$ is unbounded, so that they lie in different components from the previous part (the box topology is finer than the uniform topology).  Since $h$ is a homeomorphism $\vec{x}$ and $0$ must be in different components.

\end{enumerate}

\item \emph{Let $X$ be locally path connected.  Show that every connected open set in $X$ is path connected.}

Let $A \subset X$ be open and connected and fix $a \in A$.  Define $$P_a = \{x \in A \mid \mbox{There exists a path connecting $x$ to $a$}\}$$

To show that $A$ is path connected it is sufficient to show that $P_a$ is both open and closed in $A$ since $A$ is connected.  To show that $P_a$ is open let $y \in A$.  Then by hypothesis there exists a neighborhood in $A$ of $y$, call it $U_y$, which is path-connected.  But then $P_a = \bigcup_{y \in A} U_y$, and hence $P_a$ is open.

Likewise, to show that $P_a$ is closed, let $y \in A$ be a limit point of $P_a$.  Then there exists a path-connected neighborhood of $y$, $U_y$, which intersects with $P_a$.  Let $z \in P_a \cap U_y$.  There is a path connecting any $x \in P_a$ to $z$, and, since $U_y$ is path connected, another path connecting $z$ to $y$, and therefore a path connecting $x$ to $y$, viz., the combination of these two paths.  Hence $y \in P_a$, and $P_a$ is closed.  It follows that $P_a = A$.

\item 
\begin{enumerate}
\item \emph{Let $\T$ and $\T'$ be two topologies on the set $X$.  If $\T'$ is finer than $\T$ what does the compactness of $X$ under one of these topologies imply about compactness under the other?}

If $X$ is compact with respect to $\T'$ then it is compact with respect to $\T$.  This is obvious since any cover $\mathcal{C}$ in $\T$ is also a cover in $\T'$, and hence has a finite subcover.  The converse does not hold.  As an example, let $X = [0,1]$.  If $\T'$ is the discrete topology and $\T$ is the usual topology, then we know that $X$ is compact with respect to $\T$.  However, it is not compact with respect to $\T'$ since the union of all the singletons is a cover of $X$ with no finite subcover.

\item \emph{Show that if $X$ is compact Hausdorff under both $\T$ and $\T'$ then either $\T = \T'$ or they are not comparable.}

It is sufficient to show that if $\T$ and $\T'$ are comparable then they are equal.  Assume $\T$ and $\T'$ are comparable and $\T' \subset \T$ without loss of generality.  Denote by $X'$ and $X$ the same underlying set $X$ with the topologies $\T'$ and $\T$ respectively.  Then the inclusion map $i: X' \rightarrow X$ is a continuous bijection.  Since $X$ is compact Hausdorff under both $\T$ and $\T'$, by Theorem 26.6, it follows that $i$ is a homeomorphism and hence $\T = \T'$.

\end{enumerate}

\item \emph{Let $A$ and $B$ be disjoint compact subspaces of the Hausdorff space $X$.  Show that there exist disjoint open sets $U$ and $V$ containing $A$ and $B$ respectively.}
\begin{lemma}\label{compact_hausdorff}Let $Y$ be a compact subspace of the Hausdorff space $X$.  If $x \in X \setminus Y$ then there exist disjoint neighborhoods $U$ and $V$ containing $x$ and $Y$, respectively.
\end{lemma}
\begin{proof}Let $x \in X \setminus Y$.  For every $y \in Y$ there exist disjoint neighborhoods $U_y$ and $V_y$ of $x$ and $y$, respectively.  Then $\bigcup_{y \in Y} V_y$ is a covering of $Y$.  Since $X$ is compact there exist $V_{y_1}, \ldots, V_{y_n}$  such that $V = \bigcup_{i=1}^n V_i$ covers $Y$.  Let $U = \bigcap_{i=1}^n U_{y_i}$.  Then $U$ is open, $x \in U$ since $x \in U_{y_i}$ for every $i$, and $U \cap V = \emptyset$.
\end{proof}

Let $A$ and $B$ be disjoint compact subspaces of the Hausdorff space $X$ and let $x \in A$ be arbitrary.  From the lemma there exist $U_x$ and $V_x$ containing $x$ and $B$, respectively.  As in the lemma, $\{U_x\}$ cover $A$, and so there is a finite subset that also covers $A$, call it $U_{x_1}, \ldots, U_{x_n}$ and their union $U$.  Then the intersection of the corresponding $\{V_{x_i}\}$ is open, contains $B$, and is disjoint from $U$.

\item \emph{Show that if $Y$ is compact then the projection $\pi: X \times Y \rightarrow X$ is a closed map.}

Let $A \subset X \times Y$ be closed.  We want to show that $\pi(A)$ is closed.  Let $x \in X \setminus \pi(A)$.  Then $$\pi^{-1}(x) = x \times Y \subset (X \times Y) \setminus A$$ and therefore $\pi^{-1}(A)$ is open.  By Lemma 26.8 (the tube lemma) there exists a $W \subset X$ such that $x \times W$ and $W \times Y \subset (X \times Y) \setminus A$.  Hence $W \cap \pi(A) = \emptyset$ and $x \in W \subset X \setminus \pi(A)$, so that $X \setminus \pi(A)$ is open, i.e., $\pi(A)$ is closed.

\item \emph{Let $f: X \rightarrow Y$ and $Y$ be compact Hausdorff.  Show that $f$ is continuous if and only if the \textbf{graph} of $f$ $$G_f = \{x \times f(x) \mid x \in X\}$$ is closed in $X \times Y$.}

To begin, assume that $f$ is continuous.  We will show that $(X \times Y) \setminus G_f$ is open.  Let $(x,y) \in (X \times Y) \setminus G_f$ so that $y \neq f(x)$.  Since $Y$ is Hausdorff there exist disjoint neighborhoods $U$ and $V$ of $f(x)$ and $y$, respectively.  It follows that $f^{-1}(U) \cap f^{-1}(V) = \emptyset$ and $f^{-1}(U) \times V \subset (X \times Y) \setminus G_f$.  Since $f$ is continuous $f^{-1}(U)$ is open, and therefore $(X \times Y) \setminus G_f$ is open, i.e, $G_f$ is closed.

Now suppose $G_f$ is closed.  Let $V \subset Y$ be open.  Since $(X \times Y) \setminus (X \times (Y \setminus V)) = X \times V$ is open, $X \times (Y \setminus V)$ is closed.  Then $A = G_f \cap \left(X \times (Y \setminus V)\right)$ is also closed.  But
\[
A = \{x \times f(x) \mid f(x) \in Y \setminus V\}
\]

From the previous problem $\pi(A) = \{x \mid x \in Y \setminus V\} = Y \setminus f^{-1}(V)$ is closed, and hence $f^{-1}(V)$ is open and $f$ is continuous.
\end{enumerate}
\end{document}
