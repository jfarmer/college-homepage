\documentclass[11pt]{article}

\usepackage{amsfonts}
\usepackage{amsmath}
\usepackage{amssymb}
\usepackage{eucal}
\usepackage{enumerate}
\usepackage{geometry}

\geometry{letterpaper}

\textwidth = 6.5 in
\textheight = 9 in
\oddsidemargin = 0.0 in
\evensidemargin = 0.0 in
\topmargin = 0.0 in
\headheight = 0.0 in
\headsep = 0.0 in
\parskip = 0.2in
\parindent = 0.0in

\newcommand{\brac}[1]{
\left\langle #1 \right\rangle
}

\newcommand{\powset}[1]{
\wp\left(#1\right)
}

\newcommand{\Aut}{\text{Aut}}
\newcommand{\Sym}{\text{Sym}}
\newcommand{\Syl}{\text{Syl}}

\newcommand{\N}{\mathbb{N}}
\newcommand{\Z}{\mathbb{Z}}
\newcommand{\Q}{\mathbb{Q}}
\newcommand{\R}{\mathbb{R}}
\newcommand{\A}{\mathbb{A}}
\newcommand{\C}{\mathbb{C}}

\newcommand{\T}{\mathcal{T}}

\title{MATH 262: Homework \#3}
\author{Jesse Farmer}
\date{27 January 2005}
\begin{document}
\maketitle
\begin{enumerate}

\item \emph{For the following topologies on $\R$ determine which of the others it contains.  The standard topology, $\R_K$, the finite complement topology, the upper-limit topology, and the topology generated by the basis $(-\infty, a)$ for $a \in \R$.}

Denote these topologies as $\T_1, \T_2, \T_3, \T_4, \T_5$, respectively.  

$\T_1$ contains $\T_3$ since any finite set is closed in $\T_1$ and hence the complement is open.  $\T_1$ also contains $\T_5$ since every basis element of $\T_5$ is also a basis element of $\T_1$.

$\T_2$ contains $\T_1$ since every basis element of the latter is also a basis element of the former.  From above, it also contains $\T_3$ and $\T_5$.

$\T_3$ contains none of the other topologies.

$\T_4$ contains $\T_1$, and hence contains $\T_3$ and $\T_5$.  $\T_4$ also contains $\T_2$ since
\[
\R \setminus K = (-\infty, 0] \cup \bigcup_{n \in \Z_+} \left(\frac{1}{n+1}, \frac{1}{n}\right) \cup (1, \infty)
\]

is open in $\T_4$.  Any set open set $U$ in $\T_2$ is already open in $\T_1$ and hence open in $\T_4$, or is of the form $U \setminus K$ where $U$ is open in the standard topology.  In either case, since $\T_4$ contains both $\T_1$ and $\R \setminus K$, $U$ is open in $\T_4$.

Finally $\T_5$ contains none of the other topologies.
\item \emph{If $\T$ and $\T'$ are topologies on $X$ and $\T'$ is strictly finer than $\T$, what can you say about the corresponding subspace topologies on the subset $Y$ of $X$?}

Let $\T \subset \T'$ and denote the subspace topology inherited by $Y$ from $\T$ and $\T$ as $\T_Y$ and $\T'_Y$, respectively.  Then if $Y \cap U \in \T_Y$, $Y \cap U \in \T'_Y$ since $U \in \T \subset \T'$.  It need not be strictly finer, however.  Consider, for example, the standard topology on $\R$ and $\R_l$ restricted to the interval $[0,1]$.

\item \emph{If $L$ is a straight line in the plane, describe the topology $L$ inherits as a subspace of $\R_l \times \R$ and as a subspace of $\R_l \times \R_l$.}

In the first case, either $L$ is vertical or it is not.  If it is then the subspace topology on $L$ is simply the standard topology on $\R$.  If $L$ is not vertical then the topology is the lower-limit topology, since in this case $L$ can intersect the closed edge of the basis ``rectangle'' $[a,b) \times (c,d)$ for some $a,b,c,d$.

In the seconds case, either $L$ has positive slope or is vertical, or $L$ has negative slope.  In the first case the topology is again the lower-limit topology since $L$ will only intersect at most one of the closed edges.  If $L$ has negative slope then the topology is the discrete topology since for any $(x,y) \in L$, $[x,x+1) \times [y,y+1)$ is a basis element containing $(x,y)$.  If every point is open, then the topology is necessarily the discrete topology.

\item \emph{Let $I = [0,1]$.  Compare the product topology on $I \times I$, the dictionary order topology on $I \times I$, and the topology $I$ inherits as a subspace of $\R \times \R$ in the dictionary order topology.}

Denote these topologies as $\T_p$, $\T_o$, and $\T_s$, respectively, and note that $\T_p$ is the same as the subspace topology of $I \times I$ when $\R^2$ is given the topology induced by the usual metric, i.e., the $l^2$ norm.  We will denote the pair $(x,y)$ by $x \times y$ to avoid ambiguity.

We claim that $\T_p \subsetneq \T_s$.  Let $x \times y \in U \times B$, where $U \times V$ is a basis element of $\T_p$. 
 Then $x \times y \in \{x\} \times V \subset U \times V$ and $\{x\} \times V$ is a basis element of $\T_s$, hence $\T_p \subset \T_s$.  $\{1/2\} \times [0,1]$ is an element of $\T_s$ not contained in $\T_p$, so the inclusion is proper.

We claim that $\T_o \subsetneq \T_s$.  Every basis element of $\T_o$ is in $\T_s$, but, for example $\{1/2\} \times (0,1]$ is not in $\T_o$ since there is no basis element of the form $\{1/2\} \times (a,b)$ containing it as $b \leq 1$ by necessity.

We claim that $\T_p$ and $\T_o$ are not comparable.  As in the first case, $\{1/2\} \times (0,1) \in \T_o$ but is not in $\T_p$.  Next consider $\T_p$ and $\T_o$ neighborhoods around some point on the top edge of the box, e.g., $1/2 \times 1$.  Then any $\T_o$ neighborhood contains points of the form $x \times 0$ where $x > 1/2$, but it is easy to see that the open ball of radius $1/2$ around $1/2 \times 1$ intersected with $I \times I$, which is open in $\T_p$, contains no such point.

\item \emph{Let $A, B$ be subsets of a space $X$, and $\{A_\alpha\}$ a family of subsets in $X$.  Prove the following:}
\begin{enumerate}
\item \emph{If $A \subset B$ then $\bar{A} \subset \bar{B}$.}

$x \in \overline{A \cup B}$ if and only if every neighborhood $U$ of $x$ intersects $A \cup B$, i.e., $(A \cup B) \cap U \neq \emptyset$.  But $(A \cup B) \cap U = (A \cap U) \cup (B \cap U)$, so that $U$ intersects $A$ or $U$ intersects $B$.  This is the case if and only if $x \in \bar{A} \cup \bar{B}$.

\item \emph{$\overline{A \cup B} = \bar{A} \cup \bar{B}$.}

As above, $x \in \overline{A \cup B}$ if and only if every neighborhood $U$ of $x$ intersects $A \cap B$, i.e., $A \cap B \cap U \neq \emptyset$.  But $A \cap B \cap U = (A \cap U) \cap (B \cap U)$, so that both $A \cap U$ and $B \cap U$ must be nonempty.  This is the case if and only if $x \in \bar{A} \cap \bar{B}$.

\item \emph{$\bigcup\bar{A}_\alpha \subset \overline{\bigcup A_\alpha}$.}

If $x \in \bigcup\bar{A}_\alpha$ then every neighborhood $U$ of $x$ intersects some $A_\alpha$.  This means that for any neighborhood $U$, $U \cap \bigcup\bar{A}_\alpha =  \bigcup\bar{A}_\alpha \cap U \neq \emptyset$, i.e., $x \in \overline{\bigcup A_\alpha}$.

\end{enumerate}

\item \emph{Criticize the following proof that $\overline{\bigcup A_\alpha} \subset \bigcup\bar{A}_\alpha$.  If $\{A_\alpha\}$ is a collection of sets in $X$ and if $x \in \overline{\bigcup A_\alpha}$ then every neighborhood of $U$ intersects $\bigcup A_\alpha$.  Thus $U$ must intersect some $A_\alpha$, so that $x$ must belong to the closure of some $A_\alpha$.  Therefore $x \in \bigcup \bar{A}_\alpha$.}

This proof assumes that there exists an $\alpha$ such that for any neighborhood $U$, $U \cap A_\alpha \neq \emptyset$, whereas it is actually the case that for every neighborhood there is \emph{some} $A_\alpha$ such that $U \cap A_\alpha \neq \emptyset$.  As an example, consider $\Q \subset \R$ with the usual topology.  The union of all the singletons in $\Q$ is $\Q$, whose closure is $\R$.  But the union of the closure of each singleton is just $\Q$, since points are closed in the usual topology.

\item \emph{Show that $X$ is Hausdorff if and only if $\triangle = \{x \times x \mid x \in X\}$ is closed in $X \times X$.}

Assume $\triangle$ is closed, then $(X \times X) \setminus \triangle$ is open, i.e., for any $(x,y) \in (X \times X) \setminus \triangle$ there exists a basis element $U \times V$, where $U$ and $V$ are open in $X$, such that $(x,y) \in U \times V$.  But $U \bigcap V = \emptyset$, since if $x \in U \cap V$ then $(x,x) \in U \times V$, i.e., $U \times V$ intersects $\triangle$.  Hence $x \in U$, $y \in V$, and $U \bigcap V = \emptyset$, and $X$ is Hausdorff.

Similarly, assume $X$ is Hausdorff.  Then for any distinct $x,y \in X$ there exist disjoint neighborhoods $U,V$ of $x$ and $y$, respectively.  Then $(x,y)$ is in some basis element for $X \times X$, namely, $U \times V$.  Since $U$ and $V$ are disjoint, $(U \times V) \setminus \triangle = \emptyset$, and therefore $(X \times X)\setminus \triangle$ is open, i.e., $\triangle$ is closed.

\item \emph{Let $X$ and $X'$ denote a single set in the two topologies $\T$ and $\T'$, respectively.  Let $i: X' \rightarrow X$ be the identity function.}
\begin{enumerate}
\item \emph{Show that $i$ is continuous if and only if $\T'$ is finer than $\T$.}

Assume $i$ is continuous and let $U \in \T$.  Then $U = i(U) = i^{-1}(U) \in \T'$ and hence $\T'$ is finer than $\T$.  Assume $\T'$ is finer than $\T$ and let $U \in T$.  Then $U = i(U) = i^{-1}(U) \in \T \subset \T'$ and hence $i$ is continuous.

\item \emph{Show that $i$ is a homeomorphism if and only if $\T' = \T$.}

$i$ is a homeomorphism if and only if it is bijective, continuous, and has a continuous inverse.  Since $i^{-1} = i$, from the previous part, we have $i$ is a homeomorphism if and only if $\T'$ is finer than $\T$ and $\T$ is finer than $\T'$, i.e., if and only if $\T' = \T$.

\end{enumerate}

\item \emph{Find a function $f: \R \rightarrow \R$ that is continuous at precisely one point.}

Define $$f(x) = \begin{cases} x & x \in \Q \\ 0 & x \notin \Q \end{cases}$$

The density of the rationals in the irrationals and vice versa guarantees that this function is continuous at nowhere except $x=0$.

\item \emph{Let $Y$ be an ordered set in the order topology.  Let $f,g: X \rightarrow Y$ be continuous.}
\begin{enumerate}
\item \emph{Show that the set $\{x \mid f(x) \leq g(x)\}$ is closed in $X$.}

First, a small lemma: every order topology is Hausdorff.  Let $x,y \in X$ be distinct where $X$ has the order topology.  Then either there is some third element $a \in X$ with $x < a < y$, in which case the basis elements $(-\infty,a)$ and $(a, \infty)$ are disjoint neighborhoods containing $x$ and $y$, respectively.  If there is no such element then $(-\infty, y)$ and $(x, \infty)$ are disjoint basis elements containing $x$ and $y$, respectively.  Either way, $X$ is Hausdorff.

Let $x \in X$ be such that $f(x) > g(x)$.  Then, since $Y$ is Hausdorff, there exist disjoint neighborhoods $U$ and $V$ of $f(x)$ and $g(x)$.  Since these neighborhoods are disjoint and $Y$ has the order topology, then every element of $U$ is greater than every element of $V$, as there is one element of $U$, $f(x)$, greater than one element of $V$, $g(x)$.  Furthermore, the continuity of $f$ implies that $W = f^{-1}(U) \bigcap f^{-1}(V)$ is an open neighborhood of $x \in X$ such that $f(w) > g(w)$ for all $w \in W$.  Hence $\{x \mid f(x) > g(x)\}$ is open in $X$, and therefore $\{x \mid f(x) \leq g(x)\}$ is closed in $X$.

\item \emph{Show that $h(x) = \min\{f(x), g(x)\}$ is continuous.}

Let $A = \{x \mid f(x) \leq g(x)\}$ and $B = \{x \mid f(x) \geq g(x)\}$.  By the previous part both of these are closed, and $A \bigcap B$ is precisely the set where $f = g$.  From Theorem 18.3 (the pasting lemma) it follows that $h$ is continuous on $A \cup B = X$.

\end{enumerate}

\item \emph{Let $A \subset X$ and $f: A \rightarrow Y$ continuous, where $Y$ is Hausdorff.  Show that any continuous extension of $f$ to $\bar{A}$ is unique.}

Let $f,g: Z \rightarrow Y$ be any two continuous functions and $Y$ be Hausdorff.  Then the set
\[
C = \{x \in Z \mid f(x) = g(x)\}
\]

is closed.  This follows from the fact that $x \in X \setminus C$ then there exist disjoint neighborhoods $U, V$ of $f(x)$ and $g(x)$ respectively, and $x \in f^{-1}(U) \cup f^{-1}(V)$, i.e., $X \setminus C$ is open.

Let $g, g'$ be two continuous extensions of $f$ to $\bar{A}$, where $f: A \rightarrow Y$ is continuous.  Then $C$ from above is closed, using $Z = \bar{A}$.  But this set contains $A$ and hence contains $\bar{A}$, since $\bar{A}$ is by definition the smallest closed set containing $A$.  Hence $g$ and $g'$ agree on $\bar{A}$.

\end{enumerate}
\end{document}
